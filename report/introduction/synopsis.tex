This project covers the modelling and predictive control of a water distribution network with the aim of minimizing energy and economic cost. 
\newline
At first the non-linear model of the components and the dynamics of the system are modelled based on a graph-based approach which leads to a state space representation of the whole network. Then system identification is carried out due to the uncertain parameters of the pipe components. 
\newline
A model predictive controller is applied to the linearized model of the water distribution system extended with an elevation reservoir. The controller follows certain constraints to maintain consumer pressure-desire in two pressure management areas and to optimize the use of water tower such that the cost of pumping effort is minimized. 
The controller is implemented in a cascade system along with PI controllers and is based on the model of the network, the cost of electricity and the characteristics of end-user water usage. 
\newline
Implementation carried out ...
\newline
The results show that ...