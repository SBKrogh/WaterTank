\chapter{Explanation of Notation}

\section*{Acronyms}
	
	\begin{tabular}{|l c l|} \hline
		PMA 		&&	Pressure Management Area		\\ \hline
		CP			&&	Critical Point					\\ \hline
		WT			&&	Water Tower						\\ \hline
		MP		  	&&	Minimization Problem			\\ \hline
		OD 			&&  Opening Degree					\\ \hline
		KVL 		&&  Kirchhoff´s Voltage Law 		\\ \hline
		KCL 		&&  Kirchhoff´s Current Law			\\ \hline
		MPC			&&  Model Predictive Control		\\ \hline
		GT			&&  Graph Theory					\\ \hline
		ZOH			&&  Zero Order Hold					\\ \hline
	\end{tabular}

\section*{Symbols}


\begin{tabular}{l l l} 
	\textbf{Symbol}	&	\textbf{Description}										& \textbf{Unit}	\\\hline
	$A$				&	Cross sectional area									    & [$m^2$]\\
	$C_k$			&	The $k^{th}$ component of the distribution network     		& [$\cdot$]\\
	$C$				&	Electric capacitance								   		& [$F$]\\
	$C_H$			&	Hydraulic capacitance								   		& [$m^3/(N/m^2)$]\\
	$D$				&	Diameter									     			& [$m$]\\
	$f$				&	Moody friction factor									    & [$\cdot$]\\
	$F$				&	Force									     				& [$N$]\\
	$g$				&	Acceleration due to gravity									& [$m/s^2$]\\
	$h_f$			&	Pressure given in head									    & [$m$]\\
	$h_m$			&	Form loss									    			& [$m$]\\
	$J_k$			&	Water inertia of the $k^{th}$ component						& [$kg/m^4$]\\
	$k_f$			&	Form loss coefficient								    	& [$\cdot$]\\
	$L$				&	Length									     				& [$m$]\\
	$m$				&	Mass of body									     		& [$kg$]\\
	$M$				&	Linear momentum									     		& [$kgm/s$]\\
	$n_i$			&	The $i^{th}$ node of the distribution network     			& [$\cdot$]\\
	$n_{gl}$		&	Valve characteristic curve factor							& [$\cdot$]\\
	$p_a$			&	Atmospheric pressure							      		& [$bar$]\\
	$\Delta p_k$	&	The pressure drop across the $i^{th}$ component      		& [$bar$]\\
	$q_k$			&	Flow through the $k^{th}$ component 		     			& [$m^3/h$]\\
	$\pmb{Re}$		&	Reynolds Number								     			& [$\cdot$]\\
	$T$				&	Temperature									     			& [$\degree$]\\
	$v$				&	Velocity									     			& [$m/s$]\\
	$V_t$			&	Volume of the water in the water tower						& [$m^3$]\\
	$\alpha_k(\cdot)$	&	The pressure boost given by the	$k^{th}$ pump				& [$bar$]\\
	$\epsilon$		&	Average roughness									     	& [$\cdot$]\\
	$\zeta$			&	Pressure drop from elevation difference across the $k^{th}$ component	& [$bar$]\\
	$\theta_{max}$	&	Maximum angle of the opening degree						    & [$\degree$]\\
	$\theta_{off}$	&	Minimum angle where the valve closes						& [$\degree$]\\
	$\theta_{OD}$	&	Angle of opening degree										& [$\degree$]\\
	$\lambda_k(\cdot)$	&	Function of hydraulic resistance in the $k^{th}$ pipe		& [$bar$]\\
	$\mu_k(\cdot)$		&	Function of hydraulic resistance in the $k^{th}$ valve		& [$bar$]\\
	$\nu$			&	Kinematic viscosity									     	& [$kg/ms$]\\
	$\rho$			&	Density									     				& [$kg/m^3$]\\
	$\omega_r$		&	Impeller angular velocity									& [$rad/s$]\\
	
\end{tabular}

\section*{Mathematical tools}
vectorfields
\\
time derivatives
\\
vectors
\\
matrices
\\
derivative of vector fields
\\
Jakobi matrix
\\
chain rule in derivation
\\
pseudo inverse


% We should write somthing about notation: 

\section*{Mathematical notation}

This section will explain how the mathematical notation of this report. 

\textbf{Upper and lower bounds}




\begin{equation}
\underline{x} < x < \overline{x} 
\end{equation}

 Where x is a variable and $\overline{x}$ and $\underline{x}$ are the upper and lower bounds.

% * Matricies /vector notation 
% * Intervals
% * Upper/lower bounds 