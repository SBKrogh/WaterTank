\chapter{Nomenclature}

\section*{Acronyms}
	
	\begin{tabular}{|l c l|} \hline
		PMA 		&&	Pressure Management Area		\\ \hline
		CP			&&	Critical Point					\\ \hline
		WT			&&	Water Tower						\\ \hline
		MP		  	&&	Minimization Problem			\\ \hline
		OD 			&&  Opening Degree					\\ \hline
		KVL 		&&  Kirchhoff´s Voltage Law 		\\ \hline
		KCL 		&&  Kirchhoff´s Current Law			\\ \hline
		MPC			&&  Model Predictive Control		\\ \hline
		GT			&&  Graph Theory					\\ \hline
	\end{tabular}

\section*{Symbols}


\begin{tabular}{l l l} 
	\textbf{Symbol}	&	\textbf{Description}										& \textbf{Unit}	\\\hline
	$A$				&	Cross sectional area									    & [$m^2$]\\
	$C_k$			&	The $k^{th}$ component of the distribution network     		& [$\cdot$]\\
	$C$				&	Electric capacitance								   		& [$F$]\\
	$C_H$			&	Hydraulic capacitance								   		& [$m^3/(N/m^2)$]\\
	$D$				&	Diameter									     			& [$m$]\\
	$f$				&	Moody friction factor									    & [$\cdot$]\\
	$F$				&	Force									     				& [$N$]\\
	$g$				&	Acceleration due to gravity									& [$m/s^2$]\\
	$h_f$			&	Pressure given in head									    & [$m$]\\
	$h_m$			&	Form loss									    			& [$m$]\\
	$J_k$			&	Water inertia of the $k^{th}$ component						& [$kg/m^4$]\\
	$k_f$			&	Form loss coefficient								    	& [$\cdot$]\\
	$L$				&	Length									     				& [$m$]\\
	$m$				&	Mass of body									     		& [$kg$]\\
	$M$				&	Linear momentum									     		& [$kgm/s$]\\
	$n_i$			&	The $i^{th}$ node of the distribution network     			& [$\cdot$]\\
	$n_{gl}$		&	Valve characteristic curve factor							& [$\cdot$]\\
	$p_a$			&	Atmospheric pressure							      		& [$bar$]\\
	$\Delta p_k$	&	The pressure drop across the $i^{th}$ component      		& [$bar$]\\
	$q_k$			&	Flow through the $k^{th}$ component 		     			& [$m^3/h$]\\
	$\pmb{Re}$		&	Reynolds Number								     			& [$\cdot$]\\
	$T$				&	Temperature									     			& [$\degree$]\\
	$v$				&	Velocity									     			& [$m/s$]\\
	$V_t$			&	Volume of the water in the water tower						& [$m^3$]\\
	$\alpha_k(\cdot)$	&	The pressure boost given by the	$k^{th}$ pump				& [$bar$]\\
	$\epsilon$		&	Average roughness									     	& [$\cdot$]\\
	$\zeta$			&	Pressure drop from elevation difference across the $k^{th}$ component	& [$bar$]\\
	$\theta_{max}$	&	Maximum angle of the opening degree						    & [$\degree$]\\
	$\theta_{off}$	&	Minimum angle where the valve closes						& [$\degree$]\\
	$\theta_{OD}$	&	Angle of opening degree										& [$\degree$]\\
	$\lambda_k(\cdot)$	&	Function of hydraulic resistance in the $k^{th}$ pipe		& [$bar$]\\
	$\mu_k(\cdot)$		&	Function of hydraulic resistance in the $k^{th}$ valve		& [$bar$]\\
	$\nu$			&	Kinematic viscosity									     	& [$kg/ms$]\\
	$\rho$			&	Density									     				& [$kg/m^3$]\\
	$\omega_r$		&	Impeller angular velocity									& [$rad/s$]\\
	
\end{tabular}

% \section*{Mathematical tools}
% vectorfields
% \\
% time derivatives
% \\
% vectors
% \\
% matrices
% \\
% derivative of vector fields
% \\
% Jakobi matrix
% \\
% chain rule in derivation
% \\
% pseudo inverse
% \\
% explain that $\bm{x}[k]$ is not iteration, it shows that x is a vector and a sequence-> this is very important to state ! 

% We should write somthing about notation: 
\newpage
\section*{Glossary of mathematical notation}

This section sums up the mathematical notation and terminology used in this report.

\textbf{Upper and lower bounds of a variable}

\begin{equation}
\underline{x} < x < \overline{x} 
\end{equation}

 Where $x \in {\mathbb{R}} $ and $\overline{x}$ and $\underline{x}$ are the upper and lower bounds.
 
 \textbf{Intervals}

\begin{equation}
	[a,b] =  \{x \in \mathbb{R}|a\leq x \leq b|\}
\underline{x} < x < \overline{x} 
\end{equation}

 Where $\overline{x}$ and $\underline{x}$ are the upper and lower bounds.

 \textbf{Vectors and matrices}

Vectors and matrices are noted with bold fonts, such that $\bm{v}$ is a vector:

\begin{equation}
\bm{v} = 
\begin{bmatrix}

		 v_1 	\\
		 v_2 	\\
		 \vdots \\
		 v_n

\end{bmatrix}
\in \pmb{{\mathbb{R}}}^{(n \times 1)}
\end{equation}

and $\bm{M}$ is a matrix:

\begin{equation}
\bm{M} = 
\begin{bmatrix}

		 m_{11} & m_{12} & \hdots & m_{1k}	\\
		 m_{21} & m_{22} & \hdots & m_{2k}	\\
		 \vdots & \vdots & \ddots & \vdots	\\
		 m_{n1} & m_{n2} & \hdots &m_{nk} \\

\end{bmatrix}
\in \pmb{{\mathbb{R}}}^{(n \times k)}
\end{equation}

% Vectorfields are vector-valued functions and noted with greek letters such that:

% \begin{equation}
% \bm{v} = 
% \begin{bmatrix}

% 		 v_1 	\\
% 		 v_2 	\\
% 		 \vdots \\
% 		 v_n

% \end{bmatrix}
% \end{equation}

% $\alpha(\bm{v(t)})$

Continues vector variables are noted with $\bm{v(t)}$ such that:

\begin{equation}
\bm{v(t)} = 
\begin{bmatrix}

		 v_1(t) 	\\
		 v_2(t)	\\
		 \vdots \\
		 v_n(t)

\end{bmatrix}
\in \pmb{{\mathbb{R}}}^{(n \times 1)}
\end{equation}

While discrete vector variables are called as sequences in the report and are noted with $\bm{v[k]}$, such that:

\begin{equation}
\bm{v[k]} = 
\begin{bmatrix}

		 v_1[k] 	\\
		 v_2[k]	\\
		 \vdots \\
		 v_n[k]

\end{bmatrix}
\in \pmb{{\mathbb{R}}}^{(n \times 1)}
\end{equation}

Where $k$ is the time step between two entries of the sequence.

The pseudo inverse of a matrix is denoted with $\bm{{M}^{\dagger}}$.

 \textbf{Small-signal and operating values of signals}

Small-signals are denoted as $\hat{u}$ and the operating point values are noted with $\bar{u}$.

 \textbf{Derivatives}

 The partial derivative of a function is noted with

\begin{equation}
\frac{\partial{f(x,y)}}{\partial{x}}
\end{equation}

% The derivative of a vector by scalar is noted with 

% \begin{equation}
% \frac{\partial{\bm{v}}}{\partial{x}}
% \end{equation}

The derivative of a vector by vector is noted as:

\begin{equation}
\frac{\partial{\bm{v}}}{\partial{\bm{w}}} =
\begin{bmatrix}
    \frac{\partial v_{1}}{\partial w_1} & \frac{\partial v_{1}}{\partial w_2} &  \dots  & \frac{\partial v_{1}}{\partial w_n} \\
    \frac{\partial v_{2}}{\partial w_1} & \frac{\partial v_{2}}{\partial w_2} &  \dots  & \frac{\partial v_{2}}{\partial w_n} \\
    \vdots & \vdots &  \ddots & \vdots \\
    \frac{\partial v_{k}}{\partial w_1} & \frac{\partial v_{k}}{\partial w_2} &  \dots  & \frac{\partial v_{k}}{\partial w_n}
\end{bmatrix}
\end{equation}

If the size is of the vectors $\bm{v}$ and $\bm{w}$ are the same, the resulting matrix is called as a Jacobian matrix in the report.

The time derivative of a function is noted with

\begin{equation}
\dot{f} = \frac{d f(t)}{dt}
\end{equation}

\textbf{Vector fields}

Vector fields are introduced that represent vector valued functions such that:

\begin{equation}
\alpha(\bm{v}) : \pmb{{\mathbb{R}}}^{(n)} \rightarrow \pmb{{\mathbb{R}}}^{(n)} : [v_1, v_2, \hdots, v_n] \rightarrow [\alpha(v_1), \alpha(v_2),\hdots,\alpha(v_n)]
\end{equation}



