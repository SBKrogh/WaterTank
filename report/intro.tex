\chapter{Introduction}

Water pressure management is a vital part of the infrastructure all over the world. It ensures that a positive water pressure is present such that the consumers are supplied with water at all time. Maintaining a minimum pressure in the network is an important task as the end user is ensured a decent water pressure and also for minimizing the risk of contamination in the water system\cite{national2005public}.%, which ensure clean/healthy water to the consumers. %Another important part for the pressure is to make sure that the end consumers also is ensured water. 

Implementing one large pressure pump at the water supply is one solution to this problem. It could pressurize the water to a point such that the end consumers get the pressure needed. However this solution is simple it is not beneficial as the water leakage increases proportional to the pressure\cite{feldman2009aspects}, thus increasing the water losses which leads to a higher energy consumption. Before any water is distributed out to the consumers the water got to be prepared e.g cleaned for filth, bacteria etc and then pressurized. In \cite{appelbaum2002water} it is stated that the U.S use 4 \% of the national energy consumption is used on moving and treating water/wastewater. The estimated world wide water loss is estimated to 30 \%\cite{feldman2009aspects}, thus a high amount of energy is lost on preparing water.\\ 
Another problem that should be highlighted to a high pressure is that a high pressure will increase the ware on the pipes in a system\cite{Watts_water} , which leads to higher maintenances as they got to be replaced more often. This is not an easy task, since the pipes usually is under ground and needs to be dug up which is an expensive part, especially in a city since it also have a negative impact on the traffic flow. 

To reduce the pressure, the water network is separated into subsystems called "pressure management areas" (PMA). !!! Pumps or valves !!!! 
Each subsystem have its own or multiple pressure pumps implemented at critical points, where the pressure drop is too low. This have an positive impact on the pressure as it can be kept at a minimum and reduce the wear on the pipes and water loss. This solution however requires more pumps 







Implementing pressure pumps into the system is one solution to this problem. When the pressure get to low the pressure pumps will be activated to increase the pressure again. If the network is large, this solution is not as beneficial as 

A simple solution to this problem could be implementing a large pump at the water station, which pumped

In the U.S 4 \% of the nations electricity is used on purifying and pumping water to the consumers. 

Before water is send out to the consumers, some steps for preparing the water got to be fulfilled, one of them being purification from filth and bacteria.  




The solution to this problem could be as simple as implementing a pressure pump at the water station, that ensured a pressure high enough such that distribution of water to the end user where sufficient. This simple solution have however some drawbacks. The pressure in the system have a direct correlation to water leakage which end up in a higher increase of water that . The extra increase 


Typically the water network is distributed into sub-systems called "Pressure management areas". These areas is connected to a water supply system, where the water is pressurized to t  




%Therefore if the pressure is not sufficient it can become a health issue for the consumers. The pressure must be high enough to ensure that the end consumers have a certain amount of  

 
\begin{itemize}
\item Water pressure. 

Health, consumer satisfaction
\item Why is it necessary to increase the efficiency in PMAs?

Large amount of energy used to prepare water and to distributing it to the users. 
% http://www.epri.com/abstracts/Pages/ProductAbstract.aspx?ProductId=000000000001006787&Mode=download
\item Leakage 30 \%
\item Disadvantages of high pressure: Water Hammer, leaking water heater, banging 
water pipes. % source ==> http://www.watts.com/pages/faq.asp?catId=64&faqId=7
% https://pdfs.semanticscholar.org/1c7f/cee3eee8bf510c8846ec5c7bfeb03079a544.pdf
\item Decrease of water quality when stored. 
\begin{itemize}
\item The oxygen ($O_2$) level at the consumer should be no lower than 5 $mg/L$. 
\item Depending factors for water lifespan in a tank. (Quality of the water, open/closed tank, chemical treatments, room light and temperature )  % Source in Spanish https://www.soloarquitectura.com/foros/threads/cuanto-tiempo-puede-estar-el-agua-en-un-deposito-sin-pudrirse.53557/
\end{itemize}
%http://www.energibesparelser-vand.dk/Iltning-1239.aspx
%https://www.retsinformation.dk/Forms/R0710.aspx?id=174907#id81b8fb56-c281-4ff0-ae5c-4f0d56b03060
Health and oxygen 
\end{itemize}



% Health 
% https://www.nap.edu/read/11262/chapter/5#21