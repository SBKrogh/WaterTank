\chapter{Accepttest}
\label{accepttest}

In this section the controller designed in \chapref{control_design} will be tested on the available test setup in order to verify if the requirements described in \chapref{Requirements_and_constraints} are met. 

The designed MPC is implemented as a receding horizon control with a 24 hour control horizon due to the behavior of the electrical price and the water consumption which both have a periodicity of 24 hours. 
Based on this, a test conducted over a 24 hour period would allow to investigate the main characteristics of the MPC when price and consumption deviations during a day occur due to the periodicity of the data.   
Due to time limitations the test is scaled down to minimize the time used to conduct the measurements. 
The scaling factor has been chosen with the dynamics of the WT in mind as this has the slowest time constant in the system. 
The time constant of the WT model is determined to approximately 20 minutes, thus based on the Nyquist criterion, the scaling is chosen in such a way that a one hour period for the MPC will last 5 minutes. This equals to a sampling rate which is four times faster than the time constant of the WT. 

The controller will be tested according to the requirements found in \chapref{Requirements_and_constraints}, these are: 
\todo{wrong notation and missing requirements}
\begin{itemize}
	\item Pressure at CP, $0.08 < \pmb{p_{cp}} < 0.14 \:[\text{bar}]$
%
	\item Minimizing the total cost of running the system
\end{itemize}    