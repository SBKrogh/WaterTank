\subsection{Valve Model}  
\label{ValveModel}

%R2015-1-S1 type valve

Valves in the water distribution system are modelled according to the same principle as pipes with the difference that the length of each valve($L$) and the change in elevation($\Delta z$) are assumed to be zero. Therefore, it is assumed that the length of the valve does not influence the flow and the pressure between the endpoints, considering the fact that the length of a valve is considerably smaller than the length of a pipe. Another fair assumption is that in case of a valve, elevation is not present. \\
In the given system, valves are considered as end-user components since they are placed only in the PMAs. These user valves have a variable opening degree(OD) which influences the pressure drop across the endpoints. Valves can be also seen as pipe fittings where the OD is constant for all times, however there are not any fittings in the system, moreover the model of pipes covers it anyway. \\
Due to the above-mentioned considerations, by recalling \eqref{FinalPipeModel}, it simplifies as follows: 

\begin{equation}
\label{ValveModel}
 \Delta p =  - k_f \frac{8q^2}{\pi^2gD^4} \rho g 
\end{equation}

In \eqref{ValveModel}, the form-loss coefficient is taken into account in order to determine the pressure drop. Although $k_f$ is a coefficient that describes the resistance of the valve, manufacturers provide another constant which indicates the valve capacity instead. This coefficient is called the $k_{v100}$- factor that describes the conductivity of the valve at maximum OD. According to the definition of this parameter, it sets the relationship between the capacity through the valve and the pressure drop of $\Delta p = 1 [bar]$ at a fully open state of the valve. According to \cite{kvvalve}, the properties of water fulfil the requirements which allows to write up the following expression for flow and pressure: 

\begin{equation}
\label{kvequation}
 q =  k_{v100} \sqrt{\Delta p} 
\end{equation}

\eqref{kvequation} can be derived in detail using the law of continuity for each endpoint of the valve, however the exact derivations can be found in the datasheet \cite{kvvalve}. In the further description and derivations, the coefficients and all the technical considerations are based on this datasheet.  

\subsubsection{Valve conductivity function $k_v(OD)$}
\label{OD}

Instead of $k_{v100}$, more generally $k_v(OD)$ can be used which is a function of the opening degree, where $OD \in  [0,1]$. In case of user operated valves, $k_{v}$ does not remain constant, it ranges over a compact set of values. \cite{Kallesoe2009}

All valves in the system share the same characteristics

\subsubsection{Unit transformation}
\label{unittransform}

\subsubsection{Complete valve model}
\label{unittransform}


\cite{kvvalve}
 
sds \cite{valvedatasheet}

erer \cite{keller}
 
 
 