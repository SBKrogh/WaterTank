\section{Complete system model} 
\label{CompleteSystemModel}

Using the final expression of each component, a complete system model can be obtained. This model includes pipe, valve, pump components and the WT. 

The model of the water tower is described by a first order differential equation, consisting of the first time derivative of the pressure drop. The final expression is shown in \secref{WaterTankModel}:

\begin{equation}
  \label{CompleteWTModel}
  \Delta {\dot{p}_{WT;k}} = \frac{1}{C_{H;k}} q_k 
\end{equation}

The complete model consists of the pipe model, \eqref{CompactValveModel} the valve model, \eqref{PumpModel} the pump model and \eqref{CompleteWTModel} the WT model. For the pressure drop of the $k^{th}$ component the following expression can be written: 

%\begin{equation}
%\label{CompletePressure}
%\Delta P_k + \Delta p_{WT;k} = 0
%\end{equation}
%
%Where $\Delta P_k$ represents the pressure drop across the $k^{th}$ component and $\Delta p_{WT;k}$ represents

\begin{equation}
\label{CompleteModel}
\Delta p_k = \underbrace{\lambda_k (q_k) + \zeta_k - J_k \dot{q_k}}_\text{Pipe} + \underbrace{\mu (q_k, k_v)}_\text{Valve} - \underbrace{\alpha (u)}_\text{Pump} + \underbrace{\Delta p_{WT;k}}_\text{Water tank}
\end{equation}

The complete system model, \eqref{CompleteModel} is used to represent the pressure contribution of each component. In order to describe each component by \eqref{CompleteModel} the parameters that correspond to the specific component are chosen accordingly and the remaining are set to zero as they are not used to describe the specific component. 
%At the time of modelling the components, the proper components will be annulled to zero. 
The relation seen in \tabref{tab:parametrization_model} shows the parametrization of the system.

\begin{figure}[H]
	\centering
\begin{tabular}{c|cccccc} 
  			\bfseries Component    &     $J_k$    &     $\lambda_k$    &    $\mu_k$    &   $\alpha_k$    &  $\zeta_k$   &	$\Delta p_{WT;k}$	\\ \hline
			kth Pipe		  	   &     $J_k$    &     $\lambda_k$    &    0          &   0             &  $\zeta_k$   &   0	    			\\ 
			kth Valve              &     0        &     0              &    $\mu_k$    &   0 		     &  0    		&   0					\\ 
			kth Pump 		       &     0        &     0              &    0          &   $\alpha_k$    &  0    		&   0	    			\\
			kth Water tower 	   &     0        &     0              &    0          &   0             &  0    		&   $\Delta p_{WT;k}$	    
\end{tabular}
\captionof{table}{Complete model parametrization.}
		\label{tab:parametrization_model}

\end{figure}	

\subsubsection{Unit transformation}
\label{unittransform}

During the derivation of the dynamic model, the unit of the physical variables are considered as pascals and seconds. However, it will be concluded in a later chapter that the flow is significantly small compared to the pressure if the SI-units are kept. Therefore a unit conversion is carried out from Pascal[Pa] to bars and from seconds[s] to hours[h]. Another reason which makes this conversion reasonable is that the conductivity function,$k_v100$ in \secref{OD}, relies on the condition that the pressure drop is one bar. The reason behind the time scaling is due to conventions. Among the research community in hydraulics a convenient way to handle the volumetric flow in $[m^3/h]$ instead of$[m^3/s]$. 
\\
The detailed derivation of the unit conversion can be found in \secref{unitCON}. The result is stated here: 

\begin{equation}
   \frac{L \rho}{A\cdot10^5} \frac{d}{dt}\frac{q}{3600} = \Delta \frac{p}{10^5} - (\frac{8fL}{\pi^{2}gD^5\cdot10^5} + k_f \frac{8}{\pi^2gD^4\cdot10^5}) \rho g \frac{|q|}{3600} \frac{q}{3600} - \frac{\Delta z \rho g}{10^5}
\end{equation}
