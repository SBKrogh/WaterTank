\section{Complete system model} 
\label{CompleteSystemModel}

Writing up the final expression of each component, a complete system model can be obtained. This model includes pipe, valve and pump components. The water tower is not described in this model but instead by the final expression obtained in \secref{WaterTankModel}:

\begin{equation}
  \label{CompleteWTModel}
  \Delta \dot{p_k} = \underbrace{\frac{1}{C_H}q_k}_\text{Water tank} 
\end{equation}


 The complete model is based on \eqref{CompleteModel} the pipe model, \eqref{CompactValveModel} the valve model and \eqref{PumpModel} the pump model which are combined as shown:

\begin{equation}
  \label{CompleteModel}
  \Delta P_k = \underbrace{\lambda_k (q_k) + \zeta_k}_\text{Pipe} + \underbrace{\mu (q_k, k_v)}_\text{Valve} - \underbrace{\alpha (u)}_\text{Pump} 
\end{equation}



The complete system model, \eqref{CompleteModel} is used to represent each components except the WT which is described by \eqref{WTequation}. In order to describe each component by \eqref{CompleteModel} the parameters that correspond to the specific component are chosen accordingly and the remaining are set to zero as they are not used to describe the specific component. 
%At the time of modelling the components, the proper components will be annulled to zero. 
The relation seen in \tabref{tab:parametrization_model} shows the parametrization of the system.

\begin{figure}[H]
	\centering
\begin{tabular}{c|cccc} 
  			\bfseries Component    &     $\lambda_k$    &    $\mu_k$    &   $\alpha_k$    &  $\zeta_k$  		\\ \hline
			kth Pipe		  	   &     $\lambda_k$    &    0          &   0             &  $\zeta_k$   	    \\ 
			kth Valve              &     0              &    $\mu_k$    &   0 		      &  0    				\\ 
			kth Pump 		       &     0              &    0          &   $\alpha_k$    &  0    			    
\end{tabular}
\captionof{table}{Complete model parametrization.}
		\label{tab:parametrization_model}

\end{figure}	

