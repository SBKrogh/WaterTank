\subsection{Complete component model}
\label{CompleteSystemModel}
Gathering the pressure drops, caused by each type of component in the system, a complete system model can be obtained. This model includes the pipe, valve, pump elements and the WT. 

The model of the WT is described by a first order differential equation, consisting of the first time derivative of the pressure drop. The final expression is shown in \eqref{CompleteWTModel}:

\begin{equation}
  \label{CompleteWTModel}
  \Delta \dot{p}_{WT;k} = \frac{1}{C_{H;k}} q_k 
\end{equation}

Although \eqref{CompleteWTModel} includes indexing for the pressure drops across the WTs, it is worth mentioning that the water distribution system consists of only one WT.
\\
The complete model consists of the pipe model, \eqref{CompactValveModel}, the valve model, \eqref{eq:PumpModel}, the pump model and the WT, \eqref{CompleteWTModel}. For the pressure drop across the $k^{th}$ component the following expression can be written: 

%\begin{equation}
%\label{CompletePressure}
%\Delta P_k + \Delta p_{WT;k} = 0
%\end{equation}
%
%Where $\Delta P_k$ represents the pressure drop across the $k^{th}$ component and $\Delta p_{WT;k}$ represents

\begin{equation}
\label{CompleteModel}
\Delta p_k = \underbrace{\lambda_k (q_k) + \zeta_k + J_k \dot{q_k}}_\text{Pipe} + \underbrace{\mu_k (q_k, k_v)}_\text{Valve} - \underbrace{\alpha_k(u_k)}_\text{Pump} + \underbrace{\Delta p_{WT;k}}_\text{Water tank}
\end{equation}

The complete component model, \eqref{CompleteModel}, is used to represent the pressure loss or contribution across each component. In order to describe every part of the system by \eqref{CompleteModel}, the parameters and functions corresponding to the specific part of the system are selected. The remaining expressions are set to zero if the model does not match the specific part of the network. In other words, if the $k^th$ element of the system is a pump, then only $\alpha_k(u_k)$ is taken into account and the rest of the expressions are set to zero. 
%At the time of modelling the components, the proper components will be annulled to zero. 
\tabref{tab:parametrization_model} shows the parametrization of the system: 

\begin{figure}[H]
	\centering
\begin{tabular}{c|cccccc} 
		\bfseries Component    &     $J_k$  &$\lambda_k$    &    $\mu_k$    &   $\alpha_k$   &  $\zeta_k$   &  $\Delta p_{WT;k}$ \\\hline
		Pipe			  	   &     $J_k$  &$\lambda_k$    &    0          &   0            &  $\zeta_k$   &  0	    			\\ 
		Valve       	       &     0      &0              &    $\mu_k$    &   0 		     &  0    		&  0					\\ 
		Pump 		    	   &     0      &0              &    0          &   $\alpha_k$   &  0    		&  0	    			\\
		Water tower 	   	   &     0      &0              &    0          &   0            &  0    		&  $1$	
		%\Delta p_{WT;k}    
\end{tabular}
\captionof{table}{Complete model parametrization.}
		\label{tab:parametrization_model}

\end{figure}	

\textbf{Unit transformation}
\label{unittransform}

During the derivation of the dynamic model, the unit of the physical variables are considered as pascals and seconds. However, it is concluded in a later chapter that the flow is significantly small compared to the pressure if the SI-units are kept. Therefore a unit conversion is carried out from Pascal[Pa] to [bar]s and from seconds[s] to hours[h]. \todo{we need to ref to the chapter where we conclude this or then we need a source} Another reason which makes this conversion reasonable is that the conductivity function, $k_{v100}$ in \secref{OD}, is derived under the condition that the pressure drop is one bar \cite{keller}. The time scaling is due to conventions. Among the research community in hydraulics, a convenient way to handle the volumetric flow is in $[m^3/h]$ instead of $[m^3/s]$. 
\\
The detailed derivation of the unit conversion can be found in \appref{unitCON}. The result is stated here: 

\begin{equation}
   \frac{L \rho}{A\cdot10^5} \frac{d}{dt}\frac{q}{3600} = \Delta \frac{p}{10^5} - \Big(\frac{8fL}{\pi^{2}gD^5\cdot10^5} + k_f \frac{8}{\pi^2gD^4\cdot10^5}\Big) \rho g \frac{|q|}{3600} \frac{q}{3600} - \frac{\Delta z \rho g}{10^5}
\end{equation}
