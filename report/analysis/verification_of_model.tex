\section{Verification of model}
\label{verification_of_model}

The model have now med linearized, \secref{Linearization}, the parameters have been estimated, \secref{LinParamEst}, arranged on SS form, \secref{SystemLin_control}, and discretized, \secref{discrete_SS}. In this section the model is excited in different ways, to consider if the behavior  seems reasonable. First the time constant for the WT is compared to real measurements, then an input is applied to both pumps in order to see how the pressure at the CP change and then the disturbance, OD of the values, is changed to see how the pressure at the CP change. 

\begin{figure}[H]
   \centering
    \input{report/tikz/simulation_time_constant.tex}
    \caption{The first plot show a simulation where a step is applied to both pumps on the final model. The second plot show the measurement described in \appref{sec:WT_TimeConstant}.}
    \label{simulation_time_constant}
\end{figure}

At \figref{simulation_time_constant} a positive step is applied to both the model and the real system. As it can be seen the pressure rises, meaning that the level of the WT rises. Furthermore the time constant is marked with a blue dot on both plots. From this a time constant for the simulation of 903 seconds, and a time constant for the real system of 1155 seconds are obtained. 

\begin{figure}[H]
   \centering
    \input{report/tikz/input_simulation.tex}
    \caption{A series of steps are applied to the pumps and it can be seen that the pressure follows is expected.}
    \label{input_simulation}
\end{figure}

At \figref{input_simulation} a series of step are applied to the model and the pressure change at the CP is plotted. Here it can be seen that the when a positive step is applied the pressure increases positive, and the opposite with a negative step. Furthermore it can be seen that there is a pressure different between PMA1 and PMA2. This is also expected because of the height different.  

\begin{figure}[H]
   \centering
    \input{report/tikz/disturbance_simulation.tex}
    \caption{A series of steps are applied to the values and it can be seen that the pressure follows is expected.}
    \label{disturbance_simulation}
\end{figure}

At \figref{disturbance_simulation} a series of step is applied to the values, in order to see how a change in disturbance affect the pressure at the CP. Here a pressure drop can be seen when OD is increased and an increase in pressure when OD is lowered. 