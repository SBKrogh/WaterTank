This section will give an introduction to the available test system, including structure and components overview.  

\section{System overview}
\label{system_overview}
To develop and test different control methods for a water distribution system a test setup is required.
Such a setup is available at Aalborg university which is based on a real water distribution system, though as a 1:20 downscaled version.

%picture of test system
\begin{figure}[H]
\centering
\includegraphics[width=0.85\textwidth]{report/pictures/test_system_wide}
\caption{The available test setup used to represent a real water distribution system.}
\label{fig:test_setup}
\end{figure}


The test setup represents a real system, thus the same structure concerning piping, levelling and all the other components. To achieve different elevation levels between system parts, the setup is mounted on a wall. This also allows for a quick overview of the complete setup and eases access to the components. As the system is used for various test scenarios other equipment are also present in the test setup shown in \figref{fig:test_setup}, enabling the test system to mimic a variety of different system types and scenarios. A diagram representing the part of the test setup that will be used in this project is shown in \figref{fig:sys_model_overview}. 

%tikz of our system
\begin{figure}[H]
\centering
\input{report/tikz/sys_model_overview.tex} 
\caption{Overview of the reduced system that fulfills the scenario of this project.}
\label{fig:sys_model_overview}
\end{figure}
\todo{Should we consider changing the name "elevated reservoir" to "water tower". We have uesed the word water tower in the introduction}

The system consists of different parts, the main part being a water reservoir placed on ground level, used to supply the system. Two pumps are connected to the reservoir and they supply water to the main water ring formed around the consumers. 
Another water reservoir is connected to the water ring by a dedicated pump. This reservoir is elevated and can, when filled, be used to pressurize the system. 
From the water ring two PMA's are connected via their own pump. In each PMA a measuring point is placed and the pressure at this point shall be kept. Furthermore two consumers are placed in each PMAs.         
	
\begin{figure}
	\centering
	\begin{minipage}[b]{0.4\textwidth}
		\centering
		\input{report/tikz/basic_example_sys.tex} 
		\caption{Basic water distribution network.}
		\label{fig:Basic_example_sys}
	\end{minipage}
	\hspace{15pt}
	\begin{minipage}[b]{0.4\textwidth}
		\begin{tabular}{|c|c|} \hline
  			\bfseries Symbol 	 					&   \bfseries Name 					\\ \hline
			\input{report/tikz/pump.tex} 		  	&	Pump							\\ \hline
			\input{report/tikz/manual_valve.tex} 	&	Manual valve					\\ \hline
			\input{report/tikz/elec_valve.tex} 		&	Electronic valve				\\ \hline
			\input{report/tikz/pipe.tex} 		  	&	Pipe segment					\\ \hline
			\input{report/tikz/pressure_sensor.tex} &	Pressure sensor					\\ \hline
			\input{report/tikz/dif_pres_sens.tex} 	&	Differential pressure sensor	\\ \hline
			\input{report/tikz/gnd.tex} 		  	&	Gnd								\\ \hline
		\end{tabular}
		\captionof{table}{Symbol and name for component in the water network.}	
		\label{tab:sys_comp_overview}
	\end{minipage}
\end{figure}
%\end{minipage}