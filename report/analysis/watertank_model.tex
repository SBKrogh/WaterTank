\subsection{Water Tower} 
\label{WaterTankModel}

Water towers are used to maintain the correct pressure level in the system, ensure reliability and to improve the optimality of the water supply. The WT plays a determinative role in the flow control, therefore its dynamic model has to be derived. 

Similarly to the modelling of the other components, the relation between the two dual variables, pressure difference and flow is  derived. The structure of the WT is illustrated in \figref{fig:watertower_sketch}.

%To do so, Bernoulli´s principle is applied between two points (see figure...).

%tikz of the water tower
\begin{figure}[H]
\centering

%Old one: 
%\begin{tikzpicture} [scale=1,transform shape]
%  \fill[blue!30]
%        decorate[ragged border]{
%        (0,2) -- (3.5,2)
%        }
%        -- (3.5,-0.5) -- (2,-0.5) -- (2,-1.5) -- (1.5,-1.5) -- (1.5,-0.5) --(1.5,-0.5) -- (0,-0.5) -- cycle;
%
%  \draw [-,line width=1.1pt]  (0,3.5) -- (0,2.5) -- (0,-0.5) -- (1.5,-0.5) -- (1.5,-0.5) -- (1.5,-1.5);
%  \draw [-,line width=1.1pt] (3.5,3.5) --  (3.5,-0.5) -- (2,-0.5) -- (2,-1.5);
%  \draw[|-|] (-0.5,-0.5) --
%        node[fill=white,font=\normalsize,inner ysep=2pt,inner
%                xsep=0]{$h_1$}(-0.5,2);
%                
%  \draw[|-|] (3.5,4) --
%        node[fill=white,font=\normalsize,inner ysep=2pt,inner
%                xsep=0]{$D$}(0,4);
% 
% \draw[red,fill=black] (1.75,-0.5) circle (.3ex);
%  \draw[red,fill=black] (1.75,2) circle (.3ex);
%
%\node at (2,2) {\normalsize{$p_1$}};
%\node at (2,-0.34) {\normalsize{$p_2$}};
%
%\draw [-latex](1.75,3.0) -- (1.75,2.29);
%\node at (2,2.7) {\normalsize{$q_1$}};
%\node at (1.9,-1.67) {\normalsize{$q_2$}};
%\draw [-latex](1.75,-0.7) -- (1.75,-1.5);
%
%%\draw [dashdotdotted](1.75,3.75) -- (1.75,-2.0);
%
%
%\end{tikzpicture}%


\begin{tikzpicture} [scale=1.1,transform shape]
  \fill[cyan!30]
        decorate[ragged border]{
        (0,2) -- (3.5,2)
        }
        -- (3.5,-0.5) -- (2,-0.5) -- (2,-1.5) -- (1.5,-1.5) -- (1.5,-0.5) --(1.5,-0.5) -- (0,-0.5) -- cycle;

  \draw [-,line width=1.1pt]  (0,3.5) -- (0,2.5) -- (0,-0.5) -- (1.5,-0.5) -- (1.5,-0.5) -- (1.5,-1.5);
  \draw [-,line width=1.1pt] (3.5,3.5) --  (3.5,-0.5) -- (2,-0.5) -- (2,-1.5) ;
  \draw[|-|] (-0.5,-0.5) --
        node[fill=white,font=\normalsize,inner ysep=2pt,inner
                xsep=0]{$h$}(-0.5,2);
                
  \draw[|-|] (3.5,4) --
        node[fill=white,font=\normalsize,inner ysep=2pt,inner
                xsep=0]{$D$}(0,4);
 


\node at (2.5,2.1) {\normalsize{$p_a$}};
\node at (2.5,-0.3) {\normalsize{$p$}};


\node at (1.9,-1.9) {\normalsize{$q$}};
\draw [-latex](1.75,-2.1) -- (1.75,-1.2);



\end{tikzpicture}%
 
\caption{Sketch of the water tower}
\label{fig:watertower_sketch}
\end{figure}

Two reference points are selected which represent the pressure at specific heights inside the tank. $p_1$ represents the pressure on the surface of the water level, therefore it is the atmospheric pressure at all time.The pressure in point $p_2$ equals to the pressure value on the bottom of the tank. In order to get an expression for pressure differences in the function of the water level $h_1$, Bernoulli's law is applied as it is shown in \eqref{bernoulli}: 

\begin{equation}
  \label{bernoulli}
  h_1\rho g + p_1 + \frac{1}{2}\rho {v_1}^2 = h_2\rho g + p_2 + \frac{1}{2}\rho {v_2}^2
\end{equation}

\begin{minipage}[t]{0.20\textwidth}
Where\\
\hspace*{8mm} $h_i$ \\
\hspace*{8mm} $\rho$ \\
\hspace*{8mm} $p_i$ \\
\hspace*{8mm} $v_i$ \\

\end{minipage}
\begin{minipage}[t]{0.68\textwidth}
\vspace*{2mm}
is the elevation of the points represented\\
is the water density\\
is the pressure at a chosen point\\
is the velocity of the water level at a chosen point
\end{minipage}
\begin{minipage}[t]{0.10\textwidth}
\vspace*{2mm}
\textcolor{White}{te}$\unit{m}$\\
\textcolor{White}{te}$\unit{\frac{kg}{m^3}}$\\
\textcolor{White}{te}$\unit{Pa}$\\
\textcolor{White}{te}$\unit{\frac{m}{s}}$
\end{minipage}


%The pressure at point $1$ ($p_1$) is considered to be equal to the atmospherical pressure 
%so it is set to be $0$. 
The height of reference point $1$ changes according to the water level, therefore its position is in the function of $h1$. 
In \eqref{bernoulli}, the reference point on the bottom of the tank remains fixed.
%As a reference point in \eqref{bernoulli}, the bottom of the tank is chosen.
Therefore the height at this point($h_2$) is set to 0. 
\\

The flow rates are equal in both point $1$ 
and point $2$, so are the velocities of the water. \\
Due to the these considerations \eqref{bernoulli} simplifies to: 

\begin{equation}
  \Delta p = p_2 - p_1 = \rho g h_1
  \label{Pressuredifference}
\end{equation}

In \eqref{Pressuredifference} the gravity and the density are constants, however the height of the water level changes proportional to the pressure difference. 
\\
To derive the equation that describes the change of the height according to 
time, the law of conservation mass is applied. This law can be specifically used for the flow in the water tower, describing the difference between the inlet and outlet flow to be equal to the velocity by which the tank volume increases or decreases. 

\begin{equation}
  q_{2} - q_{1} - q_2 = \frac{d}{dt}V_t = \frac{d}{dt}Ah_1 = A \frac{d}{dt}h_1 = \frac{\pi}{4} D^2 \frac{d}{dt} h_1
  \label{FlowConservation}
\end{equation}

\begin{minipage}[t]{0.20\textwidth}
Where\\
\hspace*{8mm} $q_i$ \\
\hspace*{8mm} $V_t$ \\
\hspace*{8mm} $D$ 
\end{minipage}
\begin{minipage}[t]{0.68\textwidth}
\vspace*{2mm}
is the inlet and outlet flow\\
is the volume of the water tower\\
is the diameter of the tank 
\end{minipage}
\begin{minipage}[t]{0.10\textwidth}
\vspace*{2mm}
\textcolor{White}{te}$\unit{\frac{kg}{m^3}}$
\textcolor{White}{te}$\unit{m^3}$
\textcolor{White}{te}$\unit{m}$
\end{minipage}

In \eqref{FlowConservation}, $q1$ is zero since only $q_2$ is taken into consideration as a two-directional in- or outlet flow. 
\\
By integrating \eqref{FlowConservation} according to time the following yields:

\begin{equation}
 h_1 = \int dh_1 = \int \frac{1}{A} q_2 dt
  \label{integheight}
\end{equation}

Plugging \eqref{integheight} into \eqref{Pressuredifference}, the following overall expression can be given between pressure and flow: 

\begin{equation}
\label{WTequation}
  \Delta p =  \frac{\rho g}{A}  \int   q_2 dt = \frac{1}{C_H} \int   q_2 dt
\end{equation}

\begin{minipage}[t]{0.20\textwidth}
Where\\
\hspace*{8mm} $C_H$ 
\end{minipage}
\begin{minipage}[t]{0.68\textwidth}
\vspace*{2mm}
is the hydraulic capacitance
\end{minipage}
\begin{minipage}[t]{0.10\textwidth}
\vspace*{2mm}
\textcolor{White}{te}$\unit{\frac{m^3}{N/m^2}}$
\end{minipage}

An analogy can be made between an electronic circuit and the hydraulic system, where the WT acts as a capacitor.  Deriving the relationship between the voltage across the capacitor and the 
charge of the capacitor:

\begin{equation}
  U = \frac{1}{C} \int I dt
  \label{ElecCircuirt}
\end{equation}

\begin{minipage}[t]{0.20\textwidth}
Where\\
\hspace*{8mm} $U$ \\
\hspace*{8mm} $C$ 
\end{minipage}
\begin{minipage}[t]{0.68\textwidth}
\vspace*{2mm}
is the voltage\\
is the capacitance 
\end{minipage}
\begin{minipage}[t]{0.10\textwidth}
\vspace*{2mm}
\textcolor{White}{te}$\unit{V}$
\textcolor{White}{te}$\unit{F}$
\end{minipage}

In the \eqref{WTequation} the volume flow rate ($q$) is equivalent to the current ($I$) in a 
circuit and the constant term \big($\frac{A}{\rho g}$\big) is equivalent to the capacitance of a capacitor ($C$). The voltage drop is analogous to the pressure drop in the water system.

%
%Rewriting \eqref{Pressuredifference} in a simplified way:
%
%\begin{equation}
%  \Delta q =  \frac{\pi}{4} D^2 {dh}
%\end{equation}
%
%In order to solve for $h$ both sides are integrated: 
%\begin{equation}
%  \int dh = \int \frac{1}{A} \Delta q dt
%  \label{integheight}
%\end{equation}
%
%Simplifying \refeq{integheight}:
%\begin{equation}
%  h =\frac{1}{A}  \int  \Delta q dt
%\end{equation}
%
%Once the equation describing $h$ is obtained, it can be inserted in \refeq{Pressuredifference} 
%resulting in the final expression for the water tower model:
%
%\todo{Maybe write down the analogous expression for electric circuits ( capacitors) and make a comparison}
%
%\begin{equation}
%  \Delta p =  \frac{\rho g}{A}  \int  \Delta q dt
%\end{equation}


