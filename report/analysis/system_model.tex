\section{Network Model}  
\label{SystemModel}

After deriving the dynamics of all components in the network, the complete system can be drawn. In \appref{systemdiagram}, the topology of the test system is described in more detail. In the following discussion, all statements and notations are based on the figure in the appendix. 
\\
The way of modelling a hydraulic system is in some way analogous to an electric circuit. Most of the various hydraulic components can be represented as electronic equivalents and vice versa, however there are some differences too. It should be emphasized that there are not any analogous phenomenon as magnetic flux in hydraulic networks. 
\\
In the block diagram of the system, nodes are introduced which represent different potential points in the system. This is equivalent to hydraulic pressures. Nodes represent points in the system where pressure might have different values. These points represent interconnection between hydraulic components and take into account the fact that each individual component in the system has an effect on the pressure drop on their two corresponding endpoints. Therefore nodes are placed at all places where the modelling requires it. 
\\
In the network, volume flow rate is equivalent to current and the quantity of water has similar representation as charge in an electric circuit. Again, it should be noted that e.g. the water quantity cannot be affected by magnetic fields, therefore the word: similar.
\\ 
Although nodes can be placed across all the endpoints of each component, some simplifications are introduced. There are two different types of simplification in the network. For a better transparency, these parts of the system are shown in \figref{fig:subsys_1} and \figref{fig:subsys_2}: 
\todo{wrong reference, two figures are similar, correction needed!}

%%picture of test system
%\begin{figure}[H]
%\centering
%\includegraphics[width=0.35\textwidth]{report/pictures/missingfigure}
%\caption{Simplification when the pump is not used}
%\label{fig:subsys_1}
%\end{figure}

\begin{figure}[H]
\centering
\begin{minipage}{0.4\textwidth}
  \centering
  \input{report/tikz/valve_pump_wt_simple.tex} 
  %\captionof{figure}{$\omega = 0$}
  \label{fig:subsys_1}
\end{minipage}%
\begin{minipage}{0.4\textwidth}
  \centering
  \input{report/tikz/valve_pump_n1_n2.tex} 
  %\captionof{figure}{$\omega \neq 0$}
  \label{fig:subsys_2}
\end{minipage}
\captionof{figure}{Simplifications: In the left when $\omega = 0$ and in the right one when $\omega \neq 0$}
\end{figure}

In \figref{fig:subsys_1} three components are shown between $n_3$ and $n_{18}$. The latter is the node where the WT connects to the system. In this particular case the pump is turned off, however contributes to the pressure drop due to its resistance. The same can be said for the valves, except that they are fully open at all time but they modify the flow. Extra nodes are not introduced between the valves and the pump, instead the series connection is seen as one component. This can be modelled by lumping the resistance of the valve, \eqref{CompleteValveModel}, into the model of the pump, \eqref{PumpModel}, when the rotational speed is zero. Thus the following model yields for the case when $\omega = 0$:

\begin{equation}
  \Delta p = \Big(\frac{2}{k_{v100}^2} - a_{h2}\Big)|q| q 
  \label{omega_zero}
\end{equation}

And the following when $\omega \neq 0$

\begin{equation}
  \Delta p = \Big(\frac{2}{k_{v100}^2} - a_{h2}\Big)|q| q  + a_{h1} \omega_r q + a_{h0}{\omega_r}^2
  \label{omega_notzero}
\end{equation}

The case when $\omega = 0$ applies to components between ($n_4$-$n_8$), ($n_3$-$n_{18}$)  and to ($n_5$-$n_{13}$). The case when $\omega \neq 0$ applies to the two main pumps, between ($n_1$-$n_2$) and ($n_1$-$n_7$). All these subsystems are modelled as described above. 
\\
Since the components influence the pressure between the endpoints, as it was mentioned before, they can be represented as electric components. Valves are considered as nonlinear resistors, since the pressure is the quadratic function of the flow and it has a resistance depending on the OD. The model of the pipes is equivalent to a series connection of a linear inductor, a non-linear resistor and a voltage source which stands for the elevation, gravity term. The pumps provide pressure and therefore flow to the system. They can be seen as voltage generators. The WT is a simple capacitor, as it was described previously. The equivalence between the hydraulic and electric system is summarized in \tabref{tab:hydraulic_electrical}.

\begin{figure}[H]
	\centering
\begin{tabular}{c|c} 
  			\bfseries Hydraulic system    &     \bfseries Electrical system  \\ \hline
			Valve		  	  &     Nonlinear resistor   \\ 
			Pipe              &     Linear inductor with a nonlinear drift term       \\ 
			WT 		          &     Capacitor       \\
			Pressure	 	  &     Voltage    \\
			Flow 		      &     Current       \\  
			Pumps 		      &     Voltage source           
\end{tabular}
\captionof{table}{Equivalence of electrical and hydraulic network.}
		\label{tab:hydraulic_electrical}

\end{figure}	


	