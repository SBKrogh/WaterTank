\subsection{Pump Model} 
\label{PumpModel}







 
 \textbf{Kirchhoff´s Law implementation for hydraulic networks}\\
 \label{KirchhoffLaw}
 Kirchhoff´s law represents the interconnection of the elements of an electrical circuit. An hydraulic approach 
 can be done to Kirchhoff´s law in order to carry out the analysis of the hydraulic network of a system, having to satisfied Kirchhoff´s 
 current law (KCL) and voltage law (KVL). Therefore, voltage and currents are replaced by pressure and flows, respectively. 

The critical points in a closed hydraulic system are usually the inlets and outlets of elements such as pipes, pumps and valves. 
The pressure drops taking place in the pipes and valves can be considered as inductors and nonlinear resistors, respectively. 

\textit{Maybe here before we start with the Graph Theory could be nice to make a figure showing the changes between the hydraulic circuit to the electrical 
one.}

The analogy between hydraulic and electrical circuits described above is applied in the graph theory…(start explaining Graph Theory).






