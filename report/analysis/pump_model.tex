\subsection{Pump Model} 
\label{PumpModel}
In order to move the produced water from the reservoirs to the costumers a 
pumping equipment is needed.  Different types of pumps can be used in a water distribution to 
guarantee that the water raises every end-users with the adequate pressure 
quality.\\
Centrifugal pumps are ideal to deliver the produced water to the end-users, they 
generate a positive pressure within they guarantee a constant flow of water at a 
constant pressure for any given set of conditions. 
A model describing the pressure drop is derived which was presented in 
\cite{Phd_Carsten}, where the head introduced by the pump is given by:

\begin{equation}
  H = -a_{h2}{Q_i}^2 + a_{h1} \omega_r Q_i + a_{h0}{\omega_r}^2
  \label{PumpModel}
\end{equation}

\begin{minipage}[t]{0.20\textwidth}
Where\\
\hspace*{8mm} $H$ \\
\hspace*{8mm} $Q_i$ \\
\hspace*{8mm} $\omega_r$ 

\end{minipage}
\begin{minipage}[t]{0.68\textwidth}
\vspace*{2mm}
is the head produced by the water\\
is the volume flow through the impeller\\
is the impeller speed
\end{minipage}

The parameters inserted in \refeq{PumpModel} are given by:

\begin{minipage}[t]{0.20\textwidth}
\begin{equation*}
  a_{h2} = K_s + K_f
\end{equation*}
\end{minipage}
\begin{minipage}[t]{0.68\textwidth}
\begin{equation*}
  a_{h0} = 2K_sK_d - \sigma_s 
  \bigg(\frac{{r_2}^2}{g}-\frac{{r_1}^2}{g}\bigg)
\end{equation*}
\end{minipage}

\begin{equation}
  a_{h1} = \sigma_s 
  \bigg(\frac{r_2}{gA_2}cot(\beta_2)-\frac{r_1}{gA_1}cot(\beta_1)\bigg)-K_s{K_d}^2
\end{equation}



 
% \textbf{Kirchhoff´s Law implementation for hydraulic networks}\\
% \label{KirchhoffLaw}
% Kirchhoff´s law represents the interconnection of the elements of an electrical circuit. An hydraulic approach 
% can be done to Kirchhoff´s law in order to carry out the analysis of the hydraulic network of a system, having to satisfied Kirchhoff´s 
% current law (KCL) and voltage law (KVL). Therefore, voltage and currents are replaced by pressure and flows, respectively. 
%
%The critical points in a closed hydraulic system are usually the inlets and outlets of elements such as pipes, pumps and valves. 
%The pressure drops taking place in the pipes and valves can be considered as inductors and nonlinear resistors, respectively. 
%
%\textit{Maybe here before we start with the Graph Theory could be nice to make a figure showing the changes between the hydraulic circuit to the electrical 
%one.}
%
%The analogy between hydraulic and electrical circuits described above is applied in the graph theory…(start explaining Graph Theory).
%
%
%



