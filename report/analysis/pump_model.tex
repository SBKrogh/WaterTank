\subsection{Pump model} 
\label{PumpModel}
In order to move water from the reservoirs to the costumers, pumping is required. To guarantee that the water reaches every end-user with the appropriate pressure, different pumps can be used in the water distribution system.\\
Centrifugal pumps are ideal for this purpose. 
%They generate pressure within they guarantee a constant flow of water at a constant pressure for any given set of conditions. 
A model describing the pressure drop is derived which is presented in detail in \cite{Phd_Carsten}. The pressure provided by the pump is given by:
\todo{The list bellow should be with units.}
\begin{equation}
  \Delta p = -a_{h2}{q_i}^2 + a_{h1} \omega_r q_i + a_{h0}{\omega_r}^2
  \label{PumpModel}
\end{equation}

\begin{minipage}[t]{0.20\textwidth}
Where\\
\hspace*{8mm} $\Delta p$ \\
\hspace*{8mm} $a_{h2}$, $a_{h1}$, $a_{h0}$ \\
\hspace*{8mm} $q_i$ \\
and \hspace*{0.7mm} $\omega_r$ 

\end{minipage}
\begin{minipage}[t]{0.68\textwidth}
\vspace*{2mm}
is the head produced by the pump,\\
are constants describing the pump,\\
is the volume flow through the impeller,\\
is the impeller speed.
\end{minipage}
\begin{minipage}[t]{0.10\textwidth}
\vspace*{2mm}
\textcolor{White}{te}$\unit{Pa}$\\
\textcolor{White}{te}$\unit{\cdot}$\\
\textcolor{White}{te}$\unit{\frac{m^3}{s}}$\\
\textcolor{White}{te}$\unit{\frac{rad}{s}}$
\end{minipage}	


\subsubsection*{Hydraulic power and efficiency}
To minimize the running cost of the plant, it is necessary to know the power comsumption of the pumps and thus describe how much energy these consume. This can be done by calculating the hydraulic power created by the pumps and taking the efficient of the pumps into account. 
%the relationship between the efficiency of the pump and the cost of running them.

The hydraulic power created by a pump can be described by an equivalent to Joule's law, that is in terms of the pressure difference across the pump, multiplied with the flow through it, see \eqref{eq:Hyd_Pow}.

%\subsubsection*{Pump power consumption}
%\todo{We need to redo this part and somehow include the derivation of the first two equations!}
\begin{equation}
P_h = \Delta p \cdot q 
\label{eq:Hyd_Pow}
\end{equation}

\begin{minipage}[t]{0.20\textwidth}
Where\\
\hspace*{8mm} $P_h$
\end{minipage}
\begin{minipage}[t]{0.68\textwidth}
\vspace*{2mm}
is the hydraulic power
\end{minipage}
\begin{minipage}[t]{0.10\textwidth}
\vspace*{2mm}
\textcolor{White}{te}$\unit{W}$\\
\end{minipage}	

The energy used for running the pump can then be described through the relation between the hydraulic power and the efficiency, $\eta$. For simplicity $\eta$ is assumed to be constant, however for this assumption to hold $\eta$ must be chosen within the operating area of the pump. This can be seen in \eqref{eq:effic_pump}.

\begin{equation}
P_e = \frac{1}{\eta}\cdot\Delta p \cdot q 
\label{eq:effic_pump}
\end{equation}

\begin{minipage}[t]{0.20\textwidth}
Where\\
\hspace*{8mm} $P_e$\\
\hspace*{8mm} $\eta$
\end{minipage}
\begin{minipage}[t]{0.68\textwidth}
\vspace*{2mm}
is the power consumption of the pump\\
is the efficiency of the pump.
\end{minipage}
\begin{minipage}[t]{0.10\textwidth}
\vspace*{2mm}
\textcolor{White}{te}$\unit{W}$\\
\textcolor{White}{te}$\unit{\cdot}$\\
\end{minipage}	

\eqref{eq:effic_pump} will be used in a later section to minimize this cost of running the system. 


% In \secref{introduction} it has been introduced the aim of the project, consisting on minimizing the power consumption of the pumps in order to reduce 
% the cost of running the hydraulic setup. Therefore, a model of the pump power consumption is needed so it can be used afterwards. In \cite{Phd_Carsten} 
% an expression for the torque produced in a centrifugal pump is derived. 
% In \secref{introduction}, the aim of this project has been specified which consists of minimizing the electrical price of running the hydraulic setup, while maintaining a pressure within a limit. 
% A model of the pump power consumption is therefore needed.

% In \cite{Phd_Carsten}, an expression for the torque produced in a centrifugal pump is derived. 

% \begin{equation}
%   \tau = -a_{t2}{q_i}^2 + a_{t1} {\omega}_r {q_i} + a_{t0} {{\omega}_r}^2
%   \label{pumptorque}
% \end{equation}

% \begin{minipage}[t]{0.20\textwidth}
% Where\\
% \hspace*{8mm} $a_{t2}$, $a_{t1}$, $a_{t0}$ \\
% \hspace*{8mm} $q_i$ \\
% and \hspace*{0.7mm} $\omega_r$ 

% \end{minipage}
% \begin{minipage}[t]{0.68\textwidth}
% \vspace*{2mm}
% are constants describing the pump\\
% is the volume flow through the impeller,\\
% is the impeller speed.
% \end{minipage}
% \begin{minipage}[t]{0.10\textwidth}
% \vspace*{2mm}
% \textcolor{White}{te}$\unit{-}$\\
% \textcolor{White}{te}$\unit{\frac{m^3}{s}}$\\
% \textcolor{White}{te}$\unit{\frac{rad}{s}}$
% \end{minipage}	

% Making use of the relation between torque and electrical power, $P_e = \tau \omega$, the above expression can be rewritten in terms of electrical 
% power. 

% \begin{equation}
%   P_e = -a_{t2}{q_i}^2{\omega}_r + a_{t1} {{\omega}_r}^2 {q_i} + a_{t0} {{\omega}_r}^3
%   \label{pumptorque}
% \end{equation}



