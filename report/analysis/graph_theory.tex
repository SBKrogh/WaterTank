\section{Graph Theory}  
\label{GraphTheory}
A graph is a formal mathematical way of representation of a network which can be applicable, among others, 
in engineering or scientist research e.g., mechanicals systems, electrical circuits or hydraulic networks \cite{graph_intro}. 

In order to make the modelling of the water distribution network the Graph Theory is used. The graph is constructed 
based on the affinity between both electrical circuits and hydraulic networks. 
The behavior of the network is function of the characteristics of the 
electrical elements and how they are connected together. The pressures and flows are considered as 
voltage and currents, respectively.

\subsection{Incidence Matrix} 

The critical points in a closed hydraulic system are usually the inlets and outlets of elements such as pipes, pumps and valves. 
This components are considered as nodes and the paths connecting them are denoted as edges. 
In order to show the relationship between the nodes conforming a graph the 
incidence matrix is calculated. In hydraulic networks contest, the edges have a 
direction associated with them, this characteristic leads to have a incidence matrix, A, where:

\begin{equation}
\label{DiGraph}
 A =
		\left\{
		\begin{array}{ll}
		
		1 			&      \text{if the edges is incidence out of the node}	
\\
		-1                       &     \text{if the edges is incidence into the node}
\\

                0                       &      \text{otherwise}

		\end{array}
		\right.
\end{equation}	


 
%The pressure drops taking place in the pipes and valves can be considered as inductors and nonlinear resistors, respectively. 
%
%\textit{Maybe here before we start with the Graph Theory could be nice to make a figure showing the changes between the hydraulic circuit to the electrical 
%one.}
%
%The analogy between hydraulic and electrical circuits described above is applied in the graph theory…(start explaining Graph Theory).


 
 
 