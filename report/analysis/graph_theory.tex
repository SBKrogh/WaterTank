\section{Graph Representation}  
\label{GraphTheory}
A graph is a formal mathematical way of representation of a network which can be applicable, among others, 
in engineering or scientist research e.g., mechanicals systems, electrical circuits or hydraulic networks \cite{graph_intro}. 

In order to make the modelling of the water distribution network the Graph Theory is used. The graph, G, is constructed 
based on the affinity between both electrical circuits and hydraulic networks. 
Each terminal of the network is associated with a vertices, and the components of the system correspond to edges \cite{GraphTheoryCarsten}. 
\todo{Add reference to table of correlation}
\subsection{Incidence Matrix} 

The incidence matrix, H, of a graph with \textit{n} vertices and \textit{e} edges is 
defined by $H = [a_{ij}]$. Where the number of rows and columns are defined by the amount of vertices, i, and edges, j, respectively.
In hydraulic networks contest, the edges have a direction associated with them, this characteristic leads to have a directed incidence matrix:

\begin{equation}
\label{DiGraph}
 a_{ij} =
		\left\{
		\begin{array}{ll}
		
		1 			&      \text{if the edges is incidence out of the node}	
\\
		-1                       &     \text{if the edges is incidence into the node}
\\

                0                       &      \text{otherwise}

		\end{array}
		\right.
\end{equation}	

%The critical points in a closed hydraulic system are usually the inlets and outlets of elements such as pipes, pumps and valves. 
%This components are considered as nodes and the paths connecting them are denoted as edges. 
In \appref{IncidenceSection} the corresponding incidence matrix of the system is 
shown. % In this particular hydraulic network.  H is conformed by $25$ edges and $18$ nodes.  

%Regarding the rank of the H matrix it is necessary to point out that the last row is result of the rows above it. Since the rows conforming the incidence matrix are not linearly independent the 
%rank of the H matrix is less than n, this yields the rank to be $\text{rank(A)} \leq 
%n-1$.

\subsection{Cycle Matrix}
\label{CycleSection}
A spanning tree, T, of a G is a subgraph which contains all vertices of G but has no cycles. 
In order to obtain the spanning tree it is necessary to delete an edge from each 
circuit of the graph, the removed edges are called chords. To obtain the number 
of chords, \textit{l}, the following relationship is applied:

\begin{equation}
  \label{Numberofchords}
  l = e - n +1
\end{equation}

Adding any one chord to T, it is created one circuit which is called fundamental 
circuit. A graph is conformed by as many fundamental circuits as number of chords it has.  
This set of fundamental circuits correspond to the fundamental circuit matrix 
$B_f$, such as the number of rows and columns are defined by the amount of chords and edges, respectively. 

Similarly to the incidence matrix, the edges conforming the fundamental circuit 
matrix have a direction associated with them. Then, the B matrix can be 
expressed with $B = [b_{ij}]$ and

\begin{equation}
\label{DiGraphCycle}
 b_{ij} =
		\left\{
		\begin{array}{ll}
		
		1 			&      \text{if the edges is in the circuit i and the directions match}	
\\
		-1                       &     \text{if the edges is in the circuit i and the directions are opposite}
\\

                0                       &      \text{otherwise}

		\end{array}
		\right.
\end{equation}	

In \appref{CycleAppendix} the corresponding incidence matrix of the system is 
shown.


\todo{For the graph representation an idea for the logical order of the section:}
1. Describe how the system diagram can be expressed as a graph \\
2. Describe graph theory(matrices, relations) and our graph\\
3. Derive/describe the matrix/vector equations between dynamics and graph theory\\
4. KVL, KCL does it apply to the system 
