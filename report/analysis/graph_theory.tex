\section{Graph Theory}  
\label{GraphTheory}
A graph is a formal mathematical way of representation of a network. In general, the structure of the graph is given by mean of engineering approaches e.g., 
mechanicals systems, electrical circuits or hydraulic networks \cite{graph_intro}. 

In order to make the modelling of the water distribution network the Graph Theory is used. The graph is constructed based on the affinity between both
electrical circuits and hydraulic networks. Therefore, an analogy between hydraulic and electrical circuits is done by applying Kirchhoff´s law, 
which represents the interconnection of the elements of an electrical circuit.  
From the graphical point of view the pressures and flows will be considered as 
voltage and currents, respectively. 

\subsection{Incidence Matrix} 

%The critical points in a closed hydraulic system are usually the inlets and outlets of elements such as pipes, pumps and valves. 
%The pressure drops taking place in the pipes and valves can be considered as inductors and nonlinear resistors, respectively. 
%
%\textit{Maybe here before we start with the Graph Theory could be nice to make a figure showing the changes between the hydraulic circuit to the electrical 
%one.}
%
%The analogy between hydraulic and electrical circuits described above is applied in the graph theory…(start explaining Graph Theory).


 
 
 