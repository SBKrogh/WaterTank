\section{Hydraulic Model}
\label{HydraulicModel}

Water distribution networks are designed to deliver water  to consumers in terms of sufficient pressure and appropriate chemical composition. Distribution systems as such are generally conformed by four main components: pipes, pumps, valves and reservoirs. The common property of pipes, pumps and valves is that they are all two-terminal components, therefore they can be characterized by two algebraic or dynamic relationship between two variables. The pressure drop accross the two endpoints and the flow through the element, as the state vector of one component shows in \eqref{ComponentState}. \cite{Kallesoe2009} 

\begin{equation}
\label{ComponentState}
 \begin{bmatrix}
    \Delta P \\
    q
\end{bmatrix}
=
 \begin{bmatrix}
    P_{in} - P_{out} \\
    q
\end{bmatrix}
\end{equation}
\\
In the following chapter the hydraulic model analysis of the system is done by control volume approach \cite{Hunt_Fluidmechanics}, 
and the relationship of the two state parameters of each component is introduced.

\subsection{Pipe Model} 
\label{PipeModel}
Pipes are major components of water distribution systems since they are used for carrying pressurized and treated fresh water. A detailed model of pipes has to be derived in a general form in order to describe the relationship of pressure and flow in each pipe component in the network.  
%
The dynamic model of a pipe can be originated from Newton's second law. \eqref{NewtonLaw} describes that the rate of change of the momentum of the fluid in the pipe is proportional to the force acting on it.
\begin{equation}
  \frac{d}{dt} P = \sum F
  \label{NewtonLaw}
\end{equation} 

\begin{minipage}[t]{0.20\textwidth}
Where\\
\hspace*{8mm} $F$ \\
\hspace*{8mm} $P$ 
\end{minipage}
\begin{minipage}[t]{0.68\textwidth}
\vspace*{2mm}
is the force acting on the water,\\
is the linear momentum of the water flow.

\end{minipage}
\begin{minipage}[t]{0.10\textwidth}
\vspace*{2mm}
\textcolor{White}{te}$\unit{N}$\\
\textcolor{White}{te}$\unit{\frac{kgm}{s}}$
\end{minipage}

The dynamics of a pipe component is derived under the assumption that the flow of the fluid is uniformly distributed along the cross sectional area of the pipe and the flow is turbulent. In other words, all pipes in the system are filled up fully with water all the time and the water is assumed to be incompressible. Thus the density of water and the volume of the fluid is constant in time, as the mass of the water is.
\\
Rewriting \ref{NewtonLaw} to be dependent on the time derivative of the product of the mass and velocity, the mass of the water can be taken out in front of the derivative.

\begin{equation}
  \frac{d}{dt} P = {\frac{d(M v)}{dt}} = M \frac{dv}{dt} = \sum F
\end{equation} 

\begin{minipage}[t]{0.20\textwidth}
Where\\
\hspace*{8mm} $M$ \\
\hspace*{8mm} $v$ 
\end{minipage}
\begin{minipage}[t]{0.68\textwidth}
\vspace*{2mm}
is the mass of the water,\\
is the absolute value of the velocity of the water.

\end{minipage}
\begin{minipage}[t]{0.10\textwidth}
\vspace*{2mm}
\textcolor{White}{te}$\unit{kg}$\\
\textcolor{White}{te}$\unit{\frac{m}{s}}$
\end{minipage}

The sum of the forces working on the control volume can be listed as: input forces, output forces and resistance forces.  
These forces will be converted in terms of pressure in order to obtain the model of the pressure drop in the pipes. 

The diameter of the pipe is assumed constant and since they are cylindrical pipes as well:
\begin{equation}
  A_{in} = A_{out} = \frac{1}{4}\pi D^{2}
\end{equation}

The equation of continuity is applied to analyze the pipe flow. It is assumed a steady flow and an uniform 
mass flow in the pipe. Hence, the water flow can be described as: 

\begin{equation}
  Q=Area \cdot V
	\label{EquationOfContinuity}
\end{equation}

Where: 

\begin{itemize}
  \item  $A$ = Area of a pipe
  \item $V$ = Average flow velocity
  \item $Q$ = Volumetric rate of flow
\end{itemize}

\begin{equation}
  M_w \frac{dv}{dt} = F_in - F_out - F_res \\
\end{equation}

\begin{equation}
    A L \rho \frac{dv}{dt} = A \textit{p}_{in} - A \textit{p}_{out} - F_{res} \\
\end{equation}

\begin{equation}
    A L \rho \frac{d}{dt} \frac{q}{A} = A \textit{p}_{in} - A \textit{p}_{out} - F_{res} \\
\end{equation}

\begin{equation}
    \frac{L \rho}{A} \frac{dq}{dt} =p_{in} - p_{out} - \frac{ F_{res}}{A} \\
\end{equation}

\begin{equation}
    \frac{L \rho}{A} \frac{dq}{dt} =\Delta p - \frac{ F_{res}}{A} \\
\end{equation}


On the other hand, The Bernoulli Equation is considered to obtain the conservation of energy principle for the fluids flowing in the pipe.

\begin{equation}
  \label{Bernoulliequation}
z_1 + h_1 + \frac{v_1^{2}}{2g} = z_2 + h_2 + \frac{v_2^{2}}{2g} + h_L
\end{equation}

(This equation its a derivation of the general Bernoulli equation of energy but divided by the density and 
gravity)\\

\begin{itemize}
  \item   z1 and z2 are the elevations of the centerline
  \item   h1 and h2 are pressure heads
  \item   v1 and v2 average flow velocity
  \item   hl loss between sections 1 and 2
\end{itemize}

The $h_L$ term is conformed by two different losses. One due to the surface resistance $(h_F)$ and the other one due to form resistance $(h_M)$. 

\begin{equation}
  h_L = h_F + h_M
\end{equation}

\textbf{Surface Resistance} \\ 

\textbf{Form Resistance} \\ 

\textbf{Valve Model}  \\
\label{ValveModel}
 Valves are used in water distribution networks to control the flow through the pipe, the pressure and in 
 the case that there is a reservoir to control its level. 
 
 \textbf{Pump Model} \\
 \label{PumpModel}
 
 \textbf{Kirchhoff´s Law implementation for hydraulic networks}\\
 \label{KirchhoffLaw}
 Kirchhoff´s law represents the interconnection of the elements of an electrical circuit. An hydraulic approach 
 can be done to Kirchhoff´s law in order to carry out the analysis of the hydraulic network of a system, having to satisfied Kirchhoff´s 
 current law (KCL) and voltage law (KVL). Therefore, voltage and currents are replaced by pressure and flows, respectively. 

The critical points in a closed hydraulic system are usually the inlets and outlets of elements such as pipes, pumps and valves. 
The pressure drops taking place in the pipes and valves can be considered as inductors and nonlinear resistors, respectively. 

\textit{Maybe here before we start with the Graph Theory could be nice to make a figure showing the changes between the hydraulic circuit to the electrical 
one.}

The analogy between hydraulic and electrical circuits described above is applied in the graph theory…(start explaining Graph Theory).






