\section{Hydraulic Model}
\label{HydraulicModel}

Water distribution networks are designed to provide good water quality to the community and industries, in 
terms of sufficient pressure and appropriate chemical levels. In order to supply this service, a water distribution 
system is conformed by four main components: pipes, pumps, valves and reservoirs. 
\\

The understanding of a water distribution is done in the following chapter by the hydraulic approach, 
where the models of each component of the network will be defined. 

\textbf{Pipe Model} \\ 
\label{PipeModel}
Pipes are the major components of a water distribution system, considering that they are commonly used to carry the flow. 
Therefore especial attention should be focused in the design of the types of pipes and in the setup of the dynamic model, 
thus a detailed model can be obtained that describes the effect of pressure and flow in a network. 
\\
The equation of continuity is applied to analyze the pipe flow. It is assumed a steady flow and an uniform 
mass flow in the pipe. Hence, the water flow can be described as: 

\begin{equation}
  Q=Area \cdot V
	\label{EquationOfContinuity}
\end{equation}

Where: 

\begin{itemize}
  \item  A = Area of a pipe
  \item V = Average flow velocity
  \item Q = Volumetric rate of flow
\end{itemize}


On the other hand, The Bernoulli Equation is considered to obtain the conservation of energy principle for the fluids flowing in the pipe.

\begin{equation}
  \label{Bernoulliequation}
z_1 + h_1 + \frac{v_1^{2}}{2g} = z_2 + h_2 + \frac{v_2^{2}}{2g} + h_L
\end{equation}

(This equation its a derivation of the general Bernoulli equation of energy but divided by the density and 
gravity)\\

\begin{itemize}
  \item   z1 and z2 are the elevations of the centerline
  \item   h1 and h2 are pressure heads
  \item   v1 and v2 average flow velocity
  \item   hl loss between sections 1 and 2
\end{itemize}

The $h_L$ term is conformed by two different losses. One due to the surface resistance $(h_F)$ and the other one due to form resistance $(h_M)$. 

\begin{equation}
  h_L = h_F + h_M
\end{equation}

\textbf{Surface Resistance} \\ 

\textbf{Form Resistance} \\ 

\textbf{Valve Model}  \\
\label{ValveModel}
 Valves are used in water distribution networks to control the flow through the pipe, the pressure and in 
 the case that there is a reservoir to control its level. 
 
 \textbf{Pump Model} \\
 \label{PumpModel}
 
 \textbf{Kirchhoff´s Law implementation for hydraulic networks}\\
 \label{KirchhoffLaw}
 Kirchhoff´s law represents the interconnection of the elements of an electrical circuit. An hydraulic approach 
 can be done to Kirchhoff´s law in order to carry out the analysis of the hydraulic network of a system, having to satisfied Kirchhoff´s 
 current law (KCL) and voltage law (KVL). Therefore, voltage and currents are replaced by pressure and flows, respectively. 

The critical points in a closed hydraulic system are usually the inlets and outlets of elements such as pipes, pumps and valves. 
The pressure drops taking place in the pipes and valves can be considered as inductors and nonlinear resistors, respectively. 

\textit{Maybe here before we start with the Graph Theory could be nice to make a figure showing the changes between the hydraulic circuit to the electrical 
one.}

The analogy between hydraulic and electrical circuits described above is applied in the graph theory…(start explaining Graph Theory).









