\section{Hydraulic Model}
\label{HydraulicModel}

Water distribution networks are designed to deliver water to consumers in terms of sufficient pressure and appropriate chemical composition. Distribution systems as such are generally consist of four main components: pipes, pumps, valves and reservoirs. The common property is that they are all two-terminal components, therefore they can be characterized by the dynamic relationship between the pressure drop accross the two endpoints and the flow through the element.\cite{Kallesoe2009}  \eqref{ComponentState} shows the state vector of one component. 

\begin{equation}
\label{ComponentState}
 \begin{bmatrix}
    \Delta P \\
    q
\end{bmatrix}
=
 \begin{bmatrix}
    P_{in} - P_{out} \\
    q
\end{bmatrix}
\end{equation}
\\
In the following chapter the hydraulic model analysis of the system is done by control volume approach.\cite{Hunt_Fluidmechanics} 
The relationship between the two state parameters are introduced for each component in the hydraulic network.


\subsection{Pipe Model} 
\label{PipeModel}
Pipes are major components of water distribution systems since they are used for carrying pressurized and treated fresh water. A detailed model of pipes has to be derived in a general form in order to describe the relationship of pressure and flow for each pipe component.  
%
The dynamic model of a pipe can be originated from Newton's second law. \eqref{NewtonLaw} describes the proportionality between the rate of change of the momentum of the fluid(water) and the force acting on it.

\begin{equation}
  \frac{d}{dt} P = \sum F
  \label{NewtonLaw}
\end{equation} 

\begin{minipage}[t]{0.20\textwidth}
Where\\
\hspace*{8mm} $F$ \\
\hspace*{8mm} $P$ 
\end{minipage}
\begin{minipage}[t]{0.68\textwidth}
\vspace*{2mm}
is the force acting on the water,\\
is the linear momentum of the water flow.

\end{minipage}
\begin{minipage}[t]{0.10\textwidth}
\vspace*{2mm}
\textcolor{White}{te}$\unit{N}$\\
\textcolor{White}{te}$\unit{\frac{kgm}{s}}$
\end{minipage}

The dynamics of a pipe component is derived under the assumption that the flow of the fluid is uniformly distributed along the cross sectional area of the pipe and the flow is turbulent. In other words, all pipes in the system are filled up fully with water all the time and the water is assumed to be incompressible. Thus the density of water and the volume of the fluid is constant in time, as the mass of the water is.
\\
Rewriting \eqref{NewtonLaw}, because of the above-mentioned consideration, the mass of the water can be taken out in front of the derivative.

\begin{equation}
  \frac{d}{dt} P = {\frac{d(M v)}{dt}} = M \frac{dv}{dt} = \sum F
\end{equation} 

\begin{minipage}[t]{0.20\textwidth}
Where\\
\hspace*{8mm} $M$ \\
\hspace*{8mm} $v$ 
\end{minipage}
\begin{minipage}[t]{0.68\textwidth}
\vspace*{2mm}
is the mass of the water,\\
is the absolute value of the velocity of the water at each point of the pipe.

\end{minipage}
\begin{minipage}[t]{0.10\textwidth}
\vspace*{2mm}
\textcolor{White}{te}$\unit{kg}$\\
\textcolor{White}{te}$\unit{\frac{m}{s}}$
\end{minipage}

The sum of the forces acting on the control volume can be seen as input forces(acting on the inlet of the pipe), output forces(acting on the outlet) and resistance forces.  
These forces are expressed in terms of pressure in order to obtain the model of the pressure drop in the pipes. 

The diameter of the pipe is assumed to be constant and to have cylindrical structure:
\begin{equation}
  A_{in} = A_{out} = \frac{1}{4}\pi D^{2}
\end{equation}

Continuity law in fluid mechanics is applied to analyse the flow in the pipe.\cite{Hunt_Fluidmechanics} A steady and uniform mass flow is assumed in the pipe. Hence, the water flow can be described as: 
\begin{equation}
  q=A \cdot V
	\label{EquationOfContinuity}
\end{equation}

 \begin{minipage}[t]{0.20\textwidth}
Where\\
\hspace*{8mm} $A$ \\
\hspace*{8mm} $q$ 
\end{minipage}
\begin{minipage}[t]{0.68\textwidth}
\vspace*{2mm}
is the cross sectional area of a pipe\\
is the volumetric rate of flow.

\end{minipage}
\begin{minipage}[t]{0.10\textwidth}
\vspace*{2mm}
\textcolor{White}{te}$\unit{m^{2}}$\\
\textcolor{White}{te}$\unit{\frac{m^{3}}{s}}$
\end{minipage}

 In \eqref{LinearMomentum} the forces acting on the pipe are included, the difference between $F_{in}$ and $F_{out}$ represents the pressure drop 
 between two endpoints.

\begin{equation}
  M \frac{dv}{dt} = F_{in} - F_{out} - F_{res} \\
  \label{LinearMomentum}
\end{equation}

In order to obtain an equation consisting of only pressure parameters, the relationship between forces and pressures is used.


 \begin{equation}
    A L \rho \frac{dv}{dt} = A \textit{p}_{in} - A \textit{p}_{out} - F_{res} \\
\end{equation}

The velocity can be written in terms of volumetric water flow and cross sectional area according to the continuity law.

\begin{equation}
    A L \rho \frac{d}{dt} \frac{q}{A} = A \textit{p}_{in} - A \textit{p}_{out} - F_{res} \\
\end{equation}

Reducing the cross sectional area to obtain an expression for the pressure: 

\begin{equation}
    \frac{L \rho}{A} \frac{dq}{dt} =p_{in} - p_{out} - \frac{ F_{res}}{A} \\
\end{equation}

Thus the desired pressure drop between two endpoint is obtained. \eqref{PressureDrop} differential equation describes the change in flow as a function of the pressure drops in the system.
\begin{equation}
    \frac{L \rho}{A} \frac{dq}{dt} =\Delta p - \frac{ F_{res}}{A} \\
    \label{PressureDrop}
\end{equation}

 \begin{minipage}[t]{0.20\textwidth}
Where\\
\hspace*{8mm} $L$ \\
\hspace*{8mm} $F_{res}$ \\
\hspace*{8mm} $\Delta p$ \\
\hspace*{8mm} $\rho$ 
\end{minipage}
\begin{minipage}[t]{0.68\textwidth}
\vspace*{2mm}
is the length of a pipe\\
is the resistance force\\
is the pressure drop across the two endpoints\\
is the density of water.

\end{minipage}
\begin{minipage}[t]{0.10\textwidth}
\vspace*{2mm}
\textcolor{White}{te}$\unit{m}$\\
\textcolor{White}{te}$\unit{N}$\\
\textcolor{White}{te}$\unit{Pa}$\\
\textcolor{White}{te}$\unit{\frac{kg}{m^{3}}}$
\end{minipage}

In \eqref{PressureDrop} the term $F_{res}$ is the resistance force acting on the 
pipe, which consists of two parts: surface resistance($h_{f}$), the friction 
loss, form resistance($h_{m}$) and the resistance due to change of elevation, $\Delta z$. 

\subsubsection{Surface Resistance ($h_f$)} 
The flow of a liquid through a pipe suffers resistance from 
the turbulence occurring along the internal walls of the pipe, caused by the roughness of the surface. This surface resistance is given by the Darcy-Weisbach equation \cite{Design_Water}.

\begin{equation}
  h_f = \frac{fLV^2}{2gD}
  \label{Darcy}
\end{equation}

 \begin{minipage}[t]{0.20\textwidth}
Where\\
\hspace*{8mm} $f$ \\
\hspace*{8mm} $g$ \\
\hspace*{8mm} $D$ 
\end{minipage}
\begin{minipage}[t]{0.68\textwidth}
\vspace*{2mm}
is the Moody friction factor\\ 
is acceleration due to gravity\\
is the diameter of the pipe.
\end{minipage}
\begin{minipage}[t]{0.10\textwidth}
\vspace*{2mm}
\textcolor{White}{te}$\unit{-}$\\
\textcolor{White}{te}$\unit{\frac{m}{s^2}}$\\
\textcolor{White}{te}$\unit{m}$
\end{minipage}

Applying the continuity law, the velocity can be substituted by the volumetric flow and pipe area, resulting in:

\begin{equation}
  h_f = \frac{8fLq^2}{\pi^{2}gD^5}
  \label{DarcyWeisbach}
\end{equation} 
 
 The unknown parameter in \refeq{DarcyWeisbach} is the Moody friction factor 
 which is non-dimensional and is a function of the Reynold´s number. This friction factor depends on the flow if it is laminar, transient or turbulent, and the roughness of the tube. \\
 
 
The Reynold´s number can be used to determine the regime of the flow. When $Re<2300$ as laminar, if $2300<Re<4000$ as transient and if
$Re>4000$ as turbulent \cite{Intro_Fluid}. 
\begin{equation}
   Re = \frac{vD}{\nu}
   \label{Reynolds}
 \end{equation}
 
  \begin{minipage}[t]{0.20\textwidth}
Where\\
\hspace*{8mm} $\nu$ 
\end{minipage}
\begin{minipage}[t]{0.68\textwidth}
\vspace*{2mm}
is the kinematic viscosity

\end{minipage}
\begin{minipage}[t]{0.10\textwidth}
\vspace*{2mm}
\textcolor{White}{te}$\unit{\frac{kg}{ms}}$
\end{minipage}

The kinematic viscosity in \cite{Design_Water} is given by :
\begin{equation}
  \nu = 1.792 \cdot 10^{-6} \bigg[1+{\bigg(\frac{T}{25}\bigg)}^{1.165} \bigg]^{-1}
\end{equation}

  \begin{minipage}[t]{0.20\textwidth}
Where\\
\hspace*{8mm} $T$ 
\end{minipage}
\begin{minipage}[t]{0.68\textwidth}
\vspace*{2mm}
is the water temperature  
 \end{minipage}
\begin{minipage}[t]{0.10\textwidth}
\vspace*{2mm}
\textcolor{White}{te}$\unit{^{\circ} C}$
\end{minipage}

In order to estimate the range of the Reynolds number of a common water 
distribution, typical values of the temperature, velocity and the radius of 
the pipes are considered.\cite{Urban_Design}. 

\begin{itemize}
  \item $v = 0.5 - 1.5  \quad \frac{m}{s}$
  \item $D = 50 - 1500\quad mm$
  \item $T = 10 - 20 \quad ^{\circ} C$
\end{itemize}

These values result in a Reynold's number between 19000 and 225000, which yields a turbulent fluid flow. For turbulent flow the 
Moody friction factor is given by \cite{Design_Water} : 

\begin{equation}
  f = 1.325 \bigg(ln\bigg(\frac{\epsilon}{3.7 D}+\frac{5.74}{R^{0.9}}\bigg)\bigg)^{-2}
  \label{turbulent}
\end{equation}

\begin{minipage}[t]{0.20\textwidth}
Where\\
\hspace*{8mm} $\epsilon$ 
\end{minipage}
\begin{minipage}[t]{0.68\textwidth}
\vspace*{2mm}
is the average roughness of the wall inside the pipe.
 \end{minipage}
\begin{minipage}[t]{0.10\textwidth}
\vspace*{2mm}
\textcolor{White}{te}$\unit{-}$
\end{minipage}

%
\subsubsection{Form Resistance ($h_m$)} 
Form resistance losses appear at any time the flow changes direction, due to elbows, bends,
or due to enlargers and reducers. It is a particular frictional resistance due to the 
fittings of a pipe. Form loss can be expressed as:
\begin{equation}
  h_m = k_f \frac{V^2}{2g}
\end{equation}


Applying the continuity law:

\begin{equation}
   h_m = k_f \frac{8q^2}{\pi^2gD^4}
\label{Formloss}
\end{equation}

 \begin{minipage}[t]{0.20\textwidth}
Where\\
\hspace*{8mm} $k_f$ 
\end{minipage}
\begin{minipage}[t]{0.68\textwidth}
\vspace*{2mm}
is the form-loss coefficient.  
 \end{minipage}
\begin{minipage}[t]{0.10\textwidth}
\vspace*{2mm}
\textcolor{White}{te}$\unit{-}$
\end{minipage}

The form-loss coefficient can be split into different loss depending on the 
fitting of the pipes: pipe bend and elbows. 
\\
Pipe bends principally is determined by the 
bend angle $\alpha$ and bend radius $R$, it is given by the following 
expression \cite{Design_Water}: 

\begin{equation}
  k_f = \bigg[0.0733 + 0.923 \bigg(\frac{D}{R}\bigg)^{3.5}\bigg]\alpha^{0.5}
\end{equation}

Pipe elbows are also used to change the direction of the flow but providing 
sharp turns in pipelines. The coefficient for the losses in elbows is determined by the angle of an elbow $\alpha$ and is given by:

\begin{equation}
  k_f = 0.442\alpha^{2.17}
\end{equation}

\subsubsection{Complete Pipe Model}
In \eqref{DarcyWeisbach} and \eqref{Formloss}, the head loss of the friction losses are determined. These terms are introduced in \eqref{PressureDrop} in terms of pressure. Thus, the friction factors are multiplied by the water density and gravity. Nevertheless, the head loss due to elevation has to be added in the model, yielding the final expression:

\begin{equation}
   \frac{L \rho}{A} \frac{dq}{dt} =\Delta p - h_f \rho g - h_m \rho g - \Delta z \rho 
   g
\end{equation}

Substituting the terms $h_f$ and $h_m$ with their respective values:
\begin{equation}
   \frac{L \rho}{A} \frac{dq}{dt} =\Delta p - \frac{8fLq^2}{\pi^{2}gD^5} \rho g - k_f \frac{8q^2}{\pi^2gD^4} \rho g - \Delta z \rho g 
\end{equation}

Taking out the flow $q$ as the common factor:
\begin{equation}
\label{FinalPipeModel}
   \frac{L \rho}{A} \frac{dq}{dt} =\Delta p - \frac{8fLq^2}{\pi^{2}gD^5} \rho g - k_f \frac{8q^2}{\pi^2gD^4} \rho g - \Delta z \rho g 
\end{equation}

\eqref{FinalPipeModel} describes the rate of flow in terms of pressure losses due to pressure change, frictions and elevation. A more compact form can be expressed for the $k$th component as such:

\begin{equation}
\label{FinalPipeModelCompact}
   J_k \dot{q_k} = \Delta p_k - \lambda_k(q_k) - \zeta_k 
\end{equation}

 \begin{minipage}[t]{0.20\textwidth}
Where\\
\hspace*{8mm} $J_k$ \\
\hspace*{8mm} $\lambda_k(q_k)$ \\
\hspace*{8mm} $\zeta_k$ 
\end{minipage}
\begin{minipage}[t]{0.68\textwidth}
\vspace*{2mm}
is an analogous parameter as inertia for the pipe\\ 
is the friction as a function of flow\\
is the friction due to the elevation.
\end{minipage}

As can be seen in \eqref{FinalPipeModelCompact}, the dynamics of the pump is described by $J_k$ which is an analogous parameter as inertia in mechanical systems. However, the presence of the water tower in the system has a slow effect on the flow due to its slow integration behaviour. In other words it means that the water tank has a relatively bigger time constant compared to the time constant of the pipe. Due to this consideration it is a fair assumption that the dynamics of the pipe will not influence the flow significantly in the system, therefore it can be neglected. \eqref{FinalPipeModelCompact} then simplifies to the following expression: 
  
\begin{equation}
\label{FinalPipeModelSimplified}
  \Delta p_k  =   \lambda_k(q_k) + \zeta_k 
\end{equation}


