\section{Linear Parameter Estimation} 
\label{LinParamEst}

\subsection{Linearization of Multistate Nonlinear Functions}
\label{LinearizationIntro}

asdf

\begin{equation}
 \hat{\Delta p} = \Delta p -  \overset{*}{\Delta p}
\label{p_operating}
\end{equation}

asdf

\begin{equation}
 \hat{z} = z -  \overset{*}{z}
\label{p_operating}
\end{equation}

\subsection{Linearization of System Model}
\label{LinearizationIntro}

asdf

\begin{equation}
 \dot{\Delta p_{WT}} = \frac{1}{C_H} q_{WT}
\label{WTeq1}
\end{equation}

\begin{equation}
 \pmb{ \Delta p} = \pmb{J \dot{q_1}} + \lambda(\pmb{q_1}) + \zeta + \mu(\pmb{q_1}, \pmb{OD}) - \alpha(\pmb{dp})
\label{NonlinearPressureFunction}
\end{equation}

asdf

\begin{equation}
 \dot{\Delta p_{WT}} = \frac{1}{C_H} q_{WT}
\label{WTeq2}
\end{equation}

asdf

\begin{equation}
\pmb{\Delta p_0} =
\begin{bmatrix}
         \Delta p_{C32} \\
	\Delta p_{WT}
\end{bmatrix}
\label{pvector}
\end{equation}

asdf

\begin{equation}
\pmb{q_0} =
\begin{bmatrix}
         q_{C32} \\
	q_{WT}
\end{bmatrix}
\label{qvector}
\end{equation}

Recalling \eqref{InputOutputmodel_steadystate}:

\begin{equation}
 \pmb{B}\pmb{J} \pmb{\dot{q}} = \pmb{B} g(\pmb{q})+ \pmb{B_1} u(\pmb{\omega},\pmb{OD})
 \label{InputOutputmodel_steadystate_linmodel}
\end{equation}

\begin{equation}
\label{gfunction}
 g_{i}(q) =
		\left\{
		\begin{array}{ll}
		
		\lambda_i(q_1) 			&      \text{for i = ...}	
\\
		\lambda_i(q_1) + \zeta_i                      &     \text{for i = ...}
\\

                \Delta p_{WT;i}                       &      \text{for i = ...}
\\

                \mu_{i;q1}(q_1, OD)                       &      \text{for i = ...}

		\end{array}
		\right.
\end{equation}	

\begin{equation}
\label{ufunction}
 u_{i}(\omega, OD) =
		\left\{
		\begin{array}{ll}
		
		\alpha_i(\omega) 			&      \text{for i = ...}	
\\
		\mu_{i;OD}(q_1, OD)                      &     \text{for i = ...}

		\end{array}
		\right.
\end{equation}	

The linearization procedure of both $g(\pmb{q})$ and $u(\pmb{\omega},\pmb{OD})$, is shown:

\begin{equation}
  \begin{split}
  & B g_{i}({B_{1}}^{T}z) = B_1 \frac{\partial{\lambda_{i}({{B_{1}}^{T}z})}}{\partial{B_{1}^{T}z}} {B_1}^{T} \quad + \quad
  B_0 {\hat{\Delta P}}_{WT} \quad + \quad B_1 \frac{\partial{\mu_{i}({{B_{1}}^{T}z})}}{\partial{B_{1}^{T}z}} {B_1}^{T} \quad - \\
  &  B_0 \alpha_{i}(B_{1}^{T}z) \quad + \quad B_0 \frac{\partial{\mu_{i}({{B_{0}}^{T}z})}}{\partial{B_{0}^{T}z}} {B_0}^{T}
  \end{split}
  \label{StateLinear}
\end{equation}

\begin{equation}
  B \mu_{i}(\omega, OD, {B_1}^{T}z) = - B_1 \alpha_{i}(dP) \quad + \quad B_1 \frac{\partial{\mu_{i}(OD)}}{\partial{OD}} {B_1}^{T}
  \quad + \quad B_0 \frac{\partial{\mu_{i}(OD)}}{\partial{OD}} {B_0}^{T}
  \label{inputlinear}
\end{equation}



\subsection{Linear Model Implementation}
\label{MatlabScriptLinear}

Unlike for the nonlinear model, in the linear model the WT is taken into consideration. Therefore, the state vector is augmented into a $9x1$ vector, 
containing the 8 chord flows and the pressure across the WT. For the linear model a state-space system is designed.

The state matrix, $A$, is conformed by the linearized expression of the pipes, the linearized part of the valve function corresponding to the state and pressure contribution
of the WT.

Pipes contribution:

\begin{equation}
  {\lambda}_e = (2 * Cp * {B_1}^{T} \, \bar{z} /(10^5*3600^2))*\hat{z}
  \label{lambdafun}
\end{equation}

\eqref{lambdafun} only affects to the 8 chords of the water distribution, $\lambda$ ends up being a $25x1$ matrix. However, we will have a $\lambda$ expression for all pipes
edges, resulting in a $25x25$ setting $0$ for those non-pipe edges. 

Valve contribution:

\vspace{4mm}
\begin{equation}
  {\mu}_e =  2 \, e^{\frac{2 \, (\theta_{off} - \bar{OD}) \, n_{gl}}{\theta_{max}-\theta_{off}}+2} \, {B_1}^{T} \, \bar{z} \, \hat{z}
  \label{mufun}
\end{equation}

The above expression has been derived in \appref{chap:Lin}, for the state matrix only the part corresponding to the states is taken. In the same way 
as in the pipe, the $\mu$ expression for each edge is a $25x1$ matrix. This extended to all the edges will end up being a $25x25$ matrix, with $0$ in the 
non-valve edges. 

When augmenting the system with the WT, two new edges show up in the water distribution. This two new edges are the WT itself and a pump connected to the WT.
The pump connected to the WT is running with $0$ rotational speed, thus the mathematical formulation results in:

\begin{equation}
    \Delta p = -a_{h2}{q_i}^2 
\end{equation}

Along with the pump there are two pumps, gathering all the components expressions:

\begin{equation}
      \Delta p = (\frac{2}{{k_v}^2-a_{h2}}){q_i}^2 
\end{equation} 

The above equation needs to be linearized and consequently separate into the state and the input part, see \appref{chap:Lin}. The contribution to the $A$ matrix is as 
following:

\begin{equation}
  \alpha = 4 \, e^{\frac{2 \, (\theta_{off} - \bar{OD}) \, n_{gl}}{\theta_{max}-\theta_{off}}+2} \, {B_0}^{T} \bar{z} \, \hat{z} - 2 a_{h22} {B_0}^{T} \hat{z}
\end{equation}

The cycle matrix, $B_0$ corresponding to the augmented system is used with dimensions $8x2$ but the state vector of the 8 flow chords.
 The $\alpha$ function results in a $2x2$ matrix with $0$ in the column corresponding to the non-pump edge. 

In regards to the WT edge, the pressure across it is treated as a new state for the system, that is, the ninth state $\Delta p_{WT}$. 

Once the terms conforming the $A$ matrix are identified, the correct structure due to the linearization has to be applied, see \eqref{StateLinear} and \eqref{inputlinear}, resulting in
the following $A$ matrix.

\begin{equation}
  A = B_1 \, \lambda \, {B_1}^T + B_1 \, \mu \, {B_1}^T + B_0 \, \alpha \, {B_0}^T
\end{equation}

The remain $A$ matrix has the dimension $8x8$ and the new state $\Delta p_{WT}$ of the WT has to be included into the matrix in order to get an $9x9$ $A$ 
matrix. 

The input matrix, $B$, is conformed by the input into the system. These are the opening degree of the valves and the pressure contribution of the main 
and PMA pumps. 

Previously, for the $A$ matrix the valve expression has been linearized, this time the input term of the linearized expression will be taken:

\begin{equation}
  {\mu}_e = -2 \, \frac{e^{\frac{2 \, (\theta_{off} - \bar{OD}) \, n_{gl}}{\theta_{max}-\theta_{off}}+2}
   \, n_{gl} \,  \bar{q} \, \bar{q}}{\theta_{max}-\theta_{off}} \, \hat{OD}
\end{equation}

The pumps pressure contribution is also added into the input matrix. 



 
