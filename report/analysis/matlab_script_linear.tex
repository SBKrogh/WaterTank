\section{Linear Model Implementation} 
\label{MatlabScriptLinear}

Unlike for the nonlinear model, in the linear model the WT is taken into consideration. Therefore, the state vector is augmented into a $9x1$ vector, 
containing the 8 chord flows and the pressure across the WT. For the linear model a state-space system is designed.

The state matrix, $A$, is conformed by the linearized friction provided by the pipes, the linearized part of the valve function corresponding to the state and pressure contribution
of the WT.

Pipes contribution:

\begin{equation}
  {\lambda}_e = (2 * Cp * {B_1}^{T} \, abs(\hat{z}) /(10^5*3600^2))
  \label{lambdafun}
\end{equation}

\eqref{lambdafun} only affects to the 8 chords of the water distribution, lambda ends up being a $25x1$ matrix. However, we will have a lambda expression for all pipes
edges, resulting in a $25x25$ adding $0$ for those non-pipe edges. 

Valve contribution:

\vspace{4mm}
\begin{equation}
  {\mu}_e = e^{\frac{2 \, (\theta_{off} - \bar{OD}) \, n_{gl}}{\theta_{max}-\theta_{off}}+2} \, \bar{z} \, |\bar{z}| \quad \quad - \\
  2 \, e^{\frac{2 \, (\theta_{off} - \bar{OD}) \, n_{gl}}{\theta_{max}-\theta_{off}}+2} \, {B_1}^{T} abs(\hat{z}) \, |\bar{z}|
  \label{mufun}
\end{equation}

The above expression has been derived in \appref{chap:Lin}, for the state matrix only the part corresponding to the states is taken. In the same way 
as in the pipe, the $\mu$ expression for each edge is a $25x1$ matrix. This extended to all the edges will end up into a $25x25$ matrix, with $0$ in the 
non-valve edges. 

When augmenting the system with the WT, two new edges show up in the water distribution. This two new edges are the WT itself and a pump connected to the WT.
The pump connected to the WT is running with $0$ rotational speed, thus the mathematical model results in:

\begin{equation}
    \Delta p = -a_{h2}{q_i}^2 
\end{equation}

Along with the pump there are two pumps, gathering all the components expressions:

\begin{equation}
      \Delta p = (\frac{2}{{k_v}^2-a_{h2}}){q_i}^2 
\end{equation} \todo{Add the linearization explanation into the Appendix}

The above equation needs to be linearized and consequently separate into the state and the input part. The contribution to the $A$ matrix is as 
following:

\begin{equation}
  \alpha = 2 * e^{\frac{2 \, (\theta_{off} - \bar{OD}) \, n_{gl}}{\theta_{max}-\theta_{off}}+2} \, \bar{z} \, |\bar{z}| \quad - \\
  4 \, e^{\frac{2 \, (\theta_{off} - \bar{OD}) \, n_{gl}}{\theta_{max}-\theta_{off}}+2} \, {B_0}^{T} abs(\hat{z}) \, |\bar{z}| - 2 * {B_0}^{T} abs(\hat{z}) * c.Ah22
\end{equation}

The cycle matrix, $B_0$ corresponding to the augmented system is used with dimensions $8x2$ but the state vector of the 8 flow chords.
 The $\alpha$ function results in a $2x2$ matrix with $0$ in the column corresponding to the non-pump edge. 

In regards to the WT edge, the pressure across it is treated as a new state for the system, that is, the ninth state $\Delta p_{WT}$. 

Once the terms conforming the $A$ matrix are identified, the correct structure due to the linearization has to be applied as explained in \secref{} resulting in
the following $A$ matrix.

\begin{equation}
  A = B_1 \, \lambda \, {B_1}^T + B_1 \, \mu \, {B_1}^T + B_0 \, \alpha \, {B_0}^T
\end{equation}

The remain $A$ matriz has the dimension $8x8$ and the new state $\Delta p_{WT}$ of the WT has to be included into the matrix in order to get an $9x9$ $A$ 
matrix. 

The input matrix, $B$, is conformed by the input into the system. These are the opening degree of the valves and the pressure contribution of the main 
and PMA pumps. 

Previously, for the $A$ matrix the valve expression has been linearized, this time the input term of the linearized expression will be taken:

\begin{equation}
  {\mu}_e = -2 \, \frac{e^{\frac{2 \, (\theta_{off} - \bar{OD}) \, n_{gl}}{\theta_{max}-\theta_{off}}+2}
   \, n_{gl} \, \hat{OD} \, \bar{q} \, |\bar{q}|}{\theta_{max}-\theta_{off}}
\end{equation}

The pumps pressure contribution is also added into the input matrix. 


