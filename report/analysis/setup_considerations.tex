

\section{Setup Considerations}

% This section should contain to things. 

% 1: Req. for water pressure 
% 2: Constrain about water quality 

Adding a WT to an existing water distribution network will introduce constrains and these need to be taken into account. 

As mentioned in \secref{introduction}, a minimum pressure must be maintained at the end user. Furthermore the pressure can not exceed a maximum level as this will both increase the possibility of water leakage and wear on the pipes in the system. The system described in \secref{system_overview}, is designed to operate at a pressure around 0.1 bar, relative to the environment \cite{master_aau}. For the purpose of this project the interval for which the pressure should be within, at the end user, is chosen to be $0.08 < p < 0.14$ [Bar].

Another important aspect when implementing a WT is water quality. If the water is stored, in the WT, for too long the quality will decrease due to decreasing oxygen level. Because of this a requirement for water quality has to be formulated. As described in \secref{system_overview}, the WT has one combined input/output connection. Therefor a requirement only for flow is hard to formulate as the direction will change dependent on the usage. This could result in a flow based constraint being fulfilled by rapidly changing flow direction without actually replacing any significant water volume in the tower. Instead, a requirement for how often the content of the WT should be exchanged per time unit is proposed. For the purpose of this project the minimum requirement to volume exchange, is chosen to 30\% of the maximum volume of the WT, $V_T$ per day. This can be written as $\frac{30\cdot V_T}{100} \: \frac{m^3}{day}$.










 
 