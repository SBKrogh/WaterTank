\section{Model for the Parameter Estimation}  
\label{ParameterEstimation}
Once the corresponding incidence and cycle matrix have been identified and the analogy between hydraulic and
electrical circuits has been made, it is possible to derive a model for the hydraulic network. 

In the parameter estimation the WT will be discarded from the hydraulic system, 
in order to facilitate the process of estimation. Consequently, the WT will be isolated from the other components of the
system.  



In \secref{IncidenceSection} the form of the incidence matrix has been shown. It is worth mentioning that the last column of the $\pmb{H}$ matrix agrees with the WT edge. Hence, the $\pmb{H}$ matrix can be written as 
% It is worth mentioning that it is been also followed the structure such as the last 
% column of the H matrix agrees with the WT edge. Hence, the H matrix can 
% be written as

\begin {equation}
\pmb{H} = [\pmb{H_1} \quad \pmb{H_0}]
\label{Hmatrix}
\end{equation}

\begin{minipage}[t]{0.18\textwidth}
Where\\
\hspace*{8mm} $\pmb{H_1} \epsilon \: \mathbb{R}^{n \times e-1}$  \\
and \hspace*{0.4mm} $\pmb{H_0} \epsilon \: \mathbb{R}^{n \times 1} $ 
\end{minipage}
\begin{minipage}[t]{0.70\textwidth}
\vspace*{2mm}
\hspace*{4mm} is the $\pmb{H}$ matrix without the edge corresponding to the WT,\\
\hspace*{4mm} is the $\pmb{H}$ matrix with the column corresponding to the WT. 
\end{minipage}

In the same way, the fundamental cycle matrix, B, has been structured such as the last column agrees with the WT edge.

\begin{equation}
  \pmb{B} = [\pmb{B_1} \quad \pmb{B_0}]
\end{equation} 

\begin{minipage}[t]{0.18\textwidth}
Where\\
\hspace*{8mm} $\pmb{B_1} \: \epsilon \mathbb{R}^{l \times e-1}$  \\
and \hspace*{0.4mm} $\pmb{B_0} \: \epsilon \mathbb{R}^{l \times 1} $ 
\end{minipage}
\begin{minipage}[t]{0.70\textwidth}
\vspace*{2mm}
\hspace*{4mm} is the $\pmb{B}$ matrix without the edge corresponding to the WT,\\
\hspace*{4mm} is the $\pmb{B}$ matrix with the column corresponding to the WT.
\end{minipage}

As mentioned above, $\pmb{q}$ is a vector containing all the individual flows, which 
is re-structured as following

\begin{equation}
\pmb{q} =
\begin{bmatrix}
         \pmb{q_1} \\
	q_0 
\end{bmatrix}
\label{qmatrix}
\end{equation}

\begin{minipage}[t]{0.20\textwidth}
Where\\
\hspace*{8mm} $\pmb{q_1} \epsilon \mathbb{R}^{e-1 \times 1}$  \\
\hspace*{8mm} $q_0 \epsilon \mathbb{R} $ 
\end{minipage}
\begin{minipage}[t]{0.68\textwidth}
\vspace*{2mm}
\hspace*{4mm} is the flow through all edges expect for WT,\\
\hspace*{4mm} is the flow through the WT. 
\end{minipage}

The vector containing the pressures at the nodes is also re-structured as

\begin{equation}
\pmb{p} =
\begin{bmatrix}
         \pmb{p_1} \\
	p_0 
\end{bmatrix}
\end{equation}

\begin{minipage}[t]{0.20\textwidth}
Where\\
\hspace*{8mm} $\pmb{p_1} \epsilon \mathbb{R}^{n-1 \times 1}$  \\
\hspace*{8mm} $p_0 \epsilon \mathbb{R} $ 
\end{minipage}
\begin{minipage}[t]{0.68\textwidth}
\vspace*{2mm}
\hspace*{4mm} is the pressure at all the nodes expect for WT,\\
\hspace*{4mm} is the pressure in the WT.
\end{minipage}

In \eqref{KCL} KCL law is applied for a connected directed graph, where it is stated that the sum of all the flows entering 
into a node must be equal the the sum of all the nodes out of the node.

By choosing an independent set of flows, corresponding to the chords of a spanning tree, the 
flow through every edge of the hydraulic system can be expressed in terms of the flow through the chords, $z$ \cite{GraphModel}.
Thus, reducing the number of unknowns in the system. These elements of $z$ are called the free 
flows of the system and are independent variables \cite{GraphTheoryCarsten}.

\begin{equation}
  q_1 = B_1 ^{T}  z
  \label{ChordRelation}
\end{equation}

In order to extract the component model into a more generalized form, it is rewritten as a function of flow, $q_1$, angular velocity, $\omega$, and conductivity factor, $k_v$ as follows:

\begin{equation}
  \tilde{f}_i(q_1, \omega, k_v) = \lambda_i(q_1) + \zeta_i + \nu_i(q_1, k_v) - \alpha_i(\omega)
  \label{ComponentFunction}
\end{equation}

\begin{minipage}[t]{0.20\textwidth}
Where\\
\hspace*{2mm} $\tilde{f}_i = -C_{pi} q_i |q_i|$  \\
\hspace*{2mm} $\tilde{f}_i = -C_{vi} q_i |q_i|$  \\
\hspace*{2mm} $\tilde{f}_i = \Big(\frac{2}{k_{v100}^2} - a_{h2i}\Big)|q_i| q_i  + a_{h1i} \omega_{i} q_i + a_{h0i}{\omega_i}^2$  
\end{minipage}
\begin{minipage}[t]{0.68\textwidth}
\vspace*{2mm}
\hspace*{25mm} if i = 2,3,4,5,6,7,10,11,12,14,17,18,19,21,23\\
\hspace*{25mm} if i = 13,15,20,22						   \\
\hspace*{65mm} if i = 1,8,9,16						   
\end{minipage}

\todo{Fixe structuring error}

The following hydraulic network model shows an overall model along with the above considerations. 
Now recall that the inertia matrix, J, was defined in \secref{CompletePipe}.

\begin{equation}
  \Delta p_1 =  J \dot{q}_1 + f(q_1, w, k_v)
  \label{NoTowerModel}
\end{equation}

In \eqref{NoTowerModel} the hydraulic network model is described in terms of the 
flow through all the nodes. In order to reduce the order of the model and hence, 
the amount of unknowns, the chord flows according to \eqref{ChordRelation} are applied. 

\begin{equation}
  \Delta p_1 =  J {B_1}^T \dot{z} + f(z, w, k_v)
  \label{ChordsModel}
\end{equation}

Making use of the identity shown in \eqref{KVL}, the following is obtained

\begin{equation}
  0 = B_1 \Delta p_1 = B_1 [ J {B_1}^T \dot{z} + f(z, w, k_v)] 
 \end{equation}

Isolating the inertia matrix to the left side

\begin{equation}
 - B_1 J {B_1}^T \dot{z}  = B_1 f(z, w, k_v)
 \label{isolateZ}
 \end{equation}

It is desired to know the value of the flow through the chords, hence the above equation is solved 
for $\dot{z}$. In order to be able to invert $(B_1 J {B_1}^T)$ it has to be non-singular i.e. invertible. 

Setting $\mathcal{J} = B_1 J {B_1}^T $, for $\mathcal{J}$ to be positive-definite it has to be a square matrix and its 
determinant has to be non-zero. Observe that $\mathcal{J}$ is

\begin{equation}
  \label{Jequation}
  \mathcal{J} = (I \quad B_f) 
  \begin{pmatrix}
    J_c    &    0    \\
    0       &    J_f
  \end{pmatrix}
  \begin{pmatrix}
    I    \\
    {B_f}^T
  \end{pmatrix}
  = J_c + B_f J_f {B_f}^T
\end{equation}

\begin{minipage}[t]{0.20\textwidth}
Where\\
\hspace*{8mm} $J_c \in \mathbb{R}^{l \times l}$  \\
\hspace*{8mm} $J_f \in \mathbb{R}^{f \times f} $ 
\end{minipage}
\begin{minipage}[t]{0.68\textwidth}
\vspace*{2mm}
\hspace*{4mm} is the inertia in the chords components\\
\hspace*{4mm} is the inertia in the component of the spanning tree 
\end{minipage}

$J_c$ is the diagonal inertia matrix containing the chord elements, since all 
the components corresponding to a chord in $\mathcal{G}$ are pipes, all the 
diagonal terms are positive. Thus, $J_c > 0$. 

Nevertheless, if there was at least a chord corresponding to a non-pipe element, \eqref{Jequation} 
would be positive-definite as long as there is a possibility to create a spanning tree containing all chords as pipe elements from $\mathcal{G}$ \cite{TowerModel}.

For the remaining term $B_f J_f {B_f}^T$, $J_f$ is non-negative matrix as all its elements are zero or pipe´s inertia. 
Multiplying $B_f J_f {B_f}^T$ by a non-zero vector column $\mathbf{x}$ and its transpose $\mathbf{x^{T}}$

\begin{equation}
  \mathbf{x^{T}} B_f J_f {B_f}^T \mathbf{x}
  \label{PosDefi}
\end{equation}

Creating a new variable $y = {B_f}^T \mathbf{x}$ and applying the positive semi-definite matrix definition 
\cite{MatrixBook}

\begin{equation}
  y^{T} J_f y \geqslant 0
  \label{PosDefEq}
\end{equation}

Thus, \eqref{Jequation} is positive definite a sufficient condition for $\mathcal{J}$ being invertible. 

Therefore, the system model can be written as follows

\begin{equation}
   \dot{z}  = - (B_1 J {B_1}^T)^{-1}B_1 f(z, w, k_v)
   \label{ParatModelFinal}
 \end{equation}

\subsection{Model Relations}
\label{ModelRelationSection}

As a consequence of the system model established above, a new set of relations for the parameter estimation can be 
obtained. Starting from the component complete model

\begin{equation}
  \Delta p_1 =  J \dot{q}_1 + f(q_1, w, k_v)
  \label{RecallModel}
\end{equation}

Recall that previously, in \eqref{ChordRelation}, a new state, $z$, has been defined for the independent flows in 
the graph

\begin{equation}
  \Delta p_1 =  J {B_1}^T \dot{z} + f(z, w, k_v)
 \end{equation}

The flow rate through the chords is found in \eqref{ParatModelFinal}, thus the expression for $ \Delta p_1 $ can be rewritten as

\begin{equation}
  \Delta p_1 =  J {B_1}^T [- (B_1 J {B_1}^T)^{-1}B_1 f(z, w, k_v)] + f(z, w, k_v)
  \label{PressureLarge}
 \end{equation}
 
 Rewritten it in a shorter form
 
 \begin{equation}
  \Delta p_1 =  (-J {B_1}^T (B_1 J {B_1}^T)^{-1}B_1 + \mathcal{I}) f(z, w, k_v)
  \label{PressureShort}
 \end{equation}

\subsection{The Estimation Method}
\label{SubSecEstimation}

Applying KVL to the hydraulic model, the following expression is obtained: Moreover, the parameter estimation is carried out for steady-state 
situation where the inertia, $J$, will not act. Thus, the term containing the diagonal matrix $J$ can be disregarded from the equation. 

\begin{equation}
  0 = B_1 \Delta p_1 = B_1 [ J {B_1}^T \dot{z} + f(z, w, k_v)]  =  B_1 (f(z, w, k_v))
  \label{esteq}
 \end{equation}
 
The pressure difference in the pumps have been specified as inputs, by reason of those pressures can be obtained from the setup available so they are considered 
as known parameters. Thereby, in \eqref{esteq} the term can be split up as following:
 
\begin{equation}
 0 = f(z,k_v)+ f(z,w)
 \label{ModelNoInertia}
\end{equation}

The term $f(z,w)$ is set as input, it is renamed as U and it represents the pressure difference in the $4$ pumps acting on the system. Therefore, 
\eqref{ModelNoInertia} is considered as the input equation where the flow through the chords and the friction parameter of the pipes are unknown. 

An output equation is defined, which represents the pressure difference known from the system setup. In this way, the output equation can be compared 
to the data measured in the setup and proceed to estimate the unknown parameters. 

From the system setup $8$ different relative pressures can be measured, following \figref{systemdiagram} notation the sensors are placed in: 
$n_2$ $n_4$ $n_5$ $n_7$ $n_{10}$ $n_{11}$ $n_{15}$ $n_{16}$.

In order to compare the measurements from the system setup and the data obtained from the simulation in Matlab, a reference point has to be set in the 
simulation to calculate the desired data. 

The atmospheric is set as reference point and the pressures obtained from the simulation are dependant on it. The relationship between pressures, where DpCXX describes
the pressure difference for the XX component, can be defined as:

\text{\underline{Node 2}} 
\vspace{4mm}
\begin{equation}
    DpC2 = y_1
\end{equation}

\text{\underline{Node 7}}
\vspace{4mm}
\begin{equation}
  DpC16 = y_2
\end{equation}

\text{\underline{Node 4}}
\vspace{4mm}
\begin {equation}
    DpC18 + DpC19 + DpC23 + DpC24 = y_3
\end{equation}

\text{\underline{Node 5}}
\vspace{4mm}
\begin {equation}
    DpC25 + DpC26 + DpC30 + DpC31 = y_4
\end{equation}

\text{\underline{Node 10}}
\vspace{4mm}
\begin {equation}
    DpC24 = y_5
\end{equation}

\text{\underline{Node 11}}
\vspace{4mm}
\begin {equation}
    DpC20 + DpC21= y_6
\end{equation}

\text{\underline{Node 15}}
\vspace{4mm}
\begin {equation}
    DpC31 = y_7
\end{equation}

\text{\underline{Node 16}}
\vspace{4mm}
\begin {equation}
    DpC28 + DpC27 = y_8
\end{equation}

With the $8$ equations depicted above the output vector $y$ is obtained. 


