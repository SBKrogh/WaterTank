\chapter{Nonlinear Estimation}
\label{NonLinResults}

\textbf{Estimation data}
In order to estimate accurately the unknown parameters of the system, adequate inputs signals have to be applied to the model. In this way, the system 
will work on different scenarios regarding the different combination of inputs signals. 

\figref{systemdiagram} shows the different combination between the OD of the PMA valves, and the steps applied to the main and PMA pumps. 

\begin{figure}[H]
\centering
\resizebox{0.75\linewidth}{!}{% This file was created by matlab2tikz.
%
%The latest updates can be retrieved from
%  http://www.mathworks.com/matlabcentral/fileexchange/22022-matlab2tikz-matlab2tikz
%where you can also make suggestions and rate matlab2tikz.
%
\definecolor{mycolor1}{rgb}{0.00000,0.44700,0.74100}%
\definecolor{mycolor2}{rgb}{0.85000,0.32500,0.09800}%
\definecolor{mycolor3}{rgb}{0.92900,0.69400,0.12500}%
\definecolor{mycolor4}{rgb}{0.49400,0.18400,0.55600}%
%
\begin{tikzpicture}

\begin{axis}[%
width=2.0556in,
height=1.62135in,
at={(1.011in,0.642in)},
scale only axis,
xmin=0,
xmax=160,
xlabel={Seconds},
xmajorgrids,
ymin=0.4,
ymax=1.1,
ylabel={Degrees},
ymajorgrids,
axis background/.style={fill=white},
title style={font=\bfseries},
title={Valve opening degree},
legend style={legend cell align=left,align=left,draw=white!15!black}
]
\addplot [color=mycolor1,solid]
  table[row sep=crcr]{%
0	0.4\\
0.4	0.4\\
0.75	0.4\\
1.15	0.4\\
1.5	0.4\\
1.85	0.4\\
2.25	0.4\\
2.6	0.4\\
2.95	0.4\\
3.35	0.4\\
3.7	0.4\\
4.1	0.4\\
4.45	0.4\\
4.8	0.4\\
5.2	0.416666666666667\\
5.55	0.455555555555556\\
5.9	0.494444444444444\\
5.95	0.5\\
6.3	0.5\\
6.65	0.5\\
7.05	0.5\\
7.4	0.5\\
7.75	0.5\\
8.15	0.5\\
8.5	0.5\\
8.85	0.5\\
9.25	0.5\\
9.6	0.5\\
10	0.5\\
10.35	0.5\\
10.7	0.5\\
11.1	0.5\\
11.45	0.5\\
11.8	0.5\\
12.2	0.5\\
12.55	0.5\\
12.9	0.5\\
13.3	0.5\\
13.65	0.5\\
14.05	0.5\\
14.4	0.5\\
14.75	0.5\\
15.15	0.5\\
15.5	0.5\\
15.85	0.5\\
16.25	0.5\\
16.6	0.5\\
17	0.5\\
17.35	0.5\\
17.7	0.5\\
18.1	0.5\\
18.45	0.5\\
18.8	0.5\\
19.2	0.5\\
19.55	0.5\\
19.95	0.5\\
20.3	0.527777777777778\\
20.65	0.566666666666667\\
20.95	0.6\\
21.05	0.6\\
21.4	0.6\\
21.75	0.6\\
22.15	0.6\\
22.5	0.6\\
22.9	0.6\\
23.25	0.6\\
23.6	0.6\\
24	0.6\\
24.35	0.6\\
24.7	0.6\\
25.1	0.6\\
25.45	0.6\\
25.8	0.6\\
26.2	0.6\\
26.55	0.6\\
26.95	0.6\\
27.3	0.6\\
27.65	0.6\\
28.05	0.6\\
28.4	0.6\\
28.75	0.6\\
29.15	0.6\\
29.5	0.6\\
29.9	0.6\\
30.25	0.6\\
30.6	0.6\\
31	0.6\\
31.35	0.6\\
31.7	0.6\\
32.1	0.6\\
32.45	0.6\\
32.85	0.6\\
33.2	0.6\\
33.55	0.6\\
33.95	0.6\\
34.3	0.6\\
34.65	0.6\\
35.05	0.6\\
35.4	0.638888888888889\\
35.75	0.677777777777778\\
36.15	0.722222222222222\\
36.5	0.761111111111111\\
36.85	0.8\\
36.9	0.8\\
37.25	0.8\\
37.6	0.8\\
38	0.8\\
38.35	0.8\\
38.7	0.8\\
39.1	0.8\\
39.45	0.8\\
39.85	0.8\\
40.2	0.8\\
40.55	0.8\\
40.95	0.8\\
41.3	0.8\\
41.65	0.8\\
42.05	0.8\\
42.4	0.8\\
42.8	0.8\\
43.15	0.8\\
43.5	0.8\\
43.9	0.8\\
44.25	0.8\\
44.6	0.8\\
45	0.8\\
45.35	0.8\\
45.75	0.8\\
46.1	0.8\\
46.45	0.8\\
46.85	0.8\\
47.2	0.8\\
47.55	0.8\\
47.95	0.8\\
48.3	0.8\\
48.65	0.8\\
49.05	0.8\\
49.4	0.8\\
49.8	0.8\\
50.15	0.811111111111111\\
50.5	0.85\\
50.9	0.894444444444444\\
51.25	0.933333333333333\\
51.6	0.972222222222222\\
51.85	1\\
52	1\\
52.35	1\\
52.75	1\\
53.1	1\\
53.45	1\\
53.85	1\\
54.2	1\\
54.55	1\\
54.95	1\\
55.3	1\\
55.7	1\\
56.05	1\\
56.4	1\\
56.8	1\\
57.15	1\\
57.5	1\\
57.9	1\\
58.25	1\\
58.6	1\\
59	1\\
59.35	1\\
59.75	1\\
60.1	1\\
60.45	1\\
60.85	1\\
61.2	1\\
61.55	1\\
61.95	1\\
62.3	1\\
62.7	1\\
63.05	1\\
63.4	1\\
63.8	1\\
64.15	1\\
64.5	1\\
64.9	1\\
65.25	0.977777777777778\\
65.65	0.933333333333333\\
65.95	0.9\\
66	0.9\\
66.35	0.9\\
66.75	0.9\\
67.1	0.9\\
67.45	0.9\\
67.85	0.9\\
68.2	0.9\\
68.6	0.9\\
68.95	0.9\\
69.3	0.9\\
69.7	0.9\\
70.05	0.9\\
70.4	0.9\\
70.8	0.9\\
71.15	0.9\\
71.5	0.9\\
71.9	0.9\\
72.25	0.9\\
72.65	0.9\\
73	0.9\\
73.35	0.9\\
73.75	0.9\\
74.1	0.9\\
74.45	0.9\\
74.85	0.9\\
75.2	0.9\\
75.6	0.9\\
75.95	0.9\\
76.3	0.9\\
76.7	0.9\\
77.05	0.9\\
77.4	0.9\\
77.8	0.9\\
78.15	0.9\\
78.55	0.9\\
78.9	0.9\\
79.25	0.9\\
79.65	0.9\\
80	0.9\\
80.35	0.9\\
80.75	0.9\\
81.1	0.9\\
81.45	0.9\\
81.85	0.9\\
82.2	0.9\\
82.6	0.9\\
82.95	0.9\\
83.3	0.9\\
83.7	0.9\\
84.05	0.9\\
84.4	0.9\\
84.8	0.9\\
85.15	0.9\\
85.55	0.9\\
85.9	0.9\\
86.25	0.9\\
86.65	0.9\\
87	0.9\\
87.35	0.9\\
87.75	0.9\\
88.1	0.9\\
88.5	0.9\\
88.85	0.9\\
89.2	0.9\\
89.6	0.9\\
89.95	0.9\\
90.3	0.9\\
90.7	0.9\\
91.05	0.9\\
91.4	0.9\\
91.8	0.9\\
92.15	0.9\\
92.55	0.9\\
92.9	0.9\\
93.25	0.9\\
93.65	0.9\\
94	0.9\\
94.35	0.9\\
94.75	0.9\\
95.1	0.894444444444444\\
95.5	0.85\\
95.85	0.811111111111111\\
96.2	0.772222222222222\\
96.6	0.727777777777778\\
96.85	0.7\\
96.95	0.7\\
97.3	0.7\\
97.7	0.7\\
98.05	0.7\\
98.45	0.7\\
98.8	0.7\\
99.15	0.7\\
99.55	0.7\\
99.9	0.7\\
100.25	0.7\\
100.65	0.7\\
101	0.7\\
101.4	0.7\\
101.75	0.7\\
102.1	0.7\\
102.5	0.7\\
102.85	0.7\\
103.2	0.7\\
103.6	0.7\\
103.95	0.7\\
104.3	0.7\\
104.7	0.7\\
105.05	0.7\\
105.45	0.7\\
105.8	0.7\\
106.15	0.7\\
106.55	0.7\\
106.9	0.7\\
107.25	0.7\\
107.65	0.7\\
108	0.7\\
108.4	0.7\\
108.75	0.7\\
109.1	0.7\\
109.5	0.7\\
109.85	0.7\\
110.2	0.683333333333333\\
110.6	0.638888888888889\\
110.95	0.6\\
111.35	0.555555555555555\\
111.7	0.516666666666667\\
111.85	0.5\\
112.05	0.5\\
112.45	0.5\\
112.8	0.5\\
113.15	0.5\\
113.55	0.5\\
113.9	0.5\\
114.25	0.5\\
114.65	0.5\\
115	0.5\\
115.4	0.5\\
115.75	0.5\\
116.1	0.5\\
116.5	0.5\\
116.85	0.5\\
117.2	0.5\\
117.6	0.5\\
117.95	0.5\\
118.35	0.5\\
118.7	0.5\\
119.05	0.5\\
119.45	0.5\\
119.8	0.5\\
120.15	0.5\\
120.55	0.5\\
120.9	0.5\\
121.3	0.5\\
121.65	0.5\\
122	0.5\\
122.4	0.5\\
122.75	0.5\\
123.1	0.5\\
123.5	0.5\\
123.85	0.5\\
124.25	0.5\\
124.6	0.5\\
124.95	0.5\\
125.35	0.533333333333333\\
125.7	0.572222222222222\\
125.95	0.6\\
126.05	0.6\\
126.45	0.6\\
126.8	0.6\\
127.15	0.6\\
127.55	0.6\\
127.9	0.6\\
128.3	0.6\\
128.65	0.6\\
129	0.6\\
129.4	0.6\\
129.75	0.6\\
130.1	0.6\\
130.5	0.6\\
130.85	0.6\\
131.25	0.6\\
131.6	0.6\\
131.95	0.6\\
132.35	0.6\\
132.7	0.6\\
133.05	0.6\\
133.45	0.6\\
133.8	0.6\\
134.2	0.6\\
134.55	0.6\\
134.9	0.6\\
135.3	0.6\\
135.65	0.6\\
136	0.6\\
136.4	0.6\\
136.75	0.6\\
137.1	0.6\\
137.5	0.6\\
137.85	0.6\\
138.25	0.6\\
138.6	0.6\\
138.95	0.6\\
139.35	0.6\\
139.7	0.6\\
140.05	0.6\\
140.45	0.555555555555556\\
140.8	0.516666666666667\\
141.2	0.472222222222222\\
141.55	0.433333333333333\\
141.85	0.4\\
141.9	0.4\\
142.3	0.4\\
142.65	0.4\\
143	0.4\\
143.4	0.4\\
143.75	0.4\\
144.15	0.4\\
144.5	0.4\\
144.85	0.4\\
145.25	0.4\\
145.6	0.4\\
145.95	0.4\\
146.35	0.4\\
146.7	0.4\\
147.1	0.4\\
147.45	0.4\\
147.8	0.4\\
148.2	0.4\\
148.55	0.4\\
148.9	0.4\\
149.3	0.4\\
149.65	0.4\\
150	0.4\\
150.4	0.4\\
150.75	0.4\\
151.15	0.4\\
151.5	0.4\\
151.85	0.4\\
152.25	0.4\\
152.6	0.4\\
152.95	0.4\\
153.35	0.4\\
153.7	0.4\\
154.1	0.4\\
154.45	0.4\\
154.8	0.4\\
155.2	0.4\\
155.55	0.4\\
155.9	0.4\\
156.3	0.4\\
156.65	0.4\\
157.05	0.4\\
157.4	0.4\\
157.75	0.4\\
158.15	0.4\\
158.5	0.4\\
158.85	0.4\\
159.25	0.4\\
159.95	0.4\\
};
\addlegendentry{C20};

\addplot [color=mycolor2,solid]
  table[row sep=crcr]{%
0	0.4\\
0.4	0.4\\
0.75	0.4\\
1.15	0.4\\
1.5	0.4\\
1.85	0.4\\
2.25	0.4\\
2.6	0.4\\
2.95	0.4\\
3.35	0.4\\
3.7	0.4\\
4.1	0.4\\
4.45	0.4\\
4.8	0.4\\
5.2	0.416666666666667\\
5.55	0.455555555555556\\
5.9	0.494444444444444\\
6.3	0.538888888888889\\
6.65	0.577777777777778\\
7.05	0.622222222222222\\
7.4	0.661111111111111\\
7.75	0.7\\
8.15	0.744444444444444\\
8.5	0.783333333333333\\
8.65	0.8\\
8.85	0.8\\
9.25	0.8\\
9.6	0.8\\
10	0.8\\
10.35	0.8\\
10.7	0.8\\
11.1	0.8\\
11.45	0.8\\
11.8	0.8\\
12.2	0.8\\
12.55	0.8\\
12.9	0.8\\
13.3	0.8\\
13.65	0.8\\
14.05	0.8\\
14.4	0.8\\
14.75	0.8\\
15.15	0.8\\
15.5	0.8\\
15.85	0.8\\
16.25	0.8\\
16.6	0.8\\
17	0.8\\
17.35	0.8\\
17.7	0.8\\
18.1	0.8\\
18.45	0.8\\
18.8	0.8\\
19.2	0.8\\
19.55	0.8\\
19.95	0.8\\
20.3	0.827777777777778\\
20.65	0.866666666666667\\
21.05	0.911111111111111\\
21.4	0.95\\
21.75	0.988888888888889\\
21.85	1\\
22.15	1\\
22.5	1\\
22.9	1\\
23.25	1\\
23.6	1\\
24	1\\
24.35	1\\
24.7	1\\
25.1	1\\
25.45	1\\
25.8	1\\
26.2	1\\
26.55	1\\
26.95	1\\
27.3	1\\
27.65	1\\
28.05	1\\
28.4	1\\
28.75	1\\
29.15	1\\
29.5	1\\
29.9	1\\
30.25	1\\
30.6	1\\
31	1\\
31.35	1\\
31.7	1\\
32.1	1\\
32.45	1\\
32.85	1\\
33.2	1\\
33.55	1\\
33.95	1\\
34.3	1\\
34.65	1\\
35.05	1\\
35.4	0.961111111111111\\
35.75	0.922222222222222\\
35.95	0.9\\
36.15	0.9\\
36.5	0.9\\
36.9	0.9\\
37.25	0.9\\
37.6	0.9\\
38	0.9\\
38.35	0.9\\
38.7	0.9\\
39.1	0.9\\
39.45	0.9\\
39.85	0.9\\
40.2	0.9\\
40.55	0.9\\
40.95	0.9\\
41.3	0.9\\
41.65	0.9\\
42.05	0.9\\
42.4	0.9\\
42.8	0.9\\
43.15	0.9\\
43.5	0.9\\
43.9	0.9\\
44.25	0.9\\
44.6	0.9\\
45	0.9\\
45.35	0.9\\
45.75	0.9\\
46.1	0.9\\
46.45	0.9\\
46.85	0.9\\
47.2	0.9\\
47.55	0.9\\
47.95	0.9\\
48.3	0.9\\
48.65	0.9\\
49.05	0.9\\
49.4	0.9\\
49.8	0.9\\
50.15	0.9\\
50.5	0.9\\
50.9	0.9\\
51.25	0.9\\
51.6	0.9\\
52	0.9\\
52.35	0.9\\
52.75	0.9\\
53.1	0.9\\
53.45	0.9\\
53.85	0.9\\
54.2	0.9\\
54.55	0.9\\
54.95	0.9\\
55.3	0.9\\
55.7	0.9\\
56.05	0.9\\
56.4	0.9\\
56.8	0.9\\
57.15	0.9\\
57.5	0.9\\
57.9	0.9\\
58.25	0.9\\
58.6	0.9\\
59	0.9\\
59.35	0.9\\
59.75	0.9\\
60.1	0.9\\
60.45	0.9\\
60.85	0.9\\
61.2	0.9\\
61.55	0.9\\
61.95	0.9\\
62.3	0.9\\
62.7	0.9\\
63.05	0.9\\
63.4	0.9\\
63.8	0.9\\
64.15	0.9\\
64.5	0.9\\
64.9	0.9\\
65.25	0.877777777777778\\
65.65	0.833333333333333\\
66	0.794444444444445\\
66.35	0.755555555555556\\
66.75	0.711111111111111\\
66.85	0.7\\
67.1	0.7\\
67.45	0.7\\
67.85	0.7\\
68.2	0.7\\
68.6	0.7\\
68.95	0.7\\
69.3	0.7\\
69.7	0.7\\
70.05	0.7\\
70.4	0.7\\
70.8	0.7\\
71.15	0.7\\
71.5	0.7\\
71.9	0.7\\
72.25	0.7\\
72.65	0.7\\
73	0.7\\
73.35	0.7\\
73.75	0.7\\
74.1	0.7\\
74.45	0.7\\
74.85	0.7\\
75.2	0.7\\
75.6	0.7\\
75.95	0.7\\
76.3	0.7\\
76.7	0.7\\
77.05	0.7\\
77.4	0.7\\
77.8	0.7\\
78.15	0.7\\
78.55	0.7\\
78.9	0.7\\
79.25	0.7\\
79.65	0.7\\
80	0.7\\
80.35	0.666666666666667\\
80.75	0.622222222222222\\
81.1	0.583333333333333\\
81.45	0.544444444444444\\
81.85	0.5\\
81.9	0.5\\
82.2	0.5\\
82.6	0.5\\
82.95	0.5\\
83.3	0.5\\
83.7	0.5\\
84.05	0.5\\
84.4	0.5\\
84.8	0.5\\
85.15	0.5\\
85.55	0.5\\
85.9	0.5\\
86.25	0.5\\
86.65	0.5\\
87	0.5\\
87.35	0.5\\
87.75	0.5\\
88.1	0.5\\
88.5	0.5\\
88.85	0.5\\
89.2	0.5\\
89.6	0.5\\
89.95	0.5\\
90.3	0.5\\
90.7	0.5\\
91.05	0.5\\
91.4	0.5\\
91.8	0.5\\
92.15	0.5\\
92.55	0.5\\
92.9	0.5\\
93.25	0.5\\
93.65	0.5\\
94	0.5\\
94.35	0.5\\
94.75	0.5\\
95.1	0.505555555555556\\
95.5	0.55\\
95.85	0.588888888888889\\
95.95	0.6\\
96.2	0.6\\
96.6	0.6\\
96.95	0.6\\
97.3	0.6\\
97.7	0.6\\
98.05	0.6\\
98.45	0.6\\
98.8	0.6\\
99.15	0.6\\
99.55	0.6\\
99.9	0.6\\
100.25	0.6\\
100.65	0.6\\
101	0.6\\
101.4	0.6\\
101.75	0.6\\
102.1	0.6\\
102.5	0.6\\
102.85	0.6\\
103.2	0.6\\
103.6	0.6\\
103.95	0.6\\
104.3	0.6\\
104.7	0.6\\
105.05	0.6\\
105.45	0.6\\
105.8	0.6\\
106.15	0.6\\
106.55	0.6\\
106.9	0.6\\
107.25	0.6\\
107.65	0.6\\
108	0.6\\
108.4	0.6\\
108.75	0.6\\
109.1	0.6\\
109.5	0.6\\
109.85	0.6\\
110.2	0.583333333333333\\
110.6	0.538888888888889\\
110.95	0.5\\
111.35	0.5\\
111.7	0.5\\
112.05	0.5\\
112.45	0.5\\
112.8	0.5\\
113.15	0.5\\
113.55	0.5\\
113.9	0.5\\
114.25	0.5\\
114.65	0.5\\
115	0.5\\
115.4	0.5\\
115.75	0.5\\
116.1	0.5\\
116.5	0.5\\
116.85	0.5\\
117.2	0.5\\
117.6	0.5\\
117.95	0.5\\
118.35	0.5\\
118.7	0.5\\
119.05	0.5\\
119.45	0.5\\
119.8	0.5\\
120.15	0.5\\
120.55	0.5\\
120.9	0.5\\
121.3	0.5\\
121.65	0.5\\
122	0.5\\
122.4	0.5\\
122.75	0.5\\
123.1	0.5\\
123.5	0.5\\
123.85	0.5\\
124.25	0.5\\
124.6	0.5\\
124.95	0.5\\
125.35	0.533333333333333\\
125.7	0.572222222222222\\
125.95	0.6\\
126.05	0.6\\
126.45	0.6\\
126.8	0.6\\
127.15	0.6\\
127.55	0.6\\
127.9	0.6\\
128.3	0.6\\
128.65	0.6\\
129	0.6\\
129.4	0.6\\
129.75	0.6\\
130.1	0.6\\
130.5	0.6\\
130.85	0.6\\
131.25	0.6\\
131.6	0.6\\
131.95	0.6\\
132.35	0.6\\
132.7	0.6\\
133.05	0.6\\
133.45	0.6\\
133.8	0.6\\
134.2	0.6\\
134.55	0.6\\
134.9	0.6\\
135.3	0.6\\
135.65	0.6\\
136	0.6\\
136.4	0.6\\
136.75	0.6\\
137.1	0.6\\
137.5	0.6\\
137.85	0.6\\
138.25	0.6\\
138.6	0.6\\
138.95	0.6\\
139.35	0.6\\
139.7	0.6\\
140.05	0.6\\
140.45	0.555555555555556\\
140.8	0.516666666666667\\
141.2	0.472222222222222\\
141.55	0.433333333333333\\
141.85	0.4\\
141.9	0.4\\
142.3	0.4\\
142.65	0.4\\
143	0.4\\
143.4	0.4\\
143.75	0.4\\
144.15	0.4\\
144.5	0.4\\
144.85	0.4\\
145.25	0.4\\
145.6	0.4\\
145.95	0.4\\
146.35	0.4\\
146.7	0.4\\
147.1	0.4\\
147.45	0.4\\
147.8	0.4\\
148.2	0.4\\
148.55	0.4\\
148.9	0.4\\
149.3	0.4\\
149.65	0.4\\
150	0.4\\
150.4	0.4\\
150.75	0.4\\
151.15	0.4\\
151.5	0.4\\
151.85	0.4\\
152.25	0.4\\
152.6	0.4\\
152.95	0.4\\
153.35	0.4\\
153.7	0.4\\
154.1	0.4\\
154.45	0.4\\
154.8	0.4\\
155.2	0.4\\
155.55	0.4\\
155.9	0.4\\
156.3	0.4\\
156.65	0.4\\
157.05	0.4\\
157.4	0.4\\
157.75	0.4\\
158.15	0.4\\
158.5	0.4\\
158.85	0.4\\
159.25	0.4\\
159.95	0.4\\
};
\addlegendentry{C24};

\addplot [color=mycolor3,solid]
  table[row sep=crcr]{%
0	0.4\\
0.4	0.4\\
0.75	0.4\\
1.15	0.4\\
1.5	0.4\\
1.85	0.4\\
2.25	0.4\\
2.6	0.4\\
2.95	0.4\\
3.35	0.4\\
3.7	0.4\\
4.1	0.4\\
4.45	0.4\\
4.8	0.4\\
5.2	0.416666666666667\\
5.55	0.455555555555556\\
5.9	0.494444444444444\\
6.3	0.538888888888889\\
6.65	0.577777777777778\\
7.05	0.622222222222222\\
7.4	0.661111111111111\\
7.75	0.7\\
8.15	0.744444444444444\\
8.5	0.783333333333333\\
8.85	0.822222222222222\\
9.25	0.866666666666667\\
9.55	0.9\\
9.6	0.9\\
10	0.9\\
10.35	0.9\\
10.7	0.9\\
11.1	0.9\\
11.45	0.9\\
11.8	0.9\\
12.2	0.9\\
12.55	0.9\\
12.9	0.9\\
13.3	0.9\\
13.65	0.9\\
14.05	0.9\\
14.4	0.9\\
14.75	0.9\\
15.15	0.9\\
15.5	0.9\\
15.85	0.9\\
16.25	0.9\\
16.6	0.9\\
17	0.9\\
17.35	0.9\\
17.7	0.9\\
18.1	0.9\\
18.45	0.9\\
18.8	0.9\\
19.2	0.9\\
19.55	0.9\\
19.95	0.9\\
20.3	0.872222222222222\\
20.65	0.833333333333333\\
21.05	0.788888888888889\\
21.4	0.75\\
21.75	0.711111111111111\\
21.85	0.7\\
22.15	0.7\\
22.5	0.7\\
22.9	0.7\\
23.25	0.7\\
23.6	0.7\\
24	0.7\\
24.35	0.7\\
24.7	0.7\\
25.1	0.7\\
25.45	0.7\\
25.8	0.7\\
26.2	0.7\\
26.55	0.7\\
26.95	0.7\\
27.3	0.7\\
27.65	0.7\\
28.05	0.7\\
28.4	0.7\\
28.75	0.7\\
29.15	0.7\\
29.5	0.7\\
29.9	0.7\\
30.25	0.7\\
30.6	0.7\\
31	0.7\\
31.35	0.7\\
31.7	0.7\\
32.1	0.7\\
32.45	0.7\\
32.85	0.7\\
33.2	0.7\\
33.55	0.7\\
33.95	0.7\\
34.3	0.7\\
34.65	0.7\\
35.05	0.7\\
35.4	0.661111111111111\\
35.75	0.622222222222222\\
36.15	0.577777777777778\\
36.5	0.538888888888889\\
36.85	0.5\\
36.9	0.5\\
37.25	0.5\\
37.6	0.5\\
38	0.5\\
38.35	0.5\\
38.7	0.5\\
39.1	0.5\\
39.45	0.5\\
39.85	0.5\\
40.2	0.5\\
40.55	0.5\\
40.95	0.5\\
41.3	0.5\\
41.65	0.5\\
42.05	0.5\\
42.4	0.5\\
42.8	0.5\\
43.15	0.5\\
43.5	0.5\\
43.9	0.5\\
44.25	0.5\\
44.6	0.5\\
45	0.5\\
45.35	0.5\\
45.75	0.5\\
46.1	0.5\\
46.45	0.5\\
46.85	0.5\\
47.2	0.5\\
47.55	0.5\\
47.95	0.5\\
48.3	0.5\\
48.65	0.5\\
49.05	0.5\\
49.4	0.5\\
49.8	0.5\\
50.15	0.511111111111111\\
50.5	0.55\\
50.9	0.594444444444444\\
50.95	0.6\\
51.25	0.6\\
51.6	0.6\\
52	0.6\\
52.35	0.6\\
52.75	0.6\\
53.1	0.6\\
53.45	0.6\\
53.85	0.6\\
54.2	0.6\\
54.55	0.6\\
54.95	0.6\\
55.3	0.6\\
55.7	0.6\\
56.05	0.6\\
56.4	0.6\\
56.8	0.6\\
57.15	0.6\\
57.5	0.6\\
57.9	0.6\\
58.25	0.6\\
58.6	0.6\\
59	0.6\\
59.35	0.6\\
59.75	0.6\\
60.1	0.6\\
60.45	0.6\\
60.85	0.6\\
61.2	0.6\\
61.55	0.6\\
61.95	0.6\\
62.3	0.6\\
62.7	0.6\\
63.05	0.6\\
63.4	0.6\\
63.8	0.6\\
64.15	0.6\\
64.5	0.6\\
64.9	0.6\\
65.25	0.577777777777778\\
65.65	0.533333333333333\\
65.95	0.5\\
66	0.5\\
66.35	0.5\\
66.75	0.5\\
67.1	0.5\\
67.45	0.5\\
67.85	0.5\\
68.2	0.5\\
68.6	0.5\\
68.95	0.5\\
69.3	0.5\\
69.7	0.5\\
70.05	0.5\\
70.4	0.5\\
70.8	0.5\\
71.15	0.5\\
71.5	0.5\\
71.9	0.5\\
72.25	0.5\\
72.65	0.5\\
73	0.5\\
73.35	0.5\\
73.75	0.5\\
74.1	0.5\\
74.45	0.5\\
74.85	0.5\\
75.2	0.5\\
75.6	0.5\\
75.95	0.5\\
76.3	0.5\\
76.7	0.5\\
77.05	0.5\\
77.4	0.5\\
77.8	0.5\\
78.15	0.5\\
78.55	0.5\\
78.9	0.5\\
79.25	0.5\\
79.65	0.5\\
80	0.5\\
80.35	0.533333333333333\\
80.75	0.577777777777778\\
80.95	0.6\\
81.1	0.6\\
81.45	0.6\\
81.85	0.6\\
82.2	0.6\\
82.6	0.6\\
82.95	0.6\\
83.3	0.6\\
83.7	0.6\\
84.05	0.6\\
84.4	0.6\\
84.8	0.6\\
85.15	0.6\\
85.55	0.6\\
85.9	0.6\\
86.25	0.6\\
86.65	0.6\\
87	0.6\\
87.35	0.6\\
87.75	0.6\\
88.1	0.6\\
88.5	0.6\\
88.85	0.6\\
89.2	0.6\\
89.6	0.6\\
89.95	0.6\\
90.3	0.6\\
90.7	0.6\\
91.05	0.6\\
91.4	0.6\\
91.8	0.6\\
92.15	0.6\\
92.55	0.6\\
92.9	0.6\\
93.25	0.6\\
93.65	0.6\\
94	0.6\\
94.35	0.6\\
94.75	0.6\\
95.1	0.605555555555556\\
95.5	0.65\\
95.85	0.688888888888889\\
96.2	0.727777777777778\\
96.6	0.772222222222222\\
96.85	0.8\\
96.95	0.8\\
97.3	0.8\\
97.7	0.8\\
98.05	0.8\\
98.45	0.8\\
98.8	0.8\\
99.15	0.8\\
99.55	0.8\\
99.9	0.8\\
100.25	0.8\\
100.65	0.8\\
101	0.8\\
101.4	0.8\\
101.75	0.8\\
102.1	0.8\\
102.5	0.8\\
102.85	0.8\\
103.2	0.8\\
103.6	0.8\\
103.95	0.8\\
104.3	0.8\\
104.7	0.8\\
105.05	0.8\\
105.45	0.8\\
105.8	0.8\\
106.15	0.8\\
106.55	0.8\\
106.9	0.8\\
107.25	0.8\\
107.65	0.8\\
108	0.8\\
108.4	0.8\\
108.75	0.8\\
109.1	0.8\\
109.5	0.8\\
109.85	0.8\\
110.2	0.816666666666667\\
110.6	0.861111111111111\\
110.95	0.9\\
111.35	0.944444444444445\\
111.7	0.983333333333333\\
111.85	1\\
112.05	1\\
112.45	1\\
112.8	1\\
113.15	1\\
113.55	1\\
113.9	1\\
114.25	1\\
114.65	1\\
115	1\\
115.4	1\\
115.75	1\\
116.1	1\\
116.5	1\\
116.85	1\\
117.2	1\\
117.6	1\\
117.95	1\\
118.35	1\\
118.7	1\\
119.05	1\\
119.45	1\\
119.8	1\\
120.15	1\\
120.55	1\\
120.9	1\\
121.3	1\\
121.65	1\\
122	1\\
122.4	1\\
122.75	1\\
123.1	1\\
123.5	1\\
123.85	1\\
124.25	1\\
124.6	1\\
124.95	1\\
125.35	0.966666666666667\\
125.7	0.927777777777778\\
125.95	0.9\\
126.05	0.9\\
126.45	0.9\\
126.8	0.9\\
127.15	0.9\\
127.55	0.9\\
127.9	0.9\\
128.3	0.9\\
128.65	0.9\\
129	0.9\\
129.4	0.9\\
129.75	0.9\\
130.1	0.9\\
130.5	0.9\\
130.85	0.9\\
131.25	0.9\\
131.6	0.9\\
131.95	0.9\\
132.35	0.9\\
132.7	0.9\\
133.05	0.9\\
133.45	0.9\\
133.8	0.9\\
134.2	0.9\\
134.55	0.9\\
134.9	0.9\\
135.3	0.9\\
135.65	0.9\\
136	0.9\\
136.4	0.9\\
136.75	0.9\\
137.1	0.9\\
137.5	0.9\\
137.85	0.9\\
138.25	0.9\\
138.6	0.9\\
138.95	0.9\\
139.35	0.9\\
139.7	0.9\\
140.05	0.9\\
140.45	0.855555555555556\\
140.8	0.816666666666667\\
141.2	0.772222222222222\\
141.55	0.733333333333333\\
141.9	0.694444444444444\\
142.3	0.65\\
142.65	0.611111111111111\\
143	0.572222222222222\\
143.4	0.527777777777778\\
143.75	0.488888888888889\\
144.15	0.444444444444444\\
144.5	0.405555555555556\\
144.55	0.4\\
144.85	0.4\\
145.25	0.4\\
145.6	0.4\\
145.95	0.4\\
146.35	0.4\\
146.7	0.4\\
147.1	0.4\\
147.45	0.4\\
147.8	0.4\\
148.2	0.4\\
148.55	0.4\\
148.9	0.4\\
149.3	0.4\\
149.65	0.4\\
150	0.4\\
150.4	0.4\\
150.75	0.4\\
151.15	0.4\\
151.5	0.4\\
151.85	0.4\\
152.25	0.4\\
152.6	0.4\\
152.95	0.4\\
153.35	0.4\\
153.7	0.4\\
154.1	0.4\\
154.45	0.4\\
154.8	0.4\\
155.2	0.4\\
155.55	0.4\\
155.9	0.4\\
156.3	0.4\\
156.65	0.4\\
157.05	0.4\\
157.4	0.4\\
157.75	0.4\\
158.15	0.4\\
158.5	0.4\\
158.85	0.4\\
159.25	0.4\\
159.95	0.4\\
};
\addlegendentry{C27};

\addplot [color=mycolor4,solid]
  table[row sep=crcr]{%
0	0.4\\
0.4	0.4\\
0.75	0.4\\
1.15	0.4\\
1.5	0.4\\
1.85	0.4\\
2.25	0.4\\
2.6	0.4\\
2.95	0.4\\
3.35	0.4\\
3.7	0.4\\
4.1	0.4\\
4.45	0.4\\
4.8	0.4\\
5.2	0.416666666666667\\
5.55	0.455555555555556\\
5.9	0.494444444444444\\
5.95	0.5\\
6.3	0.5\\
6.65	0.5\\
7.05	0.5\\
7.4	0.5\\
7.75	0.5\\
8.15	0.5\\
8.5	0.5\\
8.85	0.5\\
9.25	0.5\\
9.6	0.5\\
10	0.5\\
10.35	0.5\\
10.7	0.5\\
11.1	0.5\\
11.45	0.5\\
11.8	0.5\\
12.2	0.5\\
12.55	0.5\\
12.9	0.5\\
13.3	0.5\\
13.65	0.5\\
14.05	0.5\\
14.4	0.5\\
14.75	0.5\\
15.15	0.5\\
15.5	0.5\\
15.85	0.5\\
16.25	0.5\\
16.6	0.5\\
17	0.5\\
17.35	0.5\\
17.7	0.5\\
18.1	0.5\\
18.45	0.5\\
18.8	0.5\\
19.2	0.5\\
19.55	0.5\\
19.95	0.5\\
20.3	0.527777777777778\\
20.65	0.566666666666667\\
20.95	0.6\\
21.05	0.6\\
21.4	0.6\\
21.75	0.6\\
22.15	0.6\\
22.5	0.6\\
22.9	0.6\\
23.25	0.6\\
23.6	0.6\\
24	0.6\\
24.35	0.6\\
24.7	0.6\\
25.1	0.6\\
25.45	0.6\\
25.8	0.6\\
26.2	0.6\\
26.55	0.6\\
26.95	0.6\\
27.3	0.6\\
27.65	0.6\\
28.05	0.6\\
28.4	0.6\\
28.75	0.6\\
29.15	0.6\\
29.5	0.6\\
29.9	0.6\\
30.25	0.6\\
30.6	0.6\\
31	0.6\\
31.35	0.6\\
31.7	0.6\\
32.1	0.6\\
32.45	0.6\\
32.85	0.6\\
33.2	0.6\\
33.55	0.6\\
33.95	0.6\\
34.3	0.6\\
34.65	0.6\\
35.05	0.6\\
35.4	0.561111111111111\\
35.75	0.522222222222222\\
35.95	0.5\\
36.15	0.5\\
36.5	0.5\\
36.9	0.5\\
37.25	0.5\\
37.6	0.5\\
38	0.5\\
38.35	0.5\\
38.7	0.5\\
39.1	0.5\\
39.45	0.5\\
39.85	0.5\\
40.2	0.5\\
40.55	0.5\\
40.95	0.5\\
41.3	0.5\\
41.65	0.5\\
42.05	0.5\\
42.4	0.5\\
42.8	0.5\\
43.15	0.5\\
43.5	0.5\\
43.9	0.5\\
44.25	0.5\\
44.6	0.5\\
45	0.5\\
45.35	0.5\\
45.75	0.5\\
46.1	0.5\\
46.45	0.5\\
46.85	0.5\\
47.2	0.5\\
47.55	0.5\\
47.95	0.5\\
48.3	0.5\\
48.65	0.5\\
49.05	0.5\\
49.4	0.5\\
49.8	0.5\\
50.15	0.511111111111111\\
50.5	0.55\\
50.9	0.594444444444444\\
50.95	0.6\\
51.25	0.6\\
51.6	0.6\\
52	0.6\\
52.35	0.6\\
52.75	0.6\\
53.1	0.6\\
53.45	0.6\\
53.85	0.6\\
54.2	0.6\\
54.55	0.6\\
54.95	0.6\\
55.3	0.6\\
55.7	0.6\\
56.05	0.6\\
56.4	0.6\\
56.8	0.6\\
57.15	0.6\\
57.5	0.6\\
57.9	0.6\\
58.25	0.6\\
58.6	0.6\\
59	0.6\\
59.35	0.6\\
59.75	0.6\\
60.1	0.6\\
60.45	0.6\\
60.85	0.6\\
61.2	0.6\\
61.55	0.6\\
61.95	0.6\\
62.3	0.6\\
62.7	0.6\\
63.05	0.6\\
63.4	0.6\\
63.8	0.6\\
64.15	0.6\\
64.5	0.6\\
64.9	0.6\\
65.25	0.622222222222222\\
65.65	0.666666666666667\\
66	0.705555555555555\\
66.35	0.744444444444444\\
66.75	0.788888888888889\\
66.85	0.8\\
67.1	0.8\\
67.45	0.8\\
67.85	0.8\\
68.2	0.8\\
68.6	0.8\\
68.95	0.8\\
69.3	0.8\\
69.7	0.8\\
70.05	0.8\\
70.4	0.8\\
70.8	0.8\\
71.15	0.8\\
71.5	0.8\\
71.9	0.8\\
72.25	0.8\\
72.65	0.8\\
73	0.8\\
73.35	0.8\\
73.75	0.8\\
74.1	0.8\\
74.45	0.8\\
74.85	0.8\\
75.2	0.8\\
75.6	0.8\\
75.95	0.8\\
76.3	0.8\\
76.7	0.8\\
77.05	0.8\\
77.4	0.8\\
77.8	0.8\\
78.15	0.8\\
78.55	0.8\\
78.9	0.8\\
79.25	0.8\\
79.65	0.8\\
80	0.8\\
80.35	0.833333333333333\\
80.75	0.877777777777778\\
81.1	0.916666666666667\\
81.45	0.955555555555556\\
81.85	1\\
82.2	1\\
82.6	1\\
82.95	1\\
83.3	1\\
83.7	1\\
84.05	1\\
84.4	1\\
84.8	1\\
85.15	1\\
85.55	1\\
85.9	1\\
86.25	1\\
86.65	1\\
87	1\\
87.35	1\\
87.75	1\\
88.1	1\\
88.5	1\\
88.85	1\\
89.2	1\\
89.6	1\\
89.95	1\\
90.3	1\\
90.7	1\\
91.05	1\\
91.4	1\\
91.8	1\\
92.15	1\\
92.55	1\\
92.9	1\\
93.25	1\\
93.65	1\\
94	1\\
94.35	1\\
94.75	1\\
95.1	0.994444444444444\\
95.5	0.95\\
95.85	0.911111111111111\\
95.95	0.9\\
96.2	0.9\\
96.6	0.9\\
96.95	0.9\\
97.3	0.9\\
97.7	0.9\\
98.05	0.9\\
98.45	0.9\\
98.8	0.9\\
99.15	0.9\\
99.55	0.9\\
99.9	0.9\\
100.25	0.9\\
100.65	0.9\\
101	0.9\\
101.4	0.9\\
101.75	0.9\\
102.1	0.9\\
102.5	0.9\\
102.85	0.9\\
103.2	0.9\\
103.6	0.9\\
103.95	0.9\\
104.3	0.9\\
104.7	0.9\\
105.05	0.9\\
105.45	0.9\\
105.8	0.9\\
106.15	0.9\\
106.55	0.9\\
106.9	0.9\\
107.25	0.9\\
107.65	0.9\\
108	0.9\\
108.4	0.9\\
108.75	0.9\\
109.1	0.9\\
109.5	0.9\\
109.85	0.9\\
110.2	0.9\\
110.6	0.9\\
110.95	0.9\\
111.35	0.9\\
111.7	0.9\\
112.05	0.9\\
112.45	0.9\\
112.8	0.9\\
113.15	0.9\\
113.55	0.9\\
113.9	0.9\\
114.25	0.9\\
114.65	0.9\\
115	0.9\\
115.4	0.9\\
115.75	0.9\\
116.1	0.9\\
116.5	0.9\\
116.85	0.9\\
117.2	0.9\\
117.6	0.9\\
117.95	0.9\\
118.35	0.9\\
118.7	0.9\\
119.05	0.9\\
119.45	0.9\\
119.8	0.9\\
120.15	0.9\\
120.55	0.9\\
120.9	0.9\\
121.3	0.9\\
121.65	0.9\\
122	0.9\\
122.4	0.9\\
122.75	0.9\\
123.1	0.9\\
123.5	0.9\\
123.85	0.9\\
124.25	0.9\\
124.6	0.9\\
124.95	0.9\\
125.35	0.866666666666667\\
125.7	0.827777777777778\\
126.05	0.788888888888889\\
126.45	0.744444444444444\\
126.8	0.705555555555556\\
126.85	0.7\\
127.15	0.7\\
127.55	0.7\\
127.9	0.7\\
128.3	0.7\\
128.65	0.7\\
129	0.7\\
129.4	0.7\\
129.75	0.7\\
130.1	0.7\\
130.5	0.7\\
130.85	0.7\\
131.25	0.7\\
131.6	0.7\\
131.95	0.7\\
132.35	0.7\\
132.7	0.7\\
133.05	0.7\\
133.45	0.7\\
133.8	0.7\\
134.2	0.7\\
134.55	0.7\\
134.9	0.7\\
135.3	0.7\\
135.65	0.7\\
136	0.7\\
136.4	0.7\\
136.75	0.7\\
137.1	0.7\\
137.5	0.7\\
137.85	0.7\\
138.25	0.7\\
138.6	0.7\\
138.95	0.7\\
139.35	0.7\\
139.7	0.7\\
140.05	0.7\\
140.45	0.655555555555556\\
140.8	0.616666666666667\\
141.2	0.572222222222222\\
141.55	0.533333333333333\\
141.9	0.494444444444444\\
142.3	0.45\\
142.65	0.411111111111111\\
142.75	0.4\\
143	0.4\\
143.4	0.4\\
143.75	0.4\\
144.15	0.4\\
144.5	0.4\\
144.85	0.4\\
145.25	0.4\\
145.6	0.4\\
145.95	0.4\\
146.35	0.4\\
146.7	0.4\\
147.1	0.4\\
147.45	0.4\\
147.8	0.4\\
148.2	0.4\\
148.55	0.4\\
148.9	0.4\\
149.3	0.4\\
149.65	0.4\\
150	0.4\\
150.4	0.4\\
150.75	0.4\\
151.15	0.4\\
151.5	0.4\\
151.85	0.4\\
152.25	0.4\\
152.6	0.4\\
152.95	0.4\\
153.35	0.4\\
153.7	0.4\\
154.1	0.4\\
154.45	0.4\\
154.8	0.4\\
155.2	0.4\\
155.55	0.4\\
155.9	0.4\\
156.3	0.4\\
156.65	0.4\\
157.05	0.4\\
157.4	0.4\\
157.75	0.4\\
158.15	0.4\\
158.5	0.4\\
158.85	0.4\\
159.25	0.4\\
159.95	0.4\\
};
\addlegendentry{C31};

\end{axis}
\end{tikzpicture}%}
%\tikzset{pressure/.style={draw, circle, inner sep=0pt, text width=5mm, align=center}}
\tikzset{difpres/.style={draw, circle, inner sep=0pt, text width=6mm, align=center}}
\tikzset{connect/.style={draw,circle, inner sep=0pt, text width=2mm, align=center,fill=black}}
\tikzset{evalve/.style={draw, circle, inner sep=0pt, text width=3mm, align=center}}
\begin{turn}{90}
\begin{tikzpicture}



%Pump north
\node[draw,circle,minimum size=1cm] (p0) at (6,6) {};
\node(p1) at ($(p0)+(-0.5,0)$) {};
\node(p2) at ($(p1)+(0.5,0.5)$) {};
\node(p3) at ($(p1)+(1,0)$) {};
\draw(p1.center) -- (p2.center) -- (p3.center);
\node at ($(p1)+(1.5,0)$) {\Large $C_{18}$};

%Pump north
\node[draw,circle,minimum size=1cm] (p0) at (-5,9) {};
\node(p1) at ($(p0)+(-0.5,0)$) {};
\node(p2) at ($(p1)+(0.5,0.5)$) {};
\node(p3) at ($(p1)+(1,0)$) {};
\draw(p1.center) -- (p2.center) -- (p3.center);
\node at ($(p1)+(1.5,0)$) {\Large $C_{32}$};

%Pump north
\node[draw,circle,minimum size=1cm] (p0) at (20,6) {};
\node(p1) at ($(p0)+(-0.5,0)$) {};
\node(p2) at ($(p1)+(0.5,0.5)$) {};
\node(p3) at ($(p1)+(1,0)$) {};
\draw(p1.center) -- (p2.center) -- (p3.center);
\node at ($(p1)+(1.5,0)$) {\Large $C_{25}$};

%Pump north
\node[draw,circle,minimum size=1cm] (p0) at (-1.5,0.5) {};
\node(p1) at ($(p0)+(-0.5,0)$) {};
\node(p2) at ($(p1)+(0.5,0.5)$) {};
\node(p3) at ($(p1)+(1,0)$) {};
\draw(p1.center) -- (p2.center) -- (p3.center);
\node at ($(p1)+(1.5,0)$) {\Large $C_{2}$};

%Pump north
\node[draw,circle,minimum size=1cm] (p0) at (28,0.5) {};
\node(p1) at ($(p0)+(-0.5,0)$) {};
\node(p2) at ($(p1)+(0.5,0.5)$) {};
\node(p3) at ($(p1)+(1,0)$) {};
\draw(p1.center) -- (p2.center) -- (p3.center);
\node at ($(p1)+(1.5,0)$) {\Large $C_{16}$};

%man-valve
\node(n1) at (6.25,5) {};
\draw(n1.center) -- ($(n1)-(0.5,0)$) --
($(n1)-(0,1)$) -- ($(n1)-(0.5,1)$) --  (n1.center);
\draw($(n1)-(0.75,0.25)$) -- ($(n1)-(0.75,0.75)$) -- 
($(n1)-(0.75,0.5)$) --  ($(n1)-(0.25,0.5)$);
%\node at ($(n1)+(0.5,-0.5)$) {\Large $C_{18d}$};

%man-valve
\node(n1) at (6.25,8) {};
\draw(n1.center) -- ($(n1)-(0.5,0)$) --
($(n1)-(0,1)$) -- ($(n1)-(0.5,1)$) --  (n1.center);
\draw($(n1)-(0.75,0.25)$) -- ($(n1)-(0.75,0.75)$) -- 
($(n1)-(0.75,0.5)$) --  ($(n1)-(0.25,0.5)$);
%\node at ($(n1)+(0.5,-0.5)$) {\Large $C_{18u}$};

%man-valve
\node(n1) at (20.25,8) {};
\draw(n1.center) -- ($(n1)-(0.5,0)$) --
($(n1)-(0,1)$) -- ($(n1)-(0.5,1)$) --  (n1.center);
\draw($(n1)-(0.75,0.25)$) -- ($(n1)-(0.75,0.75)$) -- 
($(n1)-(0.75,0.5)$) --  ($(n1)-(0.25,0.5)$);
%\node at ($(n1)+(0.5,-0.5)$) {\Large $C_{25u}$};

%man-valve
\node(n1) at (20.25,5) {};
\draw(n1.center) -- ($(n1)-(0.5,0)$) --
($(n1)-(0,1)$) -- ($(n1)-(0.5,1)$) --  (n1.center);
\draw($(n1)-(0.75,0.25)$) -- ($(n1)-(0.75,0.75)$) -- 
($(n1)-(0.75,0.5)$) --  ($(n1)-(0.25,0.5)$);
%\node at ($(n1)+(0.5,-0.5)$) {\Large $C_{25d}$};

%man-valve
\node(n1) at (-1.25,2.5) {};
\draw(n1.center) -- ($(n1)-(0.5,0)$) --
($(n1)-(0,1)$) -- ($(n1)-(0.5,1)$) --  (n1.center);
\draw($(n1)-(0.75,0.25)$) -- ($(n1)-(0.75,0.75)$) -- 
($(n1)-(0.75,0.5)$) --  ($(n1)-(0.25,0.5)$);
%\node at ($(n1)+(0.4,-0.5)$) {\Large $C_{3}$};

%man-valve
\node(n1) at (-1.25,-0.5) {};
\draw(n1.center) -- ($(n1)-(0.5,0)$) --
($(n1)-(0,1)$) -- ($(n1)-(0.5,1)$) --  (n1.center);
\draw($(n1)-(0.75,0.25)$) -- ($(n1)-(0.75,0.75)$) -- 
($(n1)-(0.75,0.5)$) --  ($(n1)-(0.25,0.5)$);
%\node at ($(n1)+(0.4,-0.5)$) {\Large $C_{1}$};

%man-valve
\node(n1) at (-4.75,11.5) {};
\draw(n1.center) -- ($(n1)-(0.5,0)$) --
($(n1)-(0,1)$) -- ($(n1)-(0.5,1)$) --  (n1.center);
\draw($(n1)-(0.75,0.25)$) -- ($(n1)-(0.75,0.75)$) -- 
($(n1)-(0.75,0.5)$) --  ($(n1)-(0.25,0.5)$);
%\node at ($(n1)+(0.5,-0.5)$) {\Large $C_{6}$};

%man-valve
\node(n1) at (-4.75,7.5) {};
\draw(n1.center) -- ($(n1)-(0.5,0)$) --
($(n1)-(0,1)$) -- ($(n1)-(0.5,1)$) --  (n1.center);
\draw($(n1)-(0.75,0.25)$) -- ($(n1)-(0.75,0.75)$) -- 
($(n1)-(0.75,0.5)$) --  ($(n1)-(0.25,0.5)$);
%\node at ($(n1)+(0.5,-0.5)$) {\Large $C_{5}$};

%man-valve
\node(n1) at (28.25,-0.5) {};
\draw(n1.center) -- ($(n1)-(0.5,0)$) --
($(n1)-(0,1)$) -- ($(n1)-(0.5,1)$) --  (n1.center);
\draw($(n1)-(0.75,0.25)$) -- ($(n1)-(0.75,0.75)$) -- 
($(n1)-(0.75,0.5)$) --  ($(n1)-(0.25,0.5)$);
%\node at ($(n1)+(0.5,-0.5)$) {\Large $C_{17}$};

%man-valve
\node(n1) at (28.25,2.5) {};
\draw(n1.center) -- ($(n1)-(0.5,0)$) --
($(n1)-(0,1)$) -- ($(n1)-(0.5,1)$) --  (n1.center);
\draw($(n1)-(0.75,0.25)$) -- ($(n1)-(0.75,0.75)$) -- 
($(n1)-(0.75,0.5)$) --  ($(n1)-(0.25,0.5)$);
%\node at ($(n1)+(0.5,-0.5)$) {\Large $C_{15}$};

%elec-valve
\node(n1) at (10.25,11.5) {};
\draw(n1.center) -- ($(n1)-(0.5,0)$) --
($(n1)-(0,1)$) -- ($(n1)-(0.5,1)$) --  (n1.center);
\draw($(n1)-(0.75,0.5)$) circle (1.5mm); 
\draw($(n1)-(0.6,0.5)$)--  ($(n1)-(0.25,0.5)$);
\node at ($(n1)+(0.3,-0.5)$) {\Large $C_{24}$};

%elec-valve
\node(n1) at (-1.25,11.5) {};
\draw(n1.center) -- ($(n1)-(0.5,0)$) --
($(n1)-(0,1)$) -- ($(n1)-(0.5,1)$) --  (n1.center);
\draw($(n1)-(0.75,0.5)$) circle (1.5mm); 
\draw($(n1)-(0.6,0.5)$)--  ($(n1)-(0.25,0.5)$);
\node at ($(n1)+(0.3,-0.5)$) {\Large $C_{20}$};

%elec-valve
\node(n1) at (24.25,11.5) {};
\draw(n1.center) -- ($(n1)-(0.5,0)$) --
($(n1)-(0,1)$) -- ($(n1)-(0.5,1)$) --  (n1.center);
\draw($(n1)-(0.75,0.5)$) circle (1.5mm); 
\draw($(n1)-(0.6,0.5)$)--  ($(n1)-(0.25,0.5)$);
\node at ($(n1)+(0.3,-0.5)$) {\Large $C_{31}$};

%elec-valve
\node(n1) at (13.25,11.5) {};
\draw(n1.center) -- ($(n1)-(0.5,0)$) --
($(n1)-(0,1)$) -- ($(n1)-(0.5,1)$) --  (n1.center);
\draw($(n1)-(0.75,0.5)$) circle (1.5mm); 
\draw($(n1)-(0.6,0.5)$)--  ($(n1)-(0.25,0.5)$);
\node at ($(n1)+(0.3,-0.5)$) {\Large $C_{27}$};

%man-valve
%\draw[very thick](-0.25,-1) -- (0.25,-1) -- (-0.25,-2) -- (0.25,-2) -- (-0.25,-1) -- (0.25,-1);
%\draw[very thick](0,-1.5) -- (-0.5,-1.5);
%\draw[very thick](-0.5,-1.25) -- (-0.5,-1.75);


%GND
\node(g1) at (-1.5,-2.5) {};
\draw(g1.center) -- ($(g1)+(0.3,0)$) -- ($(g1)-(0.3,0)$);
\draw($(g1)-(0.15,0.1)$) -- ($(g1)+(0.15,-0.1)$);

%GND
\node(g1) at (-1.5,9.5) {};
\draw(g1.center) -- ($(g1)+(0.3,0)$) -- ($(g1)-(0.3,0)$);
\draw($(g1)-(0.15,0.1)$) -- ($(g1)+(0.15,-0.1)$);

%GND
\node(g1) at (28,-2.5) {};
\draw(g1.center) -- ($(g1)+(0.3,0)$) -- ($(g1)-(0.3,0)$);
\draw($(g1)-(0.15,0.1)$) -- ($(g1)+(0.15,-0.1)$);

%GND
\node(g1) at (13,9.5) {};
\draw(g1.center) -- ($(g1)+(0.3,0)$) -- ($(g1)-(0.3,0)$);
\draw($(g1)-(0.15,0.1)$) -- ($(g1)+(0.15,-0.1)$);

%GND
\node(g1) at (24,9.5) {};
\draw(g1.center) -- ($(g1)+(0.3,0)$) -- ($(g1)-(0.3,0)$);
\draw($(g1)-(0.15,0.1)$) -- ($(g1)+(0.15,-0.1)$);

%GND
\node(g1) at (10,9.5) {};
\draw(g1.center) -- ($(g1)+(0.3,0)$) -- ($(g1)-(0.3,0)$);
\draw($(g1)-(0.15,0.1)$) -- ($(g1)+(0.15,-0.1)$);


%pipe
\node(r1) at (12,3) {};
\draw (r1.center) -- ($(r1)+(1.5,0)$) arc (90:-90:0.05) -- 
($(r1)+(0,-0.1)$) arc (90:270:0.05) --
($(r1)+(1.5,-0.2)$) arc (90:-90:0.05) --
($(r1)+(0,-0.3)$) arc (90:270:0.05) --
($(r1)+(1.5,-0.4)$) arc (90:-90:0.05) --
($(r1)+(0,-0.5)$) arc (90:270:0.05);
\draw[rounded corners] ($(r1)+(0,-0.6)$) -- 
($(r1)+(1.65,-0.6)$) -- ($(r1)+(1.65,0)$) -- ($(r1)+(2.5,0)$);
\node at ($(r1)+(0.75,-1)$) {\Large $C_{9},C_{10}$};

%pipe
\node(r1) at (12,-1) {};
\draw (r1.center) -- ($(r1)+(1.5,0)$) arc (90:-90:0.05) -- 
($(r1)+(0,-0.1)$) arc (90:270:0.05) --
($(r1)+(1.5,-0.2)$) arc (90:-90:0.05) --
($(r1)+(0,-0.3)$) arc (90:270:0.05) --
($(r1)+(1.5,-0.4)$) arc (90:-90:0.05) --
($(r1)+(0,-0.5)$) arc (90:270:0.05);
\draw[rounded corners] ($(r1)+(0,-0.6)$) -- 
($(r1)+(1.65,-0.6)$) -- ($(r1)+(1.65,0)$) -- ($(r1)+(2.5,0)$);
\node at ($(r1)+(0.75,-1)$) {\Large $C_{12},C_{13}$};

%pipe
\node(r1) at (12,16.5) {};
\draw (r1.center) -- ($(r1)+(1.5,0)$) arc (90:-90:0.05) -- 
($(r1)+(0,-0.1)$) arc (90:270:0.05) --
($(r1)+(1.5,-0.2)$) arc (90:-90:0.05) --
($(r1)+(0,-0.3)$) arc (90:270:0.05) --
($(r1)+(1.5,-0.4)$) arc (90:-90:0.05) --
($(r1)+(0,-0.5)$) arc (90:270:0.05);
\draw[rounded corners] ($(r1)+(0,-0.6)$) -- 
($(r1)+(1.65,-0.6)$) -- ($(r1)+(1.65,0)$) -- ($(r1)+(2.5,0)$);
\node at ($(r1)+(0.75,-1)$) {\Large $C_{42}$};

%pipe
\node(r1) at (25,3) {};
\draw (r1.center) -- ($(r1)+(1.5,0)$) arc (90:-90:0.05) -- 
($(r1)+(0,-0.1)$) arc (90:270:0.05) --
($(r1)+(1.5,-0.2)$) arc (90:-90:0.05) --
($(r1)+(0,-0.3)$) arc (90:270:0.05) --
($(r1)+(1.5,-0.4)$) arc (90:-90:0.05) --
($(r1)+(0,-0.5)$) arc (90:270:0.05);
\draw[rounded corners] ($(r1)+(0,-0.6)$) -- 
($(r1)+(1.65,-0.6)$) -- ($(r1)+(1.65,0)$) -- ($(r1)+(2.5,0)$);
\node at ($(r1)+(0.75,-1)$) {\Large $C_{14}$};

%pipe
\node(r1) at (21.5,3) {};
\draw (r1.center) -- ($(r1)+(1.5,0)$) arc (90:-90:0.05) -- 
($(r1)+(0,-0.1)$) arc (90:270:0.05) --
($(r1)+(1.5,-0.2)$) arc (90:-90:0.05) --
($(r1)+(0,-0.3)$) arc (90:270:0.05) --
($(r1)+(1.5,-0.4)$) arc (90:-90:0.05) --
($(r1)+(0,-0.5)$) arc (90:270:0.05);
\draw[rounded corners] ($(r1)+(0,-0.6)$) -- 
($(r1)+(1.65,-0.6)$) -- ($(r1)+(1.65,0)$) -- ($(r1)+(2.5,0)$);
\node at ($(r1)+(0.75,-1)$) {\Large $C_{11}$};

%pipe
\node(r1) at (6,10) {};
\begin{scope} [rotate around={90:(r1)}]
\draw (r1.center) -- ($(r1)+(1.5,0)$) arc (90:-90:0.05) -- 
($(r1)+(0,-0.1)$) arc (90:270:0.05) --
($(r1)+(1.5,-0.2)$) arc (90:-90:0.05) --
($(r1)+(0,-0.3)$) arc (90:270:0.05) --
($(r1)+(1.5,-0.4)$) arc (90:-90:0.05) --
($(r1)+(0,-0.5)$) arc (90:270:0.05);
\draw[rounded corners] ($(r1)+(0,-0.6)$) -- 
($(r1)+(1.65,-0.6)$) -- ($(r1)+(1.65,0)$) -- ($(r1)+(2.5,0)$);
\end{scope}
\node at ($(r1)+(1.25,0.75)$) {\Large $C_{19}$};

%pipe
\node(r1) at (3,13) {};
\draw (r1.center) -- ($(r1)+(1.5,0)$) arc (90:-90:0.05) -- 
($(r1)+(0,-0.1)$) arc (90:270:0.05) --
($(r1)+(1.5,-0.2)$) arc (90:-90:0.05) --
($(r1)+(0,-0.3)$) arc (90:270:0.05) --
($(r1)+(1.5,-0.4)$) arc (90:-90:0.05) --
($(r1)+(0,-0.5)$) arc (90:270:0.05);
\draw[rounded corners] ($(r1)+(0,-0.6)$) -- 
($(r1)+(1.65,-0.6)$) -- ($(r1)+(1.65,0)$) -- ($(r1)+(2.5,0)$);
\node at ($(r1)+(0.75,-1)$) {\Large $C_{22}$};

%pipe
\node(r1) at (7,13) {};
\draw (r1.center) -- ($(r1)+(1.5,0)$) arc (90:-90:0.05) -- 
($(r1)+(0,-0.1)$) arc (90:270:0.05) --
($(r1)+(1.5,-0.2)$) arc (90:-90:0.05) --
($(r1)+(0,-0.3)$) arc (90:270:0.05) --
($(r1)+(1.5,-0.4)$) arc (90:-90:0.05) --
($(r1)+(0,-0.5)$) arc (90:270:0.05);
\draw[rounded corners] ($(r1)+(0,-0.6)$) -- 
($(r1)+(1.65,-0.6)$) -- ($(r1)+(1.65,0)$) -- ($(r1)+(2.5,0)$);
\node at ($(r1)+(0.75,-1)$) {\Large $C_{23}$};

%pipe
\node(r1) at (3,3) {};
\draw (r1.center) -- ($(r1)+(1.5,0)$) arc (90:-90:0.05) -- 
($(r1)+(0,-0.1)$) arc (90:270:0.05) --
($(r1)+(1.5,-0.2)$) arc (90:-90:0.05) --
($(r1)+(0,-0.3)$) arc (90:270:0.05) --
($(r1)+(1.5,-0.4)$) arc (90:-90:0.05) --
($(r1)+(0,-0.5)$) arc (90:270:0.05);
\draw[rounded corners] ($(r1)+(0,-0.6)$) -- 
($(r1)+(1.65,-0.6)$) -- ($(r1)+(1.65,0)$) -- ($(r1)+(2.5,0)$);
\node at ($(r1)+(0.75,-1)$) {\Large $C_{8}$};

%pipe
\node(r1) at (0,13) {};
\draw (r1.center) -- ($(r1)+(1.5,0)$) arc (90:-90:0.05) -- 
($(r1)+(0,-0.1)$) arc (90:270:0.05) --
($(r1)+(1.5,-0.2)$) arc (90:-90:0.05) --
($(r1)+(0,-0.3)$) arc (90:270:0.05) --
($(r1)+(1.5,-0.4)$) arc (90:-90:0.05) --
($(r1)+(0,-0.5)$) arc (90:270:0.05);
\draw[rounded corners] ($(r1)+(0,-0.6)$) -- 
($(r1)+(1.65,-0.6)$) -- ($(r1)+(1.65,0)$) -- ($(r1)+(2.5,0)$);
\node at ($(r1)+(0.75,-1)$) {\Large $C_{21}$};

%pipe
\node(r1) at (20,10) {};
\begin{scope} [rotate around={90:(r1)}]
\draw (r1.center) -- ($(r1)+(1.5,0)$) arc (90:-90:0.05) -- 
($(r1)+(0,-0.1)$) arc (90:270:0.05) --
($(r1)+(1.5,-0.2)$) arc (90:-90:0.05) --
($(r1)+(0,-0.3)$) arc (90:270:0.05) --
($(r1)+(1.5,-0.4)$) arc (90:-90:0.05) --
($(r1)+(0,-0.5)$) arc (90:270:0.05);
\draw[rounded corners] ($(r1)+(0,-0.6)$) -- 
($(r1)+(1.65,-0.6)$) -- ($(r1)+(1.65,0)$) -- ($(r1)+(2.5,0)$);
\end{scope}
\node at ($(r1)+(1.25,0.75)$) {\Large $C_{26}$};

%pipe
\node(r1) at (14,13) {};
\draw (r1.center) -- ($(r1)+(1.5,0)$) arc (90:-90:0.05) -- 
($(r1)+(0,-0.1)$) arc (90:270:0.05) --
($(r1)+(1.5,-0.2)$) arc (90:-90:0.05) --
($(r1)+(0,-0.3)$) arc (90:270:0.05) --
($(r1)+(1.5,-0.4)$) arc (90:-90:0.05) --
($(r1)+(0,-0.5)$) arc (90:270:0.05);
\draw[rounded corners] ($(r1)+(0,-0.6)$) -- 
($(r1)+(1.65,-0.6)$) -- ($(r1)+(1.65,0)$) -- ($(r1)+(2.5,0)$);
\node at ($(r1)+(0.75,-1)$) {\Large $C_{28}$};

%pipe
\node(r1) at (17,13) {};
\draw (r1.center) -- ($(r1)+(1.5,0)$) arc (90:-90:0.05) -- 
($(r1)+(0,-0.1)$) arc (90:270:0.05) --
($(r1)+(1.5,-0.2)$) arc (90:-90:0.05) --
($(r1)+(0,-0.3)$) arc (90:270:0.05) --
($(r1)+(1.5,-0.4)$) arc (90:-90:0.05) --
($(r1)+(0,-0.5)$) arc (90:270:0.05);
\draw[rounded corners] ($(r1)+(0,-0.6)$) -- 
($(r1)+(1.65,-0.6)$) -- ($(r1)+(1.65,0)$) -- ($(r1)+(2.5,0)$);
\node at ($(r1)+(0.75,-1)$) {\Large $C_{29}$};

%pipe
\node(r1) at (21,13) {};
\draw (r1.center) -- ($(r1)+(1.5,0)$) arc (90:-90:0.05) -- 
($(r1)+(0,-0.1)$) arc (90:270:0.05) --
($(r1)+(1.5,-0.2)$) arc (90:-90:0.05) --
($(r1)+(0,-0.3)$) arc (90:270:0.05) --
($(r1)+(1.5,-0.4)$) arc (90:-90:0.05) --
($(r1)+(0,-0.5)$) arc (90:270:0.05);
\draw[rounded corners] ($(r1)+(0,-0.6)$) -- 
($(r1)+(1.65,-0.6)$) -- ($(r1)+(1.65,0)$) -- ($(r1)+(2.5,0)$);
\node at ($(r1)+(0.75,-1)$) {\Large $C_{30}$};

%pipe
\node(r1) at (-0.5,3) {};
\draw (r1.center) -- ($(r1)+(1.5,0)$) arc (90:-90:0.05) -- 
($(r1)+(0,-0.1)$) arc (90:270:0.05) --
($(r1)+(1.5,-0.2)$) arc (90:-90:0.05) --
($(r1)+(0,-0.3)$) arc (90:270:0.05) --
($(r1)+(1.5,-0.4)$) arc (90:-90:0.05) --
($(r1)+(0,-0.5)$) arc (90:270:0.05);
\draw[rounded corners] ($(r1)+(0,-0.6)$) -- 
($(r1)+(1.65,-0.6)$) -- ($(r1)+(1.65,0)$) -- ($(r1)+(2.5,0)$);
\node at ($(r1)+(0.75,-1)$) {\Large $C_{4}$};

%pressure sensor
\node (PD1) at (-1.5,3) {};
\node(P1) at ($(PD1)+(0,1)$) [pressure] {P};
\draw(PD1.center) -- (P1);

%pressure sensor
\node (PD1) at (23.5,13) {};
\node(P1) at ($(PD1)+(0,1)$) [pressure] {P};
\draw(PD1.center) -- (P1);

%pressure sensor
\node (PD1) at (9.5,13) {};
\node(P1) at ($(PD1)+(0,1)$) [pressure] {P};
\draw(PD1.center) -- (P1);

%pressure sensor
\node (PD1) at (28,3) {};
\node(P1) at ($(PD1)+(0,1)$) [pressure] {P};
\draw(PD1.center) -- (P1);

%pressure sensor
\node (PD1) at (16.5,13) {};
\node(P1) at ($(PD1)+(0,1)$) [pressure] {P};
\draw(PD1.center) -- (P1);

%pressure sensor
\node (PD1) at (2.5,13) {};
\node(P1) at ($(PD1)+(0,1)$) [pressure] {P};
\draw(PD1.center) -- (P1);

%pressure sensor vert
\node (PD1) at (6,3.5) {};
\node(P1) at ($(PD1)+(1,0)$) [pressure] {P};
\draw(PD1.center) -- (P1);

%pressure sensor vert
\node(PD1) at (20,3.5) {};
\node(P1) at ($(PD1)+(1,0)$) [pressure] {P};
\draw(PD1.center) -- (P1);

%differential pressure sensor
\node(CDP1) at (-1.5,3) {};
\node(CDP2) at (-1.5,-2) {};
\node[difpres] (DP1) at (-3,0.5) {DP};
\draw(CDP1.center) -| (DP1)  |- (CDP2.center);

%differential pressure sensor
\node(CDP1) at (20,8.5) {};
\node (CDP2) at (20,3.5) {};
\node[difpres] (DP1) at (18.5,6) {DP};
\draw(CDP1.center) -| (DP1)  |- (CDP2.center);

%differential pressure sensor
\node(CDP1) at (6,8.5) {};
\node(CDP2) at (6,3.5) {};
\node[difpres] (DP1) at (4.5,6) {DP};
\draw(CDP1.center) -| (DP1)  |- (CDP2.center);

%differential pressure sensor
\node(CDP1) at (28,3) {};
\node(CDP2) at (28,-2) {};
\node[difpres] (DP1) at (30,0.5) {DP};
\draw(CDP1.center) -| (DP1)  |- (CDP2.center);

%Connections staight lines
\draw (-1.5,3) node (v1) {} -- (-1.5,2.5);
\draw (-1.5,1.5) -- (-1.5,1);
\draw (-1.5,0) -- (-1.5,-0.5);
\draw (-1.5,-1.5) -- (-1.5,-2.5);
\draw (-0.5,3) -- (-1.5,3);
\draw (3,3) -- (2,3);
\draw(5.5,3) -- (12,3);
\draw(6,3) -- (6,4);
\draw(6,5) -- (6,5.5);
\draw(6,6.5) -- (6,7);
\draw(6,8) -- (6,10);
\draw(5.5,13) -- (7,13);
\draw(2.5,13) -- (3,13);
\draw(-1.5,10.5) -- (-1.5,9.5);
\draw(10,10.5) -- (10,9.5);
\draw(13,10.5) -- (13,9.5);
\draw(24,10.5) -- (24,9.5);
\draw(20,10) -- (20,8);
\draw(20,7) -- (20,6.5);
\draw(20,5.5) -- (20,5);
\draw(20,4) -- (20,3);
\draw(14.5,3) -- (21.5,3);
\draw(24,3) -- (25,3);
\draw(27.5,3) -- (28,3);
\draw(28,3) -- (28,2.5);
\draw(28,3) -- (28,2.5);
\draw(28,1.5) -- (28,1);
\draw(28,0) -- (28,-0.5);
\draw(28,-1.5) -- (28,-2.5);
\draw(19.5,13) -- (21,13);
\draw(-5,8.5) -- (-5,7.5);
\draw(-5,11.5) -- (-5,12.5);
\draw(-6.5,14.5) -- (-6.5,12.5) -- (-3.5,12.5) -- (-3.5,14.5);
\draw(16.5,13) -- (17,13);
\draw(-5,9.5) -- (-5,10.5);
%Connections bend lines
\draw[rounded corners](24,3) |- (14.5,-1);
\draw[rounded corners](9.5,13) -| (10,11.5);
\draw[rounded corners](23.5,13) -| (24,11.5);
\draw[rounded corners](14.5,16.5) -| (20,12.5);
\draw[rounded corners](12,16.5) -| (6,12.5);
\draw[rounded corners](0,13) -| (-1.5,11.5);
\draw[rounded corners](14,13) -| (13,11.5);
\draw[rounded corners](-5,6.5) |- (-5,5) -| (2,3);
\draw[rounded corners](2,3) |- (12,-1);
\draw (-6.5,14) .. controls (-5,13.5) and (-5,14.5) .. (-3.5,14);
%nodes
\node[connect] (N) at (-1.5,-2) {};
\node at ($(N)+(0.5,0)$) {\Large $n_{1}$};
\node[connect] (N) at (28,-2) {};
\node at ($(N)+(-0.5,0)$) {\Large $n_{1}$};
\node[connect] (N) at (24,10) {};
\node at ($(N)+(0.5,0)$) {\Large $n_{1}$};
\node[connect] (N) at (13,10) {};
\node at ($(N)+(0.5,0)$) {\Large $n_{1}$};
\node[connect] (N) at (10,10) {};
\node at ($(N)+(0.5,0)$) {\Large $n_{1}$};
\node[connect] (N) at (-1.5,10) {};
\node at ($(N)+(0.5,0)$) {\Large $n_{1}$};
\node[connect] (N) at (-1.5,3) {};
\node at ($(N)+(0.4,0.5)$) {\Large $n_{2}$};
\node[connect] (N) at (2,3) {};
\node at ($(N)+(0.4,0.5)$) {\Large $n_{3}$};
\node[connect] (N) at (6,3) {};
\node at ($(N)+(0,-0.5)$) {\Large $n_{4}$};
\node[connect] (N) at (20,3) {};
\node at ($(N)+(0,-0.5)$) {\Large $n_{5}$};
\node[connect] (N) at (24,3) {};
\node at ($(N)+(0,0.5)$) {\Large $n_{6}$};
\node[connect] (N) at (28,3) {};
\node at ($(N)+(0.4,0.4)$) {\Large $n_{7}$};
\node[connect] (N) at (6,8.5) {};
\node at ($(N)+(0.5,0)$) {\Large $n_{8}$};
\node[connect] (N) at (6,13) {};
\node at ($(N)+(0.4,0.5)$) {\Large $n_{9}$};
\node[connect] (N) at (10,13) {};
\node at ($(N)+(0,0.5)$) {\Large $n_{10}$};
\node[connect] (N) at (2.5,13) {};
\node at ($(N)+(0,-0.4)$) {\Large $n_{11}$};
\node[connect] (N) at (-1.5,13) {};
\node at ($(N)+(0,0.5)$) {\Large $n_{12}$};
\node[connect] (N) at (20,8.5) {};
\node at ($(N)+(0.5,0)$) {\Large $n_{13}$};
\node[connect] (N) at (20,13) {};
\node at ($(N)+(0.4,0.5)$) {\Large $n_{14}$};
\node[connect] (N) at (24,13) {};
\node at ($(N)+(0,0.5)$) {\Large $n_{15}$};
\node[connect] (N) at (16.5,13) {};
\node at ($(N)+(0,-0.4)$) {\Large $n_{16}$};
\node[connect] (N) at (13,13) {};
\node at ($(N)+(0,0.5)$) {\Large $n_{17}$};
\node[connect] (N) at (-5,12.5) {};
\node at ($(N)+(0.5,-0.4)$) {\Large $n_{18}$};
%PMA
\draw[thick,dashed] (25,9) node (v2) {} -- (25,14.5) -- (12,14.5) -- (12,9) -- (v2);
\draw[thick,dashed] (-2.5,9) node (v2) {} -- (-2.5,14.5) -- (11,14.5) -- (11,9) -- (v2);
\node at (0.5,15) {\Large PMA 1};
\node at (-2,15) {\Large 0 m};
\node at (22,15) {\Large PMA 2};
\node at (24.5,15) {\Large 0.5 m};
\node at (-5.15,14.85) {\Large 2 m};
\node at (-3,13.5) {\Large $C_{33}$};
\end{tikzpicture}
\end{turn}    
\caption{Inputs to the parameter identification}
\label{systemdiagram}
\end{figure} 

The estimation process, based on the applied input data, tries to fit the established output signal data to the simulation output. In the physical setup $8$ 
different pressure relative sensors are available which are used as output signals. \figref{systemdiagram1} shows the pressure measurements in the different 
nodes.

\begin{figure}[H]
\centering
\resizebox{0.75\linewidth}{!}{% This file was created by matlab2tikz.
%
%The latest updates can be retrieved from
%  http://www.mathworks.com/matlabcentral/fileexchange/22022-matlab2tikz-matlab2tikz
%where you can also make suggestions and rate matlab2tikz.
%
\definecolor{mycolor1}{rgb}{0.00000,0.44700,0.74100}%
\definecolor{mycolor2}{rgb}{0.85000,0.32500,0.09800}%
%
\begin{tikzpicture}

\begin{axis}[%
width=6.028in,
height=4.754in,
at={(1.011in,0.642in)},
scale only axis,
xmin=0,
xmax=160,
xlabel={Seconds},
xmajorgrids,
ymin=0,
ymax=0.45,
ylabel={Bar},
ymajorgrids,
axis background/.style={fill=white},
title style={font=\bfseries},
title={Relative pump pressures},
legend style={legend cell align=left,align=left,draw=white!15!black}
]
\addplot [color=mycolor1,solid]
  table[row sep=crcr]{%
0	0.0119562561094821\\
0.15	0.0119379276637343\\
0.35	0.0119990224828936\\
0.4	0.0119990224828936\\
0.75	0.0120967741935485\\
0.85	0.0121639784946238\\
1.15	0.012145650048876\\
1.5	0.012047898338221\\
1.65	0.0119929130009777\\
1.85	0.0120234604105573\\
2.25	0.0122556207233628\\
2.5	0.0123594819159337\\
2.6	0.0123106060606062\\
2.95	0.0124266862170089\\
3.2	0.0124877810361683\\
3.5	0.0124511241446727\\
3.65	0.0125061094819161\\
3.7	0.0125061094819161\\
3.8	0.0124877810361683\\
4.2	0.0125427663734117\\
4.4	0.0124755620723364\\
4.45	0.0124816715542524\\
4.8	0.01269550342131\\
5.2	0.0125855327468232\\
5.5	0.0125183284457479\\
5.55	0.0125244379276638\\
5.9	0.0126099706744869\\
6.3	0.0137829912023461\\
6.65	0.0158663245356794\\
7.05	0.0187194525904204\\
7.4	0.0213770772238515\\
7.75	0.0244440371456501\\
8.15	0.0279630987292278\\
8.5	0.0311766862170088\\
8.85	0.0343658357771261\\
9.25	0.0379093352883676\\
9.6	0.0409824046920822\\
10	0.0445320136852396\\
10.35	0.0476600684261975\\
10.7	0.0508369990224829\\
11.1	0.0539711632453568\\
11.45	0.055608504398827\\
11.8	0.0565371456500489\\
12.2	0.0570136852394917\\
12.55	0.0572030791788856\\
12.85	0.0572519550342131\\
12.9	0.0572458455522971\\
13.3	0.0571542033235581\\
13.65	0.057074780058651\\
13.8	0.0570992179863147\\
14.05	0.0569525904203323\\
14.4	0.056891495601173\\
14.45	0.0568548387096773\\
14.75	0.0569159335288367\\
15.15	0.0570564516129031\\
15.2	0.0570442326490713\\
15.4	0.0570992179863147\\
15.5	0.0570808895405669\\
15.75	0.0569892473118279\\
15.9	0.0570014662756598\\
16.2	0.0569220430107526\\
16.3	0.0569342619745845\\
16.5	0.0568670576735093\\
16.6	0.0568792766373412\\
16.9	0.0568487292277615\\
17.15	0.0568365102639296\\
17.35	0.056897605083089\\
17.45	0.0569220430107528\\
17.7	0.0568792766373414\\
17.95	0.0566898826979474\\
18.1	0.0567021016617793\\
18.45	0.0568059628543502\\
18.75	0.0568731671554256\\
18.85	0.0568609481915937\\
19.1	0.0569037145650054\\
19.25	0.0568853861192576\\
19.5	0.056952590420333\\
19.55	0.0569464809384171\\
19.9	0.056763196480939\\
19.95	0.056763196480939\\
20.2	0.0566715542522002\\
20.3	0.056683773216032\\
20.65	0.0567509775171074\\
20.8	0.0567631964809393\\
21.05	0.0564943792766382\\
21.4	0.0554252199413497\\
21.75	0.0538550830889548\\
22.15	0.0518389540566966\\
22.5	0.0499450146627571\\
22.9	0.0480144183773219\\
23.25	0.046157135874878\\
23.6	0.0443487292277615\\
24	0.0423448191593352\\
24.35	0.0405364125122187\\
24.7	0.0388379765395889\\
25.1	0.0367485337243395\\
25.45	0.0348912512218955\\
25.8	0.0330584066471153\\
26.2	0.0316776637341142\\
26.55	0.0312072336265873\\
26.95	0.0310117302052774\\
27.3	0.0309323069403703\\
27.45	0.0308895405669588\\
27.6	0.0309445259042023\\
27.65	0.0309261974584545\\
28.05	0.0311950146627556\\
28.4	0.0313660801564017\\
28.6	0.0313477517106539\\
28.75	0.0314760508308885\\
28.9	0.0314393939393928\\
29.15	0.0315554740957955\\
29.25	0.0315676930596274\\
29.45	0.0315371456500477\\
29.5	0.0315493646138796\\
29.9	0.0317326490713576\\
30.25	0.0318731671554242\\
30.55	0.0319586999022473\\
30.6	0.0319586999022473\\
30.9	0.0320686705767341\\
31.05	0.0320381231671544\\
31.15	0.0320747800586501\\
31.35	0.0320564516129023\\
31.7	0.0322947214076237\\
32	0.0324718963831859\\
32.1	0.0324718963831859\\
32.4	0.0327040566959913\\
32.55	0.0326673998044957\\
32.85	0.0327284946236551\\
33.1	0.0326429618768318\\
33.2	0.0326429618768318\\
33.5	0.0325940860215044\\
33.65	0.0326490713587478\\
33.8	0.0325879765395883\\
33.95	0.032624633431084\\
34.05	0.0326001955034202\\
34.35	0.0325940860215042\\
34.55	0.0326429618768318\\
34.9	0.0326185239491679\\
35.05	0.0325635386119244\\
35.4	0.0326551808406634\\
35.75	0.0328262463343096\\
36.15	0.0355388563049842\\
36.5	0.0448619257086988\\
36.9	0.060929863147604\\
37.25	0.0741691104594321\\
37.6	0.0875549853372427\\
38	0.102840909090908\\
38.35	0.11649560117302\\
38.7	0.130419110459432\\
39.1	0.146395405669599\\
39.45	0.160355571847507\\
39.85	0.176240224828935\\
40.2	0.190108748778104\\
40.55	0.203714565004889\\
40.95	0.21831622678397\\
41.3	0.227663734115348\\
41.55	0.228885630498535\\
41.75	0.228366324535681\\
42.05	0.228897849462367\\
42.4	0.229521016617793\\
42.8	0.22999144672532\\
43.15	0.230504643206259\\
43.5	0.230620723362661\\
43.9	0.230498533724343\\
44	0.230529081133922\\
44.35	0.230486314760511\\
44.6	0.230571847507334\\
45	0.230632942326493\\
45.35	0.230730694037148\\
45.75	0.231341642228742\\
45.85	0.23146383186706\\
46.1	0.231372189638321\\
46.45	0.23152492668622\\
46.85	0.231634897360706\\
47.2	0.231555474095799\\
47.25	0.231561583577715\\
47.5	0.23152492668622\\
47.55	0.231537145650051\\
47.9	0.231714320625613\\
47.95	0.231695992179866\\
48.3	0.231537145650051\\
48.35	0.231531036168136\\
48.65	0.231586021505379\\
48.9	0.23165933528837\\
49.1	0.231641006842622\\
49.2	0.231665444770286\\
49.4	0.231653225806454\\
49.8	0.231531036168136\\
49.95	0.23149437927664\\
50.35	0.231512707722388\\
50.5	0.231476050830892\\
50.9	0.232355816226786\\
51.25	0.237396138807431\\
51.6	0.247391251221898\\
52	0.261638563049856\\
52.35	0.275146627565985\\
52.75	0.29068914956012\\
53.1	0.30493035190616\\
53.45	0.319061583577714\\
53.85	0.335215053763442\\
54.2	0.349327956989248\\
54.55	0.363599706744868\\
54.95	0.379905913978494\\
55.3	0.393939393939393\\
55.7	0.41005009775171\\
56.05	0.422134652981426\\
56.4	0.428763440860213\\
56.8	0.431885386119255\\
57.15	0.433082844574777\\
57.5	0.433773216031277\\
57.8	0.434029814271746\\
57.95	0.434005376344084\\
58.25	0.434121456500488\\
58.5	0.434145894428154\\
58.6	0.434121456500491\\
59	0.434176441837736\\
59.25	0.434097018572831\\
59.45	0.434139784946244\\
59.65	0.433987047898347\\
59.75	0.434035923753675\\
60.1	0.434176441837744\\
60.4	0.434280303030317\\
60.55	0.434243646138822\\
60.8	0.434304740957984\\
60.85	0.434286412512236\\
61.05	0.434133675464339\\
61.2	0.434188660801584\\
61.55	0.434084799609015\\
61.75	0.434084799609016\\
61.95	0.433956500488783\\
62.1	0.433925953079205\\
62.3	0.433932062561122\\
62.7	0.434017595307948\\
62.75	0.434023704789864\\
63.05	0.433956500488789\\
63.15	0.433919843597293\\
63.4	0.433956500488789\\
63.8	0.434005376344116\\
63.9	0.434072580645192\\
64.15	0.434042033235613\\
64.5	0.43409090909094\\
64.7	0.434261974584586\\
65	0.434213098729259\\
65.2	0.434237536656922\\
65.25	0.434231427175006\\
65.65	0.433974828934536\\
66	0.430321358748804\\
66.35	0.416654447702858\\
66.75	0.396627565982426\\
67.1	0.374272971652023\\
67.45	0.350635386119273\\
67.85	0.323197702834812\\
68.2	0.298869745845562\\
68.6	0.270986070381238\\
68.95	0.246700879765399\\
69.3	0.222330156402738\\
69.7	0.194312072336264\\
70.05	0.170032991202341\\
70.4	0.145747800586502\\
70.8	0.118334555229708\\
71.15	0.102657624633424\\
71.5	0.0935300586510177\\
71.9	0.0883797653958855\\
72.25	0.0873228250244296\\
72.65	0.086772971651996\\
73	0.0865957966764341\\
73.35	0.0865163734115271\\
73.55	0.0864491691104519\\
73.75	0.0864491691104519\\
74.1	0.0863453079178811\\
74.3	0.08627199413489\\
74.45	0.0863208699902174\\
74.85	0.0861620234604032\\
75.2	0.0860826001954962\\
75.45	0.0860764907135803\\
75.6	0.0861131476050758\\
75.85	0.0861681329423192\\
75.95	0.0861436950146555\\
76.25	0.0860887096774121\\
76.3	0.0861131476050758\\
76.7	0.0862475562072262\\
76.9	0.0862231182795625\\
77.05	0.0862781036168059\\
77.1	0.0862842130987218\\
77.2	0.0862536656891422\\
77.55	0.0863025415444696\\
77.8	0.0862108993157307\\
77.9	0.086186461388067\\
78.15	0.0863330889540492\\
78.55	0.0865408113391908\\
78.7	0.0865530303030226\\
78.85	0.0865347018572749\\
78.9	0.0865408113391908\\
79.25	0.0866263440860138\\
79.3	0.0866324535679297\\
79.65	0.0865347018572749\\
79.8	0.0865530303030226\\
80	0.0864858260019474\\
80.15	0.0864247311827882\\
80.35	0.08643695014662\\
80.7	0.0864858260019474\\
80.75	0.0864858260019474\\
81.1	0.0881598240469128\\
81.45	0.0927724828934434\\
81.85	0.101454056695987\\
82.2	0.108993157380252\\
82.6	0.118047409579668\\
82.95	0.126209677419358\\
83.3	0.134133675464327\\
83.7	0.143212365591407\\
84.05	0.151337976539601\\
84.4	0.159543010752702\\
84.8	0.168908846529831\\
85.15	0.177126099706764\\
85.55	0.18627199413492\\
85.9	0.193976050830915\\
86.25	0.199382942326518\\
86.6	0.201038611925738\\
86.65	0.201038611925738\\
87	0.201423509286442\\
87.35	0.201686217008827\\
87.6	0.201747311827987\\
87.75	0.201625122189668\\
87.9	0.201551808406677\\
88.1	0.201570136852425\\
88.5	0.201661779081164\\
88.85	0.201582355816257\\
88.9	0.201588465298173\\
89.2	0.201496823069434\\
89.45	0.201527370479013\\
89.7	0.201539589442845\\
89.85	0.201454056696022\\
90.1	0.201423509286442\\
90.25	0.201466275659854\\
90.4	0.201478494623686\\
90.5	0.20144183773219\\
90.7	0.201466275659854\\
91	0.201386852394947\\
91.05	0.201399071358779\\
91.4	0.201350195503451\\
91.65	0.201380742913031\\
91.8	0.201331867057704\\
92.15	0.20130742913004\\
92.4	0.201325757575788\\
92.55	0.201270772238544\\
92.6	0.201252443792796\\
92.9	0.201295210166208\\
93.15	0.201356304985367\\
93.4	0.201319648093872\\
93.6	0.201374633431115\\
93.85	0.201417399804527\\
94	0.201344086021535\\
94.1	0.201331867057704\\
94.3	0.201405180840695\\
94.35	0.201392961876863\\
94.6	0.201484604105602\\
94.75	0.201454056696022\\
94.85	0.201392961876863\\
95.1	0.201411290322611\\
95.25	0.201460166177938\\
95.5	0.201447947214106\\
95.85	0.201063049853402\\
96	0.200647605083117\\
96.2	0.201454056696019\\
96.6	0.197519550342155\\
96.95	0.191666666666691\\
97.3	0.184927908113418\\
97.7	0.177040566959949\\
98.05	0.170216275659853\\
98.45	0.162286168132972\\
98.8	0.155278592375398\\
99.15	0.148319892473151\\
99.55	0.140298142717532\\
99.9	0.133339442815285\\
100.25	0.12625855327472\\
100.65	0.118310117302092\\
101	0.112017350928684\\
101.4	0.104508797654006\\
101.75	0.102669843597312\\
102.1	0.102046676441887\\
102.5	0.101942815249316\\
102.55	0.101955034213148\\
102.85	0.101863391984409\\
103.2	0.101741202346091\\
103.45	0.101722873900343\\
103.7	0.101735092864175\\
103.95	0.101838954056745\\
104.1	0.101857282502493\\
104.3	0.101747311828007\\
104.5	0.101728983382259\\
104.7	0.101790078201418\\
104.75	0.101808406647166\\
104.85	0.101783968719502\\
105.05	0.101796187683334\\
105.45	0.101722873900343\\
105.7	0.101735092864175\\
105.8	0.101698435972679\\
105.95	0.101667888563099\\
106.25	0.101649560117352\\
106.45	0.101765640273754\\
106.55	0.101722873900343\\
106.85	0.101619012707772\\
106.9	0.101631231671604\\
107.05	0.101765640273754\\
107.25	0.101747311828007\\
107.65	0.101814516129082\\
107.85	0.101851173020577\\
108	0.101820625610998\\
108.35	0.101753421309922\\
108.45	0.101765640273754\\
108.75	0.101661779081184\\
109.1	0.101625122189688\\
109.5	0.101533479960949\\
109.85	0.101588465298192\\
110.2	0.101759530791838\\
110.5	0.101930596285484\\
110.6	0.101906158357821\\
110.85	0.101955034213148\\
110.95	0.1019367057674\\
111.2	0.101979472140812\\
111.35	0.101979472140812\\
111.7	0.101857282502493\\
111.85	0.101863391984409\\
112.05	0.101698435972679\\
112.25	0.101722873900343\\
112.45	0.101667888563099\\
112.8	0.101515151515201\\
113.15	0.101454056696042\\
113.45	0.101392961876883\\
113.55	0.101392961876883\\
113.9	0.101435728250294\\
114.15	0.101454056696042\\
114.25	0.101411290322631\\
114.35	0.101423509286462\\
114.65	0.101368523949219\\
115	0.101197458455573\\
115.35	0.100867546432113\\
115.4	0.100873655914029\\
115.75	0.100714809384215\\
116.1	0.100562072336317\\
116.5	0.100274926686268\\
116.55	0.100281036168184\\
116.85	0.100177174975613\\
116.9	0.100158846529865\\
117.15	0.100207722385193\\
117.2	0.100195503421361\\
117.6	0.10012218963837\\
117.7	0.100152737047949\\
118.15	0.100164956011781\\
118.35	0.100036656891547\\
118.45	0.100006109481967\\
118.65	0.100073313783042\\
118.7	0.100073313783042\\
119.05	0.09992668621706\\
119.35	0.0998594819159848\\
119.45	0.0998717008798167\\
119.8	0.0998167155425733\\
120.1	0.0996639784946751\\
120.15	0.099670087976591\\
120.55	0.0998350439883211\\
120.6	0.0998594819159848\\
121	0.0998350439883211\\
121.3	0.0999633431085556\\
121.65	0.100103861192622\\
121.85	0.100201612903277\\
122	0.100183284457529\\
122.35	0.100085532746874\\
122.4	0.10009164222879\\
122.75	0.100219941349025\\
122.85	0.100250488758604\\
123.2	0.100213831867109\\
123.4	0.100384897360755\\
123.55	0.100354349951175\\
123.65	0.100378787878839\\
123.85	0.100366568915007\\
124.25	0.10045210166183\\
124.5	0.100500977517157\\
124.6	0.100494868035241\\
124.95	0.100592619745896\\
125.35	0.100635386119308\\
125.5	0.100623167155476\\
125.7	0.100684261974635\\
126	0.100788123167206\\
126.1	0.100794232649122\\
126.45	0.100580400782064\\
126.65	0.100555962854401\\
126.75	0.100592619745896\\
126.8	0.100592619745896\\
126.9	0.10061705767356\\
127.3	0.100659824046971\\
127.45	0.100604838709728\\
127.55	0.10061705767356\\
127.65	0.100598729227812\\
127.9	0.10061705767356\\
128.2	0.100678152492719\\
128.3	0.100665933528887\\
128.65	0.100537634408653\\
128.9	0.100604838709728\\
129	0.100568181818232\\
129.2	0.100507086999073\\
129.4	0.100525415444821\\
129.6	0.100555962854401\\
129.75	0.100549853372485\\
130.1	0.100727028348047\\
130.45	0.100861436950197\\
130.5	0.100861436950197\\
130.85	0.100794232649122\\
131.25	0.100983626588516\\
131.45	0.101130254154498\\
131.6	0.101050830889591\\
131.95	0.100928641251272\\
132.3	0.100836999022533\\
132.35	0.100861436950197\\
132.5	0.100983626588516\\
132.7	0.100873655914029\\
132.9	0.100843108504449\\
133.05	0.100855327468281\\
133.45	0.100989736070431\\
133.65	0.10108137829917\\
133.9	0.100995845552347\\
134.15	0.101105816226834\\
134.2	0.101105816226834\\
134.35	0.101044721407675\\
134.55	0.101063049853423\\
134.7	0.101093597263002\\
134.9	0.101069159335339\\
135.3	0.100971407624684\\
135.4	0.100965298142768\\
135.65	0.101032502443843\\
135.95	0.100940860215104\\
136	0.100940860215104\\
136.4	0.100739247311878\\
136.7	0.100885874877861\\
136.75	0.100885874877861\\
136.95	0.100934750733188\\
137.3	0.100971407624684\\
137.5	0.100818670576786\\
137.55	0.100794232649122\\
137.75	0.100849217986365\\
137.85	0.10081256109487\\
138	0.100745356793794\\
138.25	0.100775904203374\\
138.55	0.100702590420383\\
138.6	0.100714809384215\\
138.9	0.100861436950197\\
139	0.100843108504449\\
139.35	0.100891984359777\\
139.4	0.100898093841693\\
139.7	0.100806451612954\\
140.05	0.100672043010803\\
140.45	0.100568181818232\\
140.65	0.10058651026398\\
140.8	0.100384897360755\\
141.2	0.0982282502444242\\
141.55	0.0940493646139196\\
141.9	0.0885508308895744\\
142.3	0.0817143206256397\\
142.65	0.0757209188661044\\
143	0.0694525904203523\\
143.4	0.0621945259042185\\
143.75	0.0556573802541656\\
144.15	0.0483993157380318\\
144.5	0.0420821114369524\\
144.85	0.0358260019550322\\
145.25	0.0287512218963762\\
145.6	0.0223790322580534\\
145.95	0.0165933528836604\\
146.35	0.0128299120234487\\
146.7	0.0114552785923649\\
147.1	0.0108260019550244\\
147.45	0.0106182795698828\\
147.5	0.0106243890517987\\
147.7	0.0105694037145554\\
147.8	0.010599951124135\\
148.15	0.0106671554252102\\
148.3	0.0106121700879669\\
148.55	0.0107160312805377\\
148.85	0.0107771260996969\\
148.95	0.0107649071358651\\
149.2	0.0108015640273607\\
149.3	0.010771016617781\\
149.65	0.0108015640273607\\
149.7	0.0108076735092766\\
149.95	0.0107160312805377\\
150.1	0.0107404692082014\\
150.3	0.010685483870958\\
150.5	0.0107343597262854\\
150.7	0.0106488269794625\\
150.75	0.0106549364613784\\
151.1	0.0105877321603032\\
151.15	0.0105938416422191\\
151.35	0.0106488269794625\\
151.55	0.0106366080156306\\
151.75	0.0105632942326395\\
151.85	0.0105694037145554\\
152	0.0105083088953961\\
152.25	0.0105266373411439\\
152.55	0.0104655425219846\\
152.65	0.0105083088953961\\
152.9	0.0104655425219846\\
153	0.0105021994134802\\
153.15	0.0104716520039005\\
153.35	0.0104777614858165\\
153.5	0.0104227761485731\\
153.7	0.0104594330400687\\
153.85	0.0103983382209094\\
154.15	0.0103983382209094\\
154.3	0.0104594330400687\\
154.6	0.010434995112405\\
154.7	0.0103922287389935\\
154.8	0.010434995112405\\
154.95	0.010514418377312\\
155.35	0.0104411045943209\\
155.55	0.0104655425219846\\
155.7	0.010428885630489\\
155.9	0.0104838709677324\\
156	0.0104716520039005\\
156.15	0.0105327468230598\\
156.3	0.0104716520039005\\
156.6	0.0104166666666572\\
156.65	0.0104227761485731\\
157	0.0104838709677324\\
157.05	0.0104777614858165\\
157.25	0.0104227761485731\\
157.4	0.010428885630489\\
157.6	0.0103800097751616\\
157.75	0.0103922287389935\\
158.15	0.0103555718474979\\
158.2	0.010349462365582\\
158.35	0.0104044477028253\\
158.55	0.0103800097751616\\
158.85	0.0103311339198342\\
159.25	0.0102394916910953\\
159.45	0.0102761485825908\\
159.6	0.0102150537634316\\
159.95	0.010263929618759\\
};
\addlegendentry{C2};

\addplot [color=mycolor2,solid]
  table[row sep=crcr]{%
0	0.0229472140762463\\
0.2	0.022886119257087\\
0.4	0.0229227761485826\\
0.75	0.0230021994134897\\
1	0.0230571847507331\\
1.25	0.0230144183773216\\
1.5	0.0230816226783968\\
1.75	0.023069403714565\\
1.85	0.0231182795698925\\
2.15	0.0230510752688172\\
2.25	0.023069403714565\\
2.3	0.0230938416422287\\
2.6	0.0230755131964809\\
2.95	0.0231304985337244\\
3.35	0.0231915933528837\\
3.65	0.023258797653959\\
3.7	0.023258797653959\\
4.05	0.0231366080156403\\
4.2	0.0231182795698925\\
4.45	0.023161045943304\\
4.65	0.0230938416422287\\
4.9	0.0231671554252199\\
5.05	0.0231060606060606\\
5.2	0.0231427174975562\\
5.55	0.0230510752688171\\
5.6	0.0230449657869012\\
5.9	0.0231671554252198\\
6.3	0.02424853372434\\
6.65	0.0262646627565981\\
7.05	0.0292216520039099\\
7.4	0.0320686705767349\\
7.75	0.0349584555229714\\
8.15	0.0383797653958941\\
8.5	0.0414222873900292\\
8.85	0.0445442326490712\\
9.25	0.0482710166177907\\
9.6	0.0514540566959919\\
10	0.0550769794721405\\
10.35	0.0582783479960897\\
10.7	0.0613453079178884\\
11.1	0.0642961876832845\\
11.45	0.0656830400782014\\
11.8	0.0661412512218964\\
12.2	0.065988514173998\\
12.55	0.0658113391984358\\
12.9	0.0655730694037142\\
13.3	0.065133186705767\\
13.65	0.0647177419354834\\
14.05	0.064094574780058\\
14.4	0.0635691593352876\\
14.75	0.063074291300097\\
15.15	0.0624816715542513\\
15.5	0.0619134897360695\\
15.85	0.061565249266861\\
16.25	0.0611864613880729\\
16.6	0.0609909579667629\\
16.65	0.060978739002931\\
16.85	0.0610276148582584\\
17.05	0.0610031769305947\\
17.15	0.0610276148582584\\
17.35	0.0610031769305947\\
17.7	0.0610642717497541\\
17.9	0.0611070381231657\\
18.1	0.0610459433040063\\
18.45	0.0611070381231656\\
18.5	0.0611131476050815\\
18.6	0.0611009286412497\\
19	0.0611131476050815\\
19.2	0.0610642717497541\\
19.45	0.0609848484848469\\
19.6	0.0610459433040062\\
19.8	0.0609970674486787\\
20.1	0.0609848484848468\\
20.3	0.061064271749754\\
20.45	0.061058162267838\\
20.6	0.0611253665689133\\
20.65	0.0611131476050814\\
21.05	0.0607710166177893\\
21.4	0.0598423753665675\\
21.75	0.0586326979472128\\
22.15	0.0567876344086011\\
22.5	0.0551014173998038\\
22.9	0.0531280547409574\\
23.25	0.0513685239491688\\
23.6	0.0497372922776147\\
24	0.0478372434017596\\
24.35	0.0462487781036171\\
24.7	0.0446603128054746\\
25.1	0.0428824535679381\\
25.45	0.0412084555229725\\
25.8	0.0396016617790822\\
26.2	0.0383980938416432\\
26.55	0.0378665689149571\\
26.75	0.037701612903227\\
26.95	0.0377260508308907\\
27.25	0.0377688172043022\\
27.3	0.0377627077223862\\
27.55	0.0378421309872933\\
27.65	0.0378238025415455\\
28.05	0.0379276637341163\\
28.4	0.0379032258064526\\
28.5	0.0378910068426207\\
28.75	0.0379765395894437\\
29.1	0.0380987292277624\\
29.15	0.0380987292277624\\
29.2	0.0380681818181827\\
29.5	0.0380865102639305\\
29.75	0.0380254154447712\\
30.15	0.0380070869990234\\
30.25	0.0380315249266872\\
30.4	0.0379887585532757\\
30.65	0.0380193059628553\\
30.75	0.0379826490713597\\
31	0.0379887585532757\\
31.25	0.0379215542522005\\
31.4	0.0379459921798642\\
31.65	0.0378543499511252\\
31.8	0.0379093352883686\\
32.05	0.0378115835777137\\
32.1	0.0378115835777137\\
32.45	0.0377382697947226\\
32.5	0.0377443792766385\\
32.8	0.0375855327468244\\
32.85	0.0375916422287403\\
33	0.0376344086021518\\
33.35	0.0376344086021518\\
33.55	0.0375794232649085\\
33.75	0.0375244379276651\\
33.95	0.0375305474095811\\
34.1	0.0375000000000014\\
34.4	0.0375061094819173\\
34.5	0.0375427663734129\\
34.7	0.0375305474095811\\
34.95	0.0374633431085059\\
35.05	0.0375000000000014\\
35.25	0.0375488758553288\\
35.65	0.0375000000000014\\
35.75	0.0375672043010766\\
36.15	0.0396810850439893\\
36.5	0.0480205278592387\\
36.9	0.0617790811339214\\
37.25	0.0750610948191607\\
37.6	0.0888074291300111\\
38	0.104881476050832\\
38.35	0.118884408602151\\
38.7	0.132985092864125\\
39.1	0.149120234604105\\
39.45	0.163257575757575\\
39.85	0.179368279569892\\
40.2	0.193413978494622\\
40.55	0.20740469208211\\
40.95	0.222947214076245\\
41.3	0.232019794721406\\
41.65	0.234915689149558\\
42.05	0.236149804496577\\
42.15	0.236204789833821\\
42.4	0.236021505376343\\
42.8	0.235612170087976\\
43.15	0.234640762463343\\
43.5	0.233760997067448\\
43.9	0.232227517106549\\
44.25	0.230712365591399\\
44.6	0.229032258064518\\
45	0.227339931573805\\
45.35	0.225733137829915\\
45.75	0.223662023460413\\
46.1	0.221963587487782\\
46.45	0.220259042033236\\
46.85	0.218566715542521\\
47.2	0.217106549364612\\
47.55	0.215609726295207\\
47.95	0.21410068426197\\
48.3	0.212689393939388\\
48.65	0.211644672531763\\
49.05	0.210575513196474\\
49.4	0.209781280547401\\
49.8	0.20889540566959\\
50.15	0.208174486803508\\
50.5	0.20756964809383\\
50.65	0.207319159335277\\
50.9	0.207642961876823\\
51.25	0.21203567937438\\
51.6	0.220155180840657\\
52	0.231805962854343\\
52.35	0.242882453567931\\
52.75	0.255694037145645\\
53.1	0.267149315738021\\
53.45	0.278445747800582\\
53.85	0.291477272727269\\
54.2	0.302840909090905\\
54.55	0.313892961876828\\
54.95	0.326460166177904\\
55.3	0.337255620723358\\
55.7	0.349480694037141\\
56.05	0.357887341153466\\
56.4	0.36201124144672\\
56.8	0.363098729227755\\
56.85	0.363159824046915\\
57.15	0.362646627565976\\
57.5	0.362072336265879\\
57.9	0.360532746823063\\
58.25	0.359494134897356\\
58.6	0.358644916911045\\
59	0.357380254154449\\
59.35	0.356763196480942\\
59.75	0.356103372434024\\
60.1	0.355614613880752\\
60.3	0.355675708699913\\
60.45	0.355608504398838\\
60.75	0.355382453567949\\
60.85	0.355504643206268\\
61	0.355694037145663\\
61.35	0.355571847507347\\
61.45	0.355541300097767\\
61.7	0.355657380254171\\
61.85	0.355522971652022\\
61.95	0.355559628543518\\
62.3	0.355724584555251\\
62.7	0.355962854349975\\
63.05	0.35640884652984\\
63.4	0.356695992179888\\
63.65	0.356604349951149\\
63.8	0.356683773216056\\
63.9	0.356732649071384\\
64.15	0.356732649071384\\
64.45	0.357025904203351\\
64.7	0.357019794721434\\
64.9	0.356952590420358\\
65.1	0.357245845552322\\
65.25	0.357062561094843\\
65.35	0.356946480938441\\
65.65	0.357044232649096\\
66	0.353885630498555\\
66.35	0.344312072336286\\
66.75	0.328512952101683\\
67.1	0.310703812316734\\
67.45	0.291947702834815\\
67.85	0.26971529814273\\
68.2	0.249993890518095\\
68.6	0.227718719452599\\
68.95	0.20846774193549\\
69.3	0.188966275659828\\
69.7	0.16681329423265\\
70.05	0.147403470185729\\
70.4	0.128164711632453\\
70.8	0.106286656891495\\
71.15	0.093407869012708\\
71.5	0.0847690615835759\\
71.9	0.0813660801563998\\
72.25	0.0802908113391965\\
72.55	0.079905913978493\\
72.65	0.079905913978493\\
73	0.0800097751710638\\
73.35	0.0802113880742894\\
73.75	0.0805046432062539\\
74.1	0.0807001466275636\\
74.45	0.0807978983382185\\
74.5	0.0807856793743866\\
74.85	0.0810483870967715\\
75.2	0.0812377810361653\\
75.35	0.0812194525904175\\
75.6	0.0813782991202316\\
75.95	0.0815676930596254\\
76.3	0.0818365102639262\\
76.7	0.0820931085043952\\
77.05	0.0823863636363598\\
77.1	0.0823802541544438\\
77.4	0.082532991202342\\
77.8	0.0827162756598199\\
78.15	0.0829484359726251\\
78.55	0.0830889540566915\\
78.65	0.0830950635386074\\
78.9	0.0829545454545411\\
79.15	0.0829239980449614\\
79.25	0.0829912023460366\\
79.6	0.083125610948187\\
79.65	0.0831195014662711\\
79.75	0.0830645161290278\\
80	0.0831133919843552\\
80.35	0.0831622678396826\\
80.65	0.0832294721407578\\
80.75	0.0832050342130941\\
81.1	0.0850195503421258\\
81.45	0.0898582600195457\\
81.85	0.0993279569892451\\
82.2	0.107233626588463\\
82.6	0.116410068426197\\
82.95	0.124346285434996\\
83.3	0.13250244379277\\
83.7	0.141752199413496\\
84.05	0.149841153470194\\
84.4	0.15794843597264\\
84.8	0.167271505376358\\
85.15	0.175543743890534\\
85.55	0.184799608993176\\
85.9	0.192436461388096\\
86.25	0.197880009775195\\
86.5	0.199456256109507\\
86.65	0.199291300097777\\
87	0.199389051808433\\
87.35	0.199835043988298\\
87.75	0.200268817204331\\
88.1	0.200507086999053\\
88.5	0.200775904203353\\
88.85	0.201056940371487\\
89.2	0.201325757575788\\
89.5	0.201368523949199\\
89.65	0.201356304985367\\
89.95	0.201197458455553\\
90.3	0.20147238514177\\
90.5	0.201619012707752\\
90.7	0.201588465298173\\
90.85	0.201728983382239\\
91.1	0.201570136852425\\
91.4	0.20166788856308\\
91.5	0.201722873900323\\
91.75	0.201527370479013\\
91.8	0.201527370479013\\
92	0.201698435972659\\
92.15	0.201692326490743\\
92.45	0.201521260997097\\
92.75	0.201655669599248\\
92.9	0.201551808406677\\
93.2	0.20144183773219\\
93.25	0.201447947214106\\
93.65	0.201179130009805\\
94	0.201014173998075\\
94.15	0.201087487781066\\
94.35	0.200946969696999\\
94.5	0.200934750733168\\
94.75	0.201020283479991\\
94.8	0.201026392961906\\
95.1	0.200861436950176\\
95.45	0.200647605083119\\
95.5	0.200647605083119\\
95.85	0.20020772238517\\
95.9	0.200109970674515\\
96.15	0.200610948191621\\
96.2	0.200568181818209\\
96.6	0.197244623655938\\
96.95	0.191813294232672\\
97.3	0.185282258064535\\
97.7	0.177419354838725\\
98.05	0.170815004887596\\
98.45	0.162872678396879\\
98.8	0.156231671554255\\
99.15	0.149370723362658\\
99.55	0.141721652003906\\
99.9	0.134811827956981\\
100.25	0.12812194525903\\
100.65	0.120387341153454\\
101	0.113697458455506\\
101.4	0.106304985337225\\
101.75	0.104184995112393\\
102.1	0.103250244379254\\
102.5	0.103018084066449\\
102.7	0.103054740957944\\
102.85	0.102908113391962\\
102.95	0.102847018572803\\
103.5	0.102956989247289\\
103.6	0.102871456500466\\
103.95	0.102706500488737\\
104.3	0.102578201368507\\
104.7	0.102633186705756\\
104.75	0.102633186705757\\
105	0.102565982404685\\
105.2	0.102510997067445\\
105.45	0.102578201368524\\
105.6	0.10260263929619\\
105.75	0.102535434995117\\
105.8	0.102535434995117\\
106.15	0.102682062561104\\
106.55	0.102798142717513\\
106.7	0.102853128054758\\
106.9	0.102767595307938\\
107.05	0.102816471163268\\
107.2	0.102737047898363\\
107.25	0.102737047898363\\
107.65	0.10287145650052\\
107.7	0.102865347018604\\
107.95	0.102938660801599\\
108	0.102914222873936\\
108.4	0.102773704789875\\
108.5	0.102761485826045\\
108.75	0.102956989247358\\
108.8	0.102993646138854\\
109	0.102950879765444\\
109.15	0.102975317693108\\
109.5	0.102920332355865\\
109.7	0.10282258064521\\
109.85	0.102883675464369\\
109.95	0.102950879765444\\
110.2	0.102932551319697\\
110.6	0.102834799609042\\
110.85	0.102840909090958\\
110.95	0.102785923753714\\
111.3	0.102547653958993\\
111.4	0.102584310850489\\
111.7	0.102486559139834\\
111.75	0.102474340176002\\
112.05	0.102523216031329\\
112.2	0.102596529814321\\
112.45	0.102584310850489\\
112.65	0.102480449657918\\
112.8	0.102529325513245\\
113.15	0.102633186705816\\
113.2	0.102663734115396\\
113.55	0.102639296187732\\
113.8	0.102498778103666\\
113.9	0.102504887585582\\
114.25	0.102468230694086\\
114.35	0.102443792766422\\
114.65	0.102523216031329\\
114.75	0.102584310850489\\
115	0.102504887585582\\
115.35	0.102584310850489\\
115.4	0.102578201368573\\
115.75	0.102743157380303\\
116.1	0.102938660801613\\
116.3	0.103005865102688\\
116.45	0.102889784946285\\
116.65	0.102895894428201\\
116.85	0.102932551319697\\
117.2	0.103042521994183\\
117.5	0.102993646138856\\
117.65	0.103048631476099\\
117.95	0.102902003910117\\
118.15	0.102816471163294\\
118.35	0.102914222873949\\
118.7	0.103024193548436\\
119.05	0.103207478005913\\
119.4	0.103354105571896\\
119.45	0.103354105571896\\
119.8	0.103457966764466\\
120	0.103549608993205\\
120.25	0.103519061583626\\
120.55	0.103464076246382\\
120.7	0.103506842619794\\
120.85	0.103439638318719\\
120.9	0.10345185728255\\
121.2	0.10351295210171\\
121.5	0.103488514174046\\
121.65	0.103549608993205\\
122	0.103384652981475\\
122.25	0.103231915933577\\
122.45	0.103280791788904\\
122.75	0.103354105571896\\
122.95	0.103427419354887\\
123.1	0.103421309872971\\
123.4	0.103506842619794\\
123.55	0.103464076246382\\
123.85	0.103537390029373\\
124.25	0.103335777126148\\
124.4	0.103311339198484\\
124.6	0.1033174486804\\
124.75	0.103280791788904\\
124.95	0.103299120234652\\
125.2	0.103366324535727\\
125.55	0.103360215053812\\
125.7	0.10328690127082\\
125.8	0.103225806451661\\
126.1	0.103231915933577\\
126.45	0.10334799608998\\
126.6	0.103299120234652\\
126.8	0.103360215053812\\
127.15	0.103622922776196\\
127.45	0.103696236559188\\
127.55	0.103635141740028\\
127.9	0.103555718475121\\
128.05	0.103629032258112\\
128.3	0.10345185728255\\
128.65	0.10334799608998\\
129	0.103164711632502\\
129.25	0.103231915933577\\
129.4	0.103170821114418\\
129.75	0.102963098729276\\
130.1	0.102694281524975\\
130.5	0.102535434995161\\
130.85	0.102510997067498\\
131.25	0.102181085044037\\
131.6	0.102040566959971\\
131.95	0.101790078201418\\
132.35	0.101392961876883\\
132.7	0.101099706744918\\
133.05	0.100910312805524\\
133.45	0.100794232649122\\
133.8	0.100629276637392\\
134	0.100672043010803\\
134.2	0.100555962854401\\
134.55	0.100482649071409\\
134.75	0.100397116324586\\
135	0.100354349951175\\
135.2	0.100445992179914\\
135.3	0.100384897360755\\
135.65	0.100305474095848\\
136	0.100329912023511\\
136.1	0.100311583577763\\
136.3	0.100372678396923\\
136.4	0.100372678396923\\
136.55	0.100439882697998\\
136.75	0.100378787878839\\
136.95	0.100415444770334\\
137.2	0.100488758553325\\
137.4	0.100391006842671\\
137.5	0.100403225806502\\
137.85	0.100568181818232\\
137.95	0.100543743890569\\
138.25	0.100568181818232\\
138.45	0.100690371456551\\
138.65	0.100690371456551\\
138.95	0.100568181818232\\
139.2	0.100574291300148\\
139.35	0.100525415444821\\
139.6	0.100635386119308\\
139.75	0.10064760508314\\
140.05	0.100537634408653\\
140.4	0.100439882697998\\
140.45	0.100445992179914\\
140.8	0.100219941349025\\
141.2	0.0978739002933018\\
141.55	0.0936827956989653\\
141.9	0.0884836265885006\\
142.3	0.0819342619746145\\
142.65	0.0759958455523226\\
143	0.0697214076246546\\
143.4	0.062952101661795\\
143.75	0.0574108015640384\\
144.15	0.0512463343108557\\
144.5	0.045796676441838\\
144.85	0.0404447702834752\\
145.25	0.0343291788856199\\
145.6	0.0289283968719296\\
145.95	0.024065249266841\\
146.35	0.0213037634408408\\
146.7	0.0205767350928454\\
147.05	0.0207416911045755\\
147.1	0.0207294721407436\\
147.45	0.0211937927663541\\
147.8	0.0219147116324336\\
148.2	0.0227272727272521\\
148.55	0.0229533235581414\\
148.9	0.0231366080156192\\
149.05	0.0230999511241237\\
149.35	0.0231304985337033\\
149.6	0.023069403714544\\
149.65	0.02307551319646\\
149.8	0.0231488269794511\\
150.15	0.0231182795698714\\
150.4	0.0232099217986104\\
150.65	0.0232832355816015\\
150.85	0.0232710166177696\\
150.95	0.0233015640273493\\
151.15	0.0232710166177696\\
151.5	0.0233015640273493\\
151.85	0.0231427174975352\\
151.9	0.0231121700879555\\
152.25	0.0232160312805263\\
152.3	0.0232343597262741\\
152.6	0.0231304985337033\\
152.95	0.0230510752687962\\
153.3	0.0230877321602918\\
153.35	0.0230816226783759\\
153.5	0.0231243890517874\\
153.8	0.0231182795698714\\
154.1	0.0230571847507122\\
154.15	0.0230388563049644\\
154.4	0.023069403714544\\
154.45	0.023069403714544\\
154.8	0.0229960899315529\\
155.05	0.0229411045943095\\
155.2	0.0229716520038892\\
155.4	0.022904447702814\\
155.6	0.0229105571847299\\
155.85	0.0229594330400573\\
155.9	0.0229594330400573\\
156.3	0.022989980449637\\
156.45	0.022983870967721\\
156.65	0.0230388563049644\\
156.95	0.0230938416422077\\
157.3	0.02307551319646\\
157.4	0.0231121700879555\\
157.7	0.0232038123166944\\
157.8	0.0231915933528626\\
158	0.02324046920819\\
158.15	0.02324046920819\\
158.5	0.0231732649071148\\
158.7	0.0232465786901059\\
158.85	0.0232221407624422\\
159.2	0.0233015640273493\\
159.25	0.0233015640273493\\
159.55	0.0233748778103404\\
159.95	0.023411534701836\\
};
\addlegendentry{C16};

\end{axis}
\end{tikzpicture}%}
%\tikzset{pressure/.style={draw, circle, inner sep=0pt, text width=5mm, align=center}}
\tikzset{difpres/.style={draw, circle, inner sep=0pt, text width=6mm, align=center}}
\tikzset{connect/.style={draw,circle, inner sep=0pt, text width=2mm, align=center,fill=black}}
\tikzset{evalve/.style={draw, circle, inner sep=0pt, text width=3mm, align=center}}
\begin{turn}{90}
\begin{tikzpicture}



%Pump north
\node[draw,circle,minimum size=1cm] (p0) at (6,6) {};
\node(p1) at ($(p0)+(-0.5,0)$) {};
\node(p2) at ($(p1)+(0.5,0.5)$) {};
\node(p3) at ($(p1)+(1,0)$) {};
\draw(p1.center) -- (p2.center) -- (p3.center);
\node at ($(p1)+(1.5,0)$) {\Large $C_{18}$};

%Pump north
\node[draw,circle,minimum size=1cm] (p0) at (-5,9) {};
\node(p1) at ($(p0)+(-0.5,0)$) {};
\node(p2) at ($(p1)+(0.5,0.5)$) {};
\node(p3) at ($(p1)+(1,0)$) {};
\draw(p1.center) -- (p2.center) -- (p3.center);
\node at ($(p1)+(1.5,0)$) {\Large $C_{32}$};

%Pump north
\node[draw,circle,minimum size=1cm] (p0) at (20,6) {};
\node(p1) at ($(p0)+(-0.5,0)$) {};
\node(p2) at ($(p1)+(0.5,0.5)$) {};
\node(p3) at ($(p1)+(1,0)$) {};
\draw(p1.center) -- (p2.center) -- (p3.center);
\node at ($(p1)+(1.5,0)$) {\Large $C_{25}$};

%Pump north
\node[draw,circle,minimum size=1cm] (p0) at (-1.5,0.5) {};
\node(p1) at ($(p0)+(-0.5,0)$) {};
\node(p2) at ($(p1)+(0.5,0.5)$) {};
\node(p3) at ($(p1)+(1,0)$) {};
\draw(p1.center) -- (p2.center) -- (p3.center);
\node at ($(p1)+(1.5,0)$) {\Large $C_{2}$};

%Pump north
\node[draw,circle,minimum size=1cm] (p0) at (28,0.5) {};
\node(p1) at ($(p0)+(-0.5,0)$) {};
\node(p2) at ($(p1)+(0.5,0.5)$) {};
\node(p3) at ($(p1)+(1,0)$) {};
\draw(p1.center) -- (p2.center) -- (p3.center);
\node at ($(p1)+(1.5,0)$) {\Large $C_{16}$};

%man-valve
\node(n1) at (6.25,5) {};
\draw(n1.center) -- ($(n1)-(0.5,0)$) --
($(n1)-(0,1)$) -- ($(n1)-(0.5,1)$) --  (n1.center);
\draw($(n1)-(0.75,0.25)$) -- ($(n1)-(0.75,0.75)$) -- 
($(n1)-(0.75,0.5)$) --  ($(n1)-(0.25,0.5)$);
%\node at ($(n1)+(0.5,-0.5)$) {\Large $C_{18d}$};

%man-valve
\node(n1) at (6.25,8) {};
\draw(n1.center) -- ($(n1)-(0.5,0)$) --
($(n1)-(0,1)$) -- ($(n1)-(0.5,1)$) --  (n1.center);
\draw($(n1)-(0.75,0.25)$) -- ($(n1)-(0.75,0.75)$) -- 
($(n1)-(0.75,0.5)$) --  ($(n1)-(0.25,0.5)$);
%\node at ($(n1)+(0.5,-0.5)$) {\Large $C_{18u}$};

%man-valve
\node(n1) at (20.25,8) {};
\draw(n1.center) -- ($(n1)-(0.5,0)$) --
($(n1)-(0,1)$) -- ($(n1)-(0.5,1)$) --  (n1.center);
\draw($(n1)-(0.75,0.25)$) -- ($(n1)-(0.75,0.75)$) -- 
($(n1)-(0.75,0.5)$) --  ($(n1)-(0.25,0.5)$);
%\node at ($(n1)+(0.5,-0.5)$) {\Large $C_{25u}$};

%man-valve
\node(n1) at (20.25,5) {};
\draw(n1.center) -- ($(n1)-(0.5,0)$) --
($(n1)-(0,1)$) -- ($(n1)-(0.5,1)$) --  (n1.center);
\draw($(n1)-(0.75,0.25)$) -- ($(n1)-(0.75,0.75)$) -- 
($(n1)-(0.75,0.5)$) --  ($(n1)-(0.25,0.5)$);
%\node at ($(n1)+(0.5,-0.5)$) {\Large $C_{25d}$};

%man-valve
\node(n1) at (-1.25,2.5) {};
\draw(n1.center) -- ($(n1)-(0.5,0)$) --
($(n1)-(0,1)$) -- ($(n1)-(0.5,1)$) --  (n1.center);
\draw($(n1)-(0.75,0.25)$) -- ($(n1)-(0.75,0.75)$) -- 
($(n1)-(0.75,0.5)$) --  ($(n1)-(0.25,0.5)$);
%\node at ($(n1)+(0.4,-0.5)$) {\Large $C_{3}$};

%man-valve
\node(n1) at (-1.25,-0.5) {};
\draw(n1.center) -- ($(n1)-(0.5,0)$) --
($(n1)-(0,1)$) -- ($(n1)-(0.5,1)$) --  (n1.center);
\draw($(n1)-(0.75,0.25)$) -- ($(n1)-(0.75,0.75)$) -- 
($(n1)-(0.75,0.5)$) --  ($(n1)-(0.25,0.5)$);
%\node at ($(n1)+(0.4,-0.5)$) {\Large $C_{1}$};

%man-valve
\node(n1) at (-4.75,11.5) {};
\draw(n1.center) -- ($(n1)-(0.5,0)$) --
($(n1)-(0,1)$) -- ($(n1)-(0.5,1)$) --  (n1.center);
\draw($(n1)-(0.75,0.25)$) -- ($(n1)-(0.75,0.75)$) -- 
($(n1)-(0.75,0.5)$) --  ($(n1)-(0.25,0.5)$);
%\node at ($(n1)+(0.5,-0.5)$) {\Large $C_{6}$};

%man-valve
\node(n1) at (-4.75,7.5) {};
\draw(n1.center) -- ($(n1)-(0.5,0)$) --
($(n1)-(0,1)$) -- ($(n1)-(0.5,1)$) --  (n1.center);
\draw($(n1)-(0.75,0.25)$) -- ($(n1)-(0.75,0.75)$) -- 
($(n1)-(0.75,0.5)$) --  ($(n1)-(0.25,0.5)$);
%\node at ($(n1)+(0.5,-0.5)$) {\Large $C_{5}$};

%man-valve
\node(n1) at (28.25,-0.5) {};
\draw(n1.center) -- ($(n1)-(0.5,0)$) --
($(n1)-(0,1)$) -- ($(n1)-(0.5,1)$) --  (n1.center);
\draw($(n1)-(0.75,0.25)$) -- ($(n1)-(0.75,0.75)$) -- 
($(n1)-(0.75,0.5)$) --  ($(n1)-(0.25,0.5)$);
%\node at ($(n1)+(0.5,-0.5)$) {\Large $C_{17}$};

%man-valve
\node(n1) at (28.25,2.5) {};
\draw(n1.center) -- ($(n1)-(0.5,0)$) --
($(n1)-(0,1)$) -- ($(n1)-(0.5,1)$) --  (n1.center);
\draw($(n1)-(0.75,0.25)$) -- ($(n1)-(0.75,0.75)$) -- 
($(n1)-(0.75,0.5)$) --  ($(n1)-(0.25,0.5)$);
%\node at ($(n1)+(0.5,-0.5)$) {\Large $C_{15}$};

%elec-valve
\node(n1) at (10.25,11.5) {};
\draw(n1.center) -- ($(n1)-(0.5,0)$) --
($(n1)-(0,1)$) -- ($(n1)-(0.5,1)$) --  (n1.center);
\draw($(n1)-(0.75,0.5)$) circle (1.5mm); 
\draw($(n1)-(0.6,0.5)$)--  ($(n1)-(0.25,0.5)$);
\node at ($(n1)+(0.3,-0.5)$) {\Large $C_{24}$};

%elec-valve
\node(n1) at (-1.25,11.5) {};
\draw(n1.center) -- ($(n1)-(0.5,0)$) --
($(n1)-(0,1)$) -- ($(n1)-(0.5,1)$) --  (n1.center);
\draw($(n1)-(0.75,0.5)$) circle (1.5mm); 
\draw($(n1)-(0.6,0.5)$)--  ($(n1)-(0.25,0.5)$);
\node at ($(n1)+(0.3,-0.5)$) {\Large $C_{20}$};

%elec-valve
\node(n1) at (24.25,11.5) {};
\draw(n1.center) -- ($(n1)-(0.5,0)$) --
($(n1)-(0,1)$) -- ($(n1)-(0.5,1)$) --  (n1.center);
\draw($(n1)-(0.75,0.5)$) circle (1.5mm); 
\draw($(n1)-(0.6,0.5)$)--  ($(n1)-(0.25,0.5)$);
\node at ($(n1)+(0.3,-0.5)$) {\Large $C_{31}$};

%elec-valve
\node(n1) at (13.25,11.5) {};
\draw(n1.center) -- ($(n1)-(0.5,0)$) --
($(n1)-(0,1)$) -- ($(n1)-(0.5,1)$) --  (n1.center);
\draw($(n1)-(0.75,0.5)$) circle (1.5mm); 
\draw($(n1)-(0.6,0.5)$)--  ($(n1)-(0.25,0.5)$);
\node at ($(n1)+(0.3,-0.5)$) {\Large $C_{27}$};

%man-valve
%\draw[very thick](-0.25,-1) -- (0.25,-1) -- (-0.25,-2) -- (0.25,-2) -- (-0.25,-1) -- (0.25,-1);
%\draw[very thick](0,-1.5) -- (-0.5,-1.5);
%\draw[very thick](-0.5,-1.25) -- (-0.5,-1.75);


%GND
\node(g1) at (-1.5,-2.5) {};
\draw(g1.center) -- ($(g1)+(0.3,0)$) -- ($(g1)-(0.3,0)$);
\draw($(g1)-(0.15,0.1)$) -- ($(g1)+(0.15,-0.1)$);

%GND
\node(g1) at (-1.5,9.5) {};
\draw(g1.center) -- ($(g1)+(0.3,0)$) -- ($(g1)-(0.3,0)$);
\draw($(g1)-(0.15,0.1)$) -- ($(g1)+(0.15,-0.1)$);

%GND
\node(g1) at (28,-2.5) {};
\draw(g1.center) -- ($(g1)+(0.3,0)$) -- ($(g1)-(0.3,0)$);
\draw($(g1)-(0.15,0.1)$) -- ($(g1)+(0.15,-0.1)$);

%GND
\node(g1) at (13,9.5) {};
\draw(g1.center) -- ($(g1)+(0.3,0)$) -- ($(g1)-(0.3,0)$);
\draw($(g1)-(0.15,0.1)$) -- ($(g1)+(0.15,-0.1)$);

%GND
\node(g1) at (24,9.5) {};
\draw(g1.center) -- ($(g1)+(0.3,0)$) -- ($(g1)-(0.3,0)$);
\draw($(g1)-(0.15,0.1)$) -- ($(g1)+(0.15,-0.1)$);

%GND
\node(g1) at (10,9.5) {};
\draw(g1.center) -- ($(g1)+(0.3,0)$) -- ($(g1)-(0.3,0)$);
\draw($(g1)-(0.15,0.1)$) -- ($(g1)+(0.15,-0.1)$);


%pipe
\node(r1) at (12,3) {};
\draw (r1.center) -- ($(r1)+(1.5,0)$) arc (90:-90:0.05) -- 
($(r1)+(0,-0.1)$) arc (90:270:0.05) --
($(r1)+(1.5,-0.2)$) arc (90:-90:0.05) --
($(r1)+(0,-0.3)$) arc (90:270:0.05) --
($(r1)+(1.5,-0.4)$) arc (90:-90:0.05) --
($(r1)+(0,-0.5)$) arc (90:270:0.05);
\draw[rounded corners] ($(r1)+(0,-0.6)$) -- 
($(r1)+(1.65,-0.6)$) -- ($(r1)+(1.65,0)$) -- ($(r1)+(2.5,0)$);
\node at ($(r1)+(0.75,-1)$) {\Large $C_{9},C_{10}$};

%pipe
\node(r1) at (12,-1) {};
\draw (r1.center) -- ($(r1)+(1.5,0)$) arc (90:-90:0.05) -- 
($(r1)+(0,-0.1)$) arc (90:270:0.05) --
($(r1)+(1.5,-0.2)$) arc (90:-90:0.05) --
($(r1)+(0,-0.3)$) arc (90:270:0.05) --
($(r1)+(1.5,-0.4)$) arc (90:-90:0.05) --
($(r1)+(0,-0.5)$) arc (90:270:0.05);
\draw[rounded corners] ($(r1)+(0,-0.6)$) -- 
($(r1)+(1.65,-0.6)$) -- ($(r1)+(1.65,0)$) -- ($(r1)+(2.5,0)$);
\node at ($(r1)+(0.75,-1)$) {\Large $C_{12},C_{13}$};

%pipe
\node(r1) at (12,16.5) {};
\draw (r1.center) -- ($(r1)+(1.5,0)$) arc (90:-90:0.05) -- 
($(r1)+(0,-0.1)$) arc (90:270:0.05) --
($(r1)+(1.5,-0.2)$) arc (90:-90:0.05) --
($(r1)+(0,-0.3)$) arc (90:270:0.05) --
($(r1)+(1.5,-0.4)$) arc (90:-90:0.05) --
($(r1)+(0,-0.5)$) arc (90:270:0.05);
\draw[rounded corners] ($(r1)+(0,-0.6)$) -- 
($(r1)+(1.65,-0.6)$) -- ($(r1)+(1.65,0)$) -- ($(r1)+(2.5,0)$);
\node at ($(r1)+(0.75,-1)$) {\Large $C_{42}$};

%pipe
\node(r1) at (25,3) {};
\draw (r1.center) -- ($(r1)+(1.5,0)$) arc (90:-90:0.05) -- 
($(r1)+(0,-0.1)$) arc (90:270:0.05) --
($(r1)+(1.5,-0.2)$) arc (90:-90:0.05) --
($(r1)+(0,-0.3)$) arc (90:270:0.05) --
($(r1)+(1.5,-0.4)$) arc (90:-90:0.05) --
($(r1)+(0,-0.5)$) arc (90:270:0.05);
\draw[rounded corners] ($(r1)+(0,-0.6)$) -- 
($(r1)+(1.65,-0.6)$) -- ($(r1)+(1.65,0)$) -- ($(r1)+(2.5,0)$);
\node at ($(r1)+(0.75,-1)$) {\Large $C_{14}$};

%pipe
\node(r1) at (21.5,3) {};
\draw (r1.center) -- ($(r1)+(1.5,0)$) arc (90:-90:0.05) -- 
($(r1)+(0,-0.1)$) arc (90:270:0.05) --
($(r1)+(1.5,-0.2)$) arc (90:-90:0.05) --
($(r1)+(0,-0.3)$) arc (90:270:0.05) --
($(r1)+(1.5,-0.4)$) arc (90:-90:0.05) --
($(r1)+(0,-0.5)$) arc (90:270:0.05);
\draw[rounded corners] ($(r1)+(0,-0.6)$) -- 
($(r1)+(1.65,-0.6)$) -- ($(r1)+(1.65,0)$) -- ($(r1)+(2.5,0)$);
\node at ($(r1)+(0.75,-1)$) {\Large $C_{11}$};

%pipe
\node(r1) at (6,10) {};
\begin{scope} [rotate around={90:(r1)}]
\draw (r1.center) -- ($(r1)+(1.5,0)$) arc (90:-90:0.05) -- 
($(r1)+(0,-0.1)$) arc (90:270:0.05) --
($(r1)+(1.5,-0.2)$) arc (90:-90:0.05) --
($(r1)+(0,-0.3)$) arc (90:270:0.05) --
($(r1)+(1.5,-0.4)$) arc (90:-90:0.05) --
($(r1)+(0,-0.5)$) arc (90:270:0.05);
\draw[rounded corners] ($(r1)+(0,-0.6)$) -- 
($(r1)+(1.65,-0.6)$) -- ($(r1)+(1.65,0)$) -- ($(r1)+(2.5,0)$);
\end{scope}
\node at ($(r1)+(1.25,0.75)$) {\Large $C_{19}$};

%pipe
\node(r1) at (3,13) {};
\draw (r1.center) -- ($(r1)+(1.5,0)$) arc (90:-90:0.05) -- 
($(r1)+(0,-0.1)$) arc (90:270:0.05) --
($(r1)+(1.5,-0.2)$) arc (90:-90:0.05) --
($(r1)+(0,-0.3)$) arc (90:270:0.05) --
($(r1)+(1.5,-0.4)$) arc (90:-90:0.05) --
($(r1)+(0,-0.5)$) arc (90:270:0.05);
\draw[rounded corners] ($(r1)+(0,-0.6)$) -- 
($(r1)+(1.65,-0.6)$) -- ($(r1)+(1.65,0)$) -- ($(r1)+(2.5,0)$);
\node at ($(r1)+(0.75,-1)$) {\Large $C_{22}$};

%pipe
\node(r1) at (7,13) {};
\draw (r1.center) -- ($(r1)+(1.5,0)$) arc (90:-90:0.05) -- 
($(r1)+(0,-0.1)$) arc (90:270:0.05) --
($(r1)+(1.5,-0.2)$) arc (90:-90:0.05) --
($(r1)+(0,-0.3)$) arc (90:270:0.05) --
($(r1)+(1.5,-0.4)$) arc (90:-90:0.05) --
($(r1)+(0,-0.5)$) arc (90:270:0.05);
\draw[rounded corners] ($(r1)+(0,-0.6)$) -- 
($(r1)+(1.65,-0.6)$) -- ($(r1)+(1.65,0)$) -- ($(r1)+(2.5,0)$);
\node at ($(r1)+(0.75,-1)$) {\Large $C_{23}$};

%pipe
\node(r1) at (3,3) {};
\draw (r1.center) -- ($(r1)+(1.5,0)$) arc (90:-90:0.05) -- 
($(r1)+(0,-0.1)$) arc (90:270:0.05) --
($(r1)+(1.5,-0.2)$) arc (90:-90:0.05) --
($(r1)+(0,-0.3)$) arc (90:270:0.05) --
($(r1)+(1.5,-0.4)$) arc (90:-90:0.05) --
($(r1)+(0,-0.5)$) arc (90:270:0.05);
\draw[rounded corners] ($(r1)+(0,-0.6)$) -- 
($(r1)+(1.65,-0.6)$) -- ($(r1)+(1.65,0)$) -- ($(r1)+(2.5,0)$);
\node at ($(r1)+(0.75,-1)$) {\Large $C_{8}$};

%pipe
\node(r1) at (0,13) {};
\draw (r1.center) -- ($(r1)+(1.5,0)$) arc (90:-90:0.05) -- 
($(r1)+(0,-0.1)$) arc (90:270:0.05) --
($(r1)+(1.5,-0.2)$) arc (90:-90:0.05) --
($(r1)+(0,-0.3)$) arc (90:270:0.05) --
($(r1)+(1.5,-0.4)$) arc (90:-90:0.05) --
($(r1)+(0,-0.5)$) arc (90:270:0.05);
\draw[rounded corners] ($(r1)+(0,-0.6)$) -- 
($(r1)+(1.65,-0.6)$) -- ($(r1)+(1.65,0)$) -- ($(r1)+(2.5,0)$);
\node at ($(r1)+(0.75,-1)$) {\Large $C_{21}$};

%pipe
\node(r1) at (20,10) {};
\begin{scope} [rotate around={90:(r1)}]
\draw (r1.center) -- ($(r1)+(1.5,0)$) arc (90:-90:0.05) -- 
($(r1)+(0,-0.1)$) arc (90:270:0.05) --
($(r1)+(1.5,-0.2)$) arc (90:-90:0.05) --
($(r1)+(0,-0.3)$) arc (90:270:0.05) --
($(r1)+(1.5,-0.4)$) arc (90:-90:0.05) --
($(r1)+(0,-0.5)$) arc (90:270:0.05);
\draw[rounded corners] ($(r1)+(0,-0.6)$) -- 
($(r1)+(1.65,-0.6)$) -- ($(r1)+(1.65,0)$) -- ($(r1)+(2.5,0)$);
\end{scope}
\node at ($(r1)+(1.25,0.75)$) {\Large $C_{26}$};

%pipe
\node(r1) at (14,13) {};
\draw (r1.center) -- ($(r1)+(1.5,0)$) arc (90:-90:0.05) -- 
($(r1)+(0,-0.1)$) arc (90:270:0.05) --
($(r1)+(1.5,-0.2)$) arc (90:-90:0.05) --
($(r1)+(0,-0.3)$) arc (90:270:0.05) --
($(r1)+(1.5,-0.4)$) arc (90:-90:0.05) --
($(r1)+(0,-0.5)$) arc (90:270:0.05);
\draw[rounded corners] ($(r1)+(0,-0.6)$) -- 
($(r1)+(1.65,-0.6)$) -- ($(r1)+(1.65,0)$) -- ($(r1)+(2.5,0)$);
\node at ($(r1)+(0.75,-1)$) {\Large $C_{28}$};

%pipe
\node(r1) at (17,13) {};
\draw (r1.center) -- ($(r1)+(1.5,0)$) arc (90:-90:0.05) -- 
($(r1)+(0,-0.1)$) arc (90:270:0.05) --
($(r1)+(1.5,-0.2)$) arc (90:-90:0.05) --
($(r1)+(0,-0.3)$) arc (90:270:0.05) --
($(r1)+(1.5,-0.4)$) arc (90:-90:0.05) --
($(r1)+(0,-0.5)$) arc (90:270:0.05);
\draw[rounded corners] ($(r1)+(0,-0.6)$) -- 
($(r1)+(1.65,-0.6)$) -- ($(r1)+(1.65,0)$) -- ($(r1)+(2.5,0)$);
\node at ($(r1)+(0.75,-1)$) {\Large $C_{29}$};

%pipe
\node(r1) at (21,13) {};
\draw (r1.center) -- ($(r1)+(1.5,0)$) arc (90:-90:0.05) -- 
($(r1)+(0,-0.1)$) arc (90:270:0.05) --
($(r1)+(1.5,-0.2)$) arc (90:-90:0.05) --
($(r1)+(0,-0.3)$) arc (90:270:0.05) --
($(r1)+(1.5,-0.4)$) arc (90:-90:0.05) --
($(r1)+(0,-0.5)$) arc (90:270:0.05);
\draw[rounded corners] ($(r1)+(0,-0.6)$) -- 
($(r1)+(1.65,-0.6)$) -- ($(r1)+(1.65,0)$) -- ($(r1)+(2.5,0)$);
\node at ($(r1)+(0.75,-1)$) {\Large $C_{30}$};

%pipe
\node(r1) at (-0.5,3) {};
\draw (r1.center) -- ($(r1)+(1.5,0)$) arc (90:-90:0.05) -- 
($(r1)+(0,-0.1)$) arc (90:270:0.05) --
($(r1)+(1.5,-0.2)$) arc (90:-90:0.05) --
($(r1)+(0,-0.3)$) arc (90:270:0.05) --
($(r1)+(1.5,-0.4)$) arc (90:-90:0.05) --
($(r1)+(0,-0.5)$) arc (90:270:0.05);
\draw[rounded corners] ($(r1)+(0,-0.6)$) -- 
($(r1)+(1.65,-0.6)$) -- ($(r1)+(1.65,0)$) -- ($(r1)+(2.5,0)$);
\node at ($(r1)+(0.75,-1)$) {\Large $C_{4}$};

%pressure sensor
\node (PD1) at (-1.5,3) {};
\node(P1) at ($(PD1)+(0,1)$) [pressure] {P};
\draw(PD1.center) -- (P1);

%pressure sensor
\node (PD1) at (23.5,13) {};
\node(P1) at ($(PD1)+(0,1)$) [pressure] {P};
\draw(PD1.center) -- (P1);

%pressure sensor
\node (PD1) at (9.5,13) {};
\node(P1) at ($(PD1)+(0,1)$) [pressure] {P};
\draw(PD1.center) -- (P1);

%pressure sensor
\node (PD1) at (28,3) {};
\node(P1) at ($(PD1)+(0,1)$) [pressure] {P};
\draw(PD1.center) -- (P1);

%pressure sensor
\node (PD1) at (16.5,13) {};
\node(P1) at ($(PD1)+(0,1)$) [pressure] {P};
\draw(PD1.center) -- (P1);

%pressure sensor
\node (PD1) at (2.5,13) {};
\node(P1) at ($(PD1)+(0,1)$) [pressure] {P};
\draw(PD1.center) -- (P1);

%pressure sensor vert
\node (PD1) at (6,3.5) {};
\node(P1) at ($(PD1)+(1,0)$) [pressure] {P};
\draw(PD1.center) -- (P1);

%pressure sensor vert
\node(PD1) at (20,3.5) {};
\node(P1) at ($(PD1)+(1,0)$) [pressure] {P};
\draw(PD1.center) -- (P1);

%differential pressure sensor
\node(CDP1) at (-1.5,3) {};
\node(CDP2) at (-1.5,-2) {};
\node[difpres] (DP1) at (-3,0.5) {DP};
\draw(CDP1.center) -| (DP1)  |- (CDP2.center);

%differential pressure sensor
\node(CDP1) at (20,8.5) {};
\node (CDP2) at (20,3.5) {};
\node[difpres] (DP1) at (18.5,6) {DP};
\draw(CDP1.center) -| (DP1)  |- (CDP2.center);

%differential pressure sensor
\node(CDP1) at (6,8.5) {};
\node(CDP2) at (6,3.5) {};
\node[difpres] (DP1) at (4.5,6) {DP};
\draw(CDP1.center) -| (DP1)  |- (CDP2.center);

%differential pressure sensor
\node(CDP1) at (28,3) {};
\node(CDP2) at (28,-2) {};
\node[difpres] (DP1) at (30,0.5) {DP};
\draw(CDP1.center) -| (DP1)  |- (CDP2.center);

%Connections staight lines
\draw (-1.5,3) node (v1) {} -- (-1.5,2.5);
\draw (-1.5,1.5) -- (-1.5,1);
\draw (-1.5,0) -- (-1.5,-0.5);
\draw (-1.5,-1.5) -- (-1.5,-2.5);
\draw (-0.5,3) -- (-1.5,3);
\draw (3,3) -- (2,3);
\draw(5.5,3) -- (12,3);
\draw(6,3) -- (6,4);
\draw(6,5) -- (6,5.5);
\draw(6,6.5) -- (6,7);
\draw(6,8) -- (6,10);
\draw(5.5,13) -- (7,13);
\draw(2.5,13) -- (3,13);
\draw(-1.5,10.5) -- (-1.5,9.5);
\draw(10,10.5) -- (10,9.5);
\draw(13,10.5) -- (13,9.5);
\draw(24,10.5) -- (24,9.5);
\draw(20,10) -- (20,8);
\draw(20,7) -- (20,6.5);
\draw(20,5.5) -- (20,5);
\draw(20,4) -- (20,3);
\draw(14.5,3) -- (21.5,3);
\draw(24,3) -- (25,3);
\draw(27.5,3) -- (28,3);
\draw(28,3) -- (28,2.5);
\draw(28,3) -- (28,2.5);
\draw(28,1.5) -- (28,1);
\draw(28,0) -- (28,-0.5);
\draw(28,-1.5) -- (28,-2.5);
\draw(19.5,13) -- (21,13);
\draw(-5,8.5) -- (-5,7.5);
\draw(-5,11.5) -- (-5,12.5);
\draw(-6.5,14.5) -- (-6.5,12.5) -- (-3.5,12.5) -- (-3.5,14.5);
\draw(16.5,13) -- (17,13);
\draw(-5,9.5) -- (-5,10.5);
%Connections bend lines
\draw[rounded corners](24,3) |- (14.5,-1);
\draw[rounded corners](9.5,13) -| (10,11.5);
\draw[rounded corners](23.5,13) -| (24,11.5);
\draw[rounded corners](14.5,16.5) -| (20,12.5);
\draw[rounded corners](12,16.5) -| (6,12.5);
\draw[rounded corners](0,13) -| (-1.5,11.5);
\draw[rounded corners](14,13) -| (13,11.5);
\draw[rounded corners](-5,6.5) |- (-5,5) -| (2,3);
\draw[rounded corners](2,3) |- (12,-1);
\draw (-6.5,14) .. controls (-5,13.5) and (-5,14.5) .. (-3.5,14);
%nodes
\node[connect] (N) at (-1.5,-2) {};
\node at ($(N)+(0.5,0)$) {\Large $n_{1}$};
\node[connect] (N) at (28,-2) {};
\node at ($(N)+(-0.5,0)$) {\Large $n_{1}$};
\node[connect] (N) at (24,10) {};
\node at ($(N)+(0.5,0)$) {\Large $n_{1}$};
\node[connect] (N) at (13,10) {};
\node at ($(N)+(0.5,0)$) {\Large $n_{1}$};
\node[connect] (N) at (10,10) {};
\node at ($(N)+(0.5,0)$) {\Large $n_{1}$};
\node[connect] (N) at (-1.5,10) {};
\node at ($(N)+(0.5,0)$) {\Large $n_{1}$};
\node[connect] (N) at (-1.5,3) {};
\node at ($(N)+(0.4,0.5)$) {\Large $n_{2}$};
\node[connect] (N) at (2,3) {};
\node at ($(N)+(0.4,0.5)$) {\Large $n_{3}$};
\node[connect] (N) at (6,3) {};
\node at ($(N)+(0,-0.5)$) {\Large $n_{4}$};
\node[connect] (N) at (20,3) {};
\node at ($(N)+(0,-0.5)$) {\Large $n_{5}$};
\node[connect] (N) at (24,3) {};
\node at ($(N)+(0,0.5)$) {\Large $n_{6}$};
\node[connect] (N) at (28,3) {};
\node at ($(N)+(0.4,0.4)$) {\Large $n_{7}$};
\node[connect] (N) at (6,8.5) {};
\node at ($(N)+(0.5,0)$) {\Large $n_{8}$};
\node[connect] (N) at (6,13) {};
\node at ($(N)+(0.4,0.5)$) {\Large $n_{9}$};
\node[connect] (N) at (10,13) {};
\node at ($(N)+(0,0.5)$) {\Large $n_{10}$};
\node[connect] (N) at (2.5,13) {};
\node at ($(N)+(0,-0.4)$) {\Large $n_{11}$};
\node[connect] (N) at (-1.5,13) {};
\node at ($(N)+(0,0.5)$) {\Large $n_{12}$};
\node[connect] (N) at (20,8.5) {};
\node at ($(N)+(0.5,0)$) {\Large $n_{13}$};
\node[connect] (N) at (20,13) {};
\node at ($(N)+(0.4,0.5)$) {\Large $n_{14}$};
\node[connect] (N) at (24,13) {};
\node at ($(N)+(0,0.5)$) {\Large $n_{15}$};
\node[connect] (N) at (16.5,13) {};
\node at ($(N)+(0,-0.4)$) {\Large $n_{16}$};
\node[connect] (N) at (13,13) {};
\node at ($(N)+(0,0.5)$) {\Large $n_{17}$};
\node[connect] (N) at (-5,12.5) {};
\node at ($(N)+(0.5,-0.4)$) {\Large $n_{18}$};
%PMA
\draw[thick,dashed] (25,9) node (v2) {} -- (25,14.5) -- (12,14.5) -- (12,9) -- (v2);
\draw[thick,dashed] (-2.5,9) node (v2) {} -- (-2.5,14.5) -- (11,14.5) -- (11,9) -- (v2);
\node at (0.5,15) {\Large PMA 1};
\node at (-2,15) {\Large 0 m};
\node at (22,15) {\Large PMA 2};
\node at (24.5,15) {\Large 0.5 m};
\node at (-5.15,14.85) {\Large 2 m};
\node at (-3,13.5) {\Large $C_{33}$};
\end{tikzpicture}
\end{turn}    
\caption{Output pressure measurements}
\label{systemdiagram1}
\end{figure} 

\textbf{Estimation Results}

Together with the inputs from the lab and the nonlinear differential model the nonlinear parameter estimation is carried out. 



