\chapter{Nonlinear Estimation}
\label{NonLinResults}

\textbf{Estimation data}
In order to estimate accurately the unknown parameters of the system, adequate inputs signals have to be applied to the model. In this way, the system 
will work on different scenarios regarding the different combination of inputs signals. 

\figref{systemdiagram} shows the different combination between the OD of the PMA valves, and the steps applied to the main and PMA pumps. 

\begin{figure}[H]
\centering
\resizebox{0.75\linewidth}{!}{\input{report/tikz/InputPlot.tex}}
%\tikzset{pressure/.style={draw, circle, inner sep=0pt, text width=4mm, align=center}}
\tikzset{difpres/.style={draw, circle, inner sep=0pt, text width=5mm, align=center}}
\tikzset{connect/.style={draw,circle, inner sep=0pt, text width=2mm, align=center,fill=black}}
\tikzset{evalve/.style={draw, circle, inner sep=0pt, text width=3mm, align=center}}
\begin{tikzpicture}
%Pump
\begin{scope} [rotate around={90:(0,1)}, shift={(0,1)}]
\draw[transform shape] (0,0) circle (0.5);
\draw[transform shape] (0.5,0) -- (0,0.5);
\draw[transform shape] (0.5,0) -- (0,-0.5);
\end{scope}

\draw[transform shape] (2,0) circle (0.5);
\draw[transform shape] (2.5,0) -- (2,0.5);
\draw[transform shape] (2.5,0) -- (2,-0.5);

%Pump north
\node[draw,circle,minimum size=1cm] (p0) at (10.5,8.5) {};
\node(p1) at ($(p0)+(-0.5,0)$) {};
\node(p2) at ($(p1)+(0.5,0.5)$) {};
\node(p3) at ($(p1)+(1,0)$) {};
\draw(p1.center) -- (p2.center) -- (p3.center);

%Pump north
\node[draw,circle,minimum size=1cm] (p0) at (21.5,8.5) {};
\node(p1) at ($(p0)+(-0.5,0)$) {};
\node(p2) at ($(p1)+(0.5,0.5)$) {};
\node(p3) at ($(p1)+(1,0)$) {};
\draw(p1.center) -- (p2.center) -- (p3.center);

% %man-valve
\node(n1) at (0.25,3.5) {};
\node(n2) at (-0.25,3.5) {};
\node(n3) at (0.25,2.5) {};
\node(n4) at (-0.25,2.5) {};
\node(n5) at (0,3) {};
\node(n6) at (-0.5,3.25) {};
\node(n7) at (-0.5,2.75) {};
\draw(n1.center)--(n2.center)--(n3.center)--(n4.center)--(n1.center)--(n2.center);
\draw(n5.center)-|(n6.center)--(n7.center);

%man-valve
\node(n1) at (10.75,7.5) {};
\draw(n1.center) -- ($(n1)-(0.5,0)$) --
($(n1)-(0,1)$) -- ($(n1)-(0.5,1)$) --  (n1.center);
\draw($(n1)-(0.75,0.25)$) -- ($(n1)-(0.75,0.75)$) -- 
($(n1)-(0.75,0.5)$) --  ($(n1)-(0.25,0.5)$);

%man-valve
\node(n1) at (10.75,10.5) {};
\draw(n1.center) -- ($(n1)-(0.5,0)$) --
($(n1)-(0,1)$) -- ($(n1)-(0.5,1)$) --  (n1.center);
\draw($(n1)-(0.75,0.25)$) -- ($(n1)-(0.75,0.75)$) -- 
($(n1)-(0.75,0.5)$) --  ($(n1)-(0.25,0.5)$);

%man-valve
\node(n1) at (21.75,10.5) {};
\draw(n1.center) -- ($(n1)-(0.5,0)$) --
($(n1)-(0,1)$) -- ($(n1)-(0.5,1)$) --  (n1.center);
\draw($(n1)-(0.75,0.25)$) -- ($(n1)-(0.75,0.75)$) -- 
($(n1)-(0.75,0.5)$) --  ($(n1)-(0.25,0.5)$);

%man-valve
\node(n1) at (21.75,7.5) {};
\draw(n1.center) -- ($(n1)-(0.5,0)$) --
($(n1)-(0,1)$) -- ($(n1)-(0.5,1)$) --  (n1.center);
\draw($(n1)-(0.75,0.25)$) -- ($(n1)-(0.75,0.75)$) -- 
($(n1)-(0.75,0.5)$) --  ($(n1)-(0.25,0.5)$);

%elec-valve
\node(n1) at (4.25,10.5) {};
\node(n2) at (3.75,10.5) {};
\node(n3) at (4.25,9.5) {};
\node(n4) at (3.75,9.5) {};
\node(n5) at (4,10) {};
\node[evalve] (n6) at (3.5,10) {};
\draw(n1.center)--(n2.center)--(n3.center)--(n4.center)--(n1.center)--(n2.center);
\draw(n5.center)--(n6);

%man-valve
\node(n1) at (0.25,-0.5) {};
\node(n2) at (-0.25,-0.5) {};
\node(n3) at (0.25,-1.5) {};
\node(n4) at (-0.25,-1.5) {};
\node(n5) at (0,-1) {};
\node(n6) at (-0.5,-0.75) {};
\node(n7) at (-0.5,-1.25) {};
\draw(n1.center)--(n2.center)--(n3.center)--(n4.center)--(n1.center)--(n2.center);
\draw(n5.center)-|(n6.center)--(n7.center);

%man-valve
%\draw[very thick](-0.25,-1) -- (0.25,-1) -- (-0.25,-2) -- (0.25,-2) -- (-0.25,-1) -- (0.25,-1);
%\draw[very thick](0,-1.5) -- (-0.5,-1.5);
%\draw[very thick](-0.5,-1.25) -- (-0.5,-1.75);

%GND
\draw(-0.3,-2.5)--(0.3,-2.5);
\draw(-0.15,-2.6)--(0.15,-2.6);

%GND
\draw(3.7,8.5)--(4.3,8.5);
\draw(3.85,8.4)--(4.15,8.4);

%pipe
%right arcs
\draw (3,5.5) arc (90:-90:0.05);
\draw (3,5.3) arc (90:-90:0.05);
\draw (3,5.1) arc (90:-90:0.05);
%left arcs
\draw (1.5,5.4) arc (90:270:0.05);
\draw (1.5,5.2) arc (90:270:0.05);
\draw (1.5,5) arc (90:270:0.05);
%lines
\draw(1.5,5.5) -- (3,5.5);
\draw(1.5,5.4) -- (3,5.4);
\draw(1.5,5.3) -- (3,5.3);
\draw(1.5,5.2) -- (3,5.2);
\draw(1.5,5.1) -- (3,5.1);
\draw(1.5,5) -- (3,5);
\draw[rounded corners](1.5,4.9) -- (3.15,4.9) -- (3.15,5.5) -- (4.5,5.5);
%pipe name
%\node at (2,4) {$\text{C}_1$};

%pipe
%right arcs
\draw (7.5,5.5) arc (90:-90:0.05);
\draw (7.5,5.3) arc (90:-90:0.05);
\draw (7.5,5.1) arc (90:-90:0.05);
%left arcs
\draw (6,5.4) arc (90:270:0.05);
\draw (6,5.2) arc (90:270:0.05);
\draw (6,5) arc (90:270:0.05);
%lines
\draw(6,5.5) -- (7.5,5.5);
\draw(6,5.4) -- (7.5,5.4);
\draw(6,5.3) -- (7.5,5.3);
\draw(6,5.2) -- (7.5,5.2);
\draw(6,5.1) -- (7.5,5.1);
\draw(6,5) -- (7.5,5);
\draw[rounded corners](6,4.9) -- (7.65,4.9) -- (7.65,5.5) -- (9,5.5);

%pipe
\node(r1) at (12,5.5) {};
\draw (r1.center) -- ($(r1)+(1.5,0)$) arc (90:-90:0.05) -- 
($(r1)+(0,-0.1)$) arc (90:270:0.05) --
($(r1)+(1.5,-0.2)$) arc (90:-90:0.05) --
($(r1)+(0,-0.3)$) arc (90:270:0.05) --
($(r1)+(1.5,-0.4)$) arc (90:-90:0.05) --
($(r1)+(0,-0.5)$) arc (90:270:0.05);
\draw[rounded corners] ($(r1)+(0,-0.6)$) -- 
($(r1)+(1.65,-0.6)$) -- ($(r1)+(1.65,0)$) -- ($(r1)+(3,0)$);

%pipe
\node(r1) at (17,5.5) {};
\draw (r1.center) -- ($(r1)+(1.5,0)$) arc (90:-90:0.05) -- 
($(r1)+(0,-0.1)$) arc (90:270:0.05) --
($(r1)+(1.5,-0.2)$) arc (90:-90:0.05) --
($(r1)+(0,-0.3)$) arc (90:270:0.05) --
($(r1)+(1.5,-0.4)$) arc (90:-90:0.05) --
($(r1)+(0,-0.5)$) arc (90:270:0.05);
\draw[rounded corners] ($(r1)+(0,-0.6)$) -- 
($(r1)+(1.65,-0.6)$) -- ($(r1)+(1.65,0)$) -- ($(r1)+(3,0)$);

%pipe
\node(r1) at (19.5,-2.5) {};
\draw (r1.center) -- ($(r1)+(1.5,0)$) arc (90:-90:0.05) -- 
($(r1)+(0,-0.1)$) arc (90:270:0.05) --
($(r1)+(1.5,-0.2)$) arc (90:-90:0.05) --
($(r1)+(0,-0.3)$) arc (90:270:0.05) --
($(r1)+(1.5,-0.4)$) arc (90:-90:0.05) --
($(r1)+(0,-0.5)$) arc (90:270:0.05);
\draw[rounded corners] ($(r1)+(0,-0.6)$) -- 
($(r1)+(1.65,-0.6)$) -- ($(r1)+(1.65,0)$) -- ($(r1)+(3,0)$);

%pipe
\node(r1) at (30.5,5.5) {};
\draw (r1.center) -- ($(r1)+(1.5,0)$) arc (90:-90:0.05) -- 
($(r1)+(0,-0.1)$) arc (90:270:0.05) --
($(r1)+(1.5,-0.2)$) arc (90:-90:0.05) --
($(r1)+(0,-0.3)$) arc (90:270:0.05) --
($(r1)+(1.5,-0.4)$) arc (90:-90:0.05) --
($(r1)+(0,-0.5)$) arc (90:270:0.05);
\draw[rounded corners] ($(r1)+(0,-0.6)$) -- 
($(r1)+(1.65,-0.6)$) -- ($(r1)+(1.65,0)$) -- ($(r1)+(3,0)$);

%pipe
\node(r1) at (24,5.5) {};
\draw (r1.center) -- ($(r1)+(1.5,0)$) arc (90:-90:0.05) -- 
($(r1)+(0,-0.1)$) arc (90:270:0.05) --
($(r1)+(1.5,-0.2)$) arc (90:-90:0.05) --
($(r1)+(0,-0.3)$) arc (90:270:0.05) --
($(r1)+(1.5,-0.4)$) arc (90:-90:0.05) --
($(r1)+(0,-0.5)$) arc (90:270:0.05);
\draw[rounded corners] ($(r1)+(0,-0.6)$) -- 
($(r1)+(1.65,-0.6)$) -- ($(r1)+(1.65,0)$) -- ($(r1)+(3,0)$);

%pipe
\node(r1) at (10.5,12) {};
\begin{scope} [rotate around={90:(r1)}]
\draw (r1.center) -- ($(r1)+(1.5,0)$) arc (90:-90:0.05) -- 
($(r1)+(0,-0.1)$) arc (90:270:0.05) --
($(r1)+(1.5,-0.2)$) arc (90:-90:0.05) --
($(r1)+(0,-0.3)$) arc (90:270:0.05) --
($(r1)+(1.5,-0.4)$) arc (90:-90:0.05) --
($(r1)+(0,-0.5)$) arc (90:270:0.05);
\draw[rounded corners] ($(r1)+(0,-0.6)$) -- 
($(r1)+(1.65,-0.6)$) -- ($(r1)+(1.65,0)$) -- ($(r1)+(3,0)$);
\end{scope}

%pipe
\node(r1) at (7.5,15) {};
\draw (r1.center) -- ($(r1)+(1.5,0)$) arc (90:-90:0.05) -- 
($(r1)+(0,-0.1)$) arc (90:270:0.05) --
($(r1)+(1.5,-0.2)$) arc (90:-90:0.05) --
($(r1)+(0,-0.3)$) arc (90:270:0.05) --
($(r1)+(1.5,-0.4)$) arc (90:-90:0.05) --
($(r1)+(0,-0.5)$) arc (90:270:0.05);
\draw[rounded corners] ($(r1)+(0,-0.6)$) -- 
($(r1)+(1.65,-0.6)$) -- ($(r1)+(1.65,0)$) -- ($(r1)+(3,0)$);

%pipe
\node(r1) at (12,15) {};
\draw (r1.center) -- ($(r1)+(1.5,0)$) arc (90:-90:0.05) -- 
($(r1)+(0,-0.1)$) arc (90:270:0.05) --
($(r1)+(1.5,-0.2)$) arc (90:-90:0.05) --
($(r1)+(0,-0.3)$) arc (90:270:0.05) --
($(r1)+(1.5,-0.4)$) arc (90:-90:0.05) --
($(r1)+(0,-0.5)$) arc (90:270:0.05);
\draw[rounded corners] ($(r1)+(0,-0.6)$) -- 
($(r1)+(1.65,-0.6)$) -- ($(r1)+(1.65,0)$) -- ($(r1)+(3,0)$);

%pressure sensor
\node[connect] (PD1) at (1,9) {};
\node(P1) at ($(PD1)+(0,1)$) [pressure] {P};
\draw(PD1) -- (P1);

%differential pressure sensor
\node[connect] (CDP1) at (0,4) {};
\node[connect] (CDP2) at (0,-2) {};
\node[difpres] (DP1) at (-1.5,0.5) {DP};
\draw(CDP1) -| (DP1)  |- (CDP2);

%Connections
\draw(0,1.5) -- (0,2.5);
\draw(0,0.5) -- (0,-0.5);
\draw(0,-1.5) -- (0,-2.5);
%\draw[rounded corners](0,3.5)--(0,5.5)--(1.5,5.5);


\end{tikzpicture}    
\caption{Inputs to the parameter identification}
\label{systemdiagram}
\end{figure} 

The estimation process, based on the applied input data, tries to fit the established output signal data to the simulation output. In the physical setup $8$ 
different pressure relative sensors are available which are used as output signals. \figref{systemdiagram1} shows the pressure measurements in the different 
nodes.

\begin{figure}[H]
\centering
\resizebox{0.75\linewidth}{!}{\input{report/tikz/OutputPlot.tex}}
%\tikzset{pressure/.style={draw, circle, inner sep=0pt, text width=4mm, align=center}}
\tikzset{difpres/.style={draw, circle, inner sep=0pt, text width=5mm, align=center}}
\tikzset{connect/.style={draw,circle, inner sep=0pt, text width=2mm, align=center,fill=black}}
\tikzset{evalve/.style={draw, circle, inner sep=0pt, text width=3mm, align=center}}
\begin{tikzpicture}
%Pump
\begin{scope} [rotate around={90:(0,1)}, shift={(0,1)}]
\draw[transform shape] (0,0) circle (0.5);
\draw[transform shape] (0.5,0) -- (0,0.5);
\draw[transform shape] (0.5,0) -- (0,-0.5);
\end{scope}

\draw[transform shape] (2,0) circle (0.5);
\draw[transform shape] (2.5,0) -- (2,0.5);
\draw[transform shape] (2.5,0) -- (2,-0.5);

%Pump north
\node[draw,circle,minimum size=1cm] (p0) at (10.5,8.5) {};
\node(p1) at ($(p0)+(-0.5,0)$) {};
\node(p2) at ($(p1)+(0.5,0.5)$) {};
\node(p3) at ($(p1)+(1,0)$) {};
\draw(p1.center) -- (p2.center) -- (p3.center);

%Pump north
\node[draw,circle,minimum size=1cm] (p0) at (21.5,8.5) {};
\node(p1) at ($(p0)+(-0.5,0)$) {};
\node(p2) at ($(p1)+(0.5,0.5)$) {};
\node(p3) at ($(p1)+(1,0)$) {};
\draw(p1.center) -- (p2.center) -- (p3.center);

% %man-valve
\node(n1) at (0.25,3.5) {};
\node(n2) at (-0.25,3.5) {};
\node(n3) at (0.25,2.5) {};
\node(n4) at (-0.25,2.5) {};
\node(n5) at (0,3) {};
\node(n6) at (-0.5,3.25) {};
\node(n7) at (-0.5,2.75) {};
\draw(n1.center)--(n2.center)--(n3.center)--(n4.center)--(n1.center)--(n2.center);
\draw(n5.center)-|(n6.center)--(n7.center);

%man-valve
\node(n1) at (10.75,7.5) {};
\draw(n1.center) -- ($(n1)-(0.5,0)$) --
($(n1)-(0,1)$) -- ($(n1)-(0.5,1)$) --  (n1.center);
\draw($(n1)-(0.75,0.25)$) -- ($(n1)-(0.75,0.75)$) -- 
($(n1)-(0.75,0.5)$) --  ($(n1)-(0.25,0.5)$);

%man-valve
\node(n1) at (10.75,10.5) {};
\draw(n1.center) -- ($(n1)-(0.5,0)$) --
($(n1)-(0,1)$) -- ($(n1)-(0.5,1)$) --  (n1.center);
\draw($(n1)-(0.75,0.25)$) -- ($(n1)-(0.75,0.75)$) -- 
($(n1)-(0.75,0.5)$) --  ($(n1)-(0.25,0.5)$);

%man-valve
\node(n1) at (21.75,10.5) {};
\draw(n1.center) -- ($(n1)-(0.5,0)$) --
($(n1)-(0,1)$) -- ($(n1)-(0.5,1)$) --  (n1.center);
\draw($(n1)-(0.75,0.25)$) -- ($(n1)-(0.75,0.75)$) -- 
($(n1)-(0.75,0.5)$) --  ($(n1)-(0.25,0.5)$);

%man-valve
\node(n1) at (21.75,7.5) {};
\draw(n1.center) -- ($(n1)-(0.5,0)$) --
($(n1)-(0,1)$) -- ($(n1)-(0.5,1)$) --  (n1.center);
\draw($(n1)-(0.75,0.25)$) -- ($(n1)-(0.75,0.75)$) -- 
($(n1)-(0.75,0.5)$) --  ($(n1)-(0.25,0.5)$);

%elec-valve
\node(n1) at (4.25,10.5) {};
\node(n2) at (3.75,10.5) {};
\node(n3) at (4.25,9.5) {};
\node(n4) at (3.75,9.5) {};
\node(n5) at (4,10) {};
\node[evalve] (n6) at (3.5,10) {};
\draw(n1.center)--(n2.center)--(n3.center)--(n4.center)--(n1.center)--(n2.center);
\draw(n5.center)--(n6);

%man-valve
\node(n1) at (0.25,-0.5) {};
\node(n2) at (-0.25,-0.5) {};
\node(n3) at (0.25,-1.5) {};
\node(n4) at (-0.25,-1.5) {};
\node(n5) at (0,-1) {};
\node(n6) at (-0.5,-0.75) {};
\node(n7) at (-0.5,-1.25) {};
\draw(n1.center)--(n2.center)--(n3.center)--(n4.center)--(n1.center)--(n2.center);
\draw(n5.center)-|(n6.center)--(n7.center);

%man-valve
%\draw[very thick](-0.25,-1) -- (0.25,-1) -- (-0.25,-2) -- (0.25,-2) -- (-0.25,-1) -- (0.25,-1);
%\draw[very thick](0,-1.5) -- (-0.5,-1.5);
%\draw[very thick](-0.5,-1.25) -- (-0.5,-1.75);

%GND
\draw(-0.3,-2.5)--(0.3,-2.5);
\draw(-0.15,-2.6)--(0.15,-2.6);

%GND
\draw(3.7,8.5)--(4.3,8.5);
\draw(3.85,8.4)--(4.15,8.4);

%pipe
%right arcs
\draw (3,5.5) arc (90:-90:0.05);
\draw (3,5.3) arc (90:-90:0.05);
\draw (3,5.1) arc (90:-90:0.05);
%left arcs
\draw (1.5,5.4) arc (90:270:0.05);
\draw (1.5,5.2) arc (90:270:0.05);
\draw (1.5,5) arc (90:270:0.05);
%lines
\draw(1.5,5.5) -- (3,5.5);
\draw(1.5,5.4) -- (3,5.4);
\draw(1.5,5.3) -- (3,5.3);
\draw(1.5,5.2) -- (3,5.2);
\draw(1.5,5.1) -- (3,5.1);
\draw(1.5,5) -- (3,5);
\draw[rounded corners](1.5,4.9) -- (3.15,4.9) -- (3.15,5.5) -- (4.5,5.5);
%pipe name
%\node at (2,4) {$\text{C}_1$};

%pipe
%right arcs
\draw (7.5,5.5) arc (90:-90:0.05);
\draw (7.5,5.3) arc (90:-90:0.05);
\draw (7.5,5.1) arc (90:-90:0.05);
%left arcs
\draw (6,5.4) arc (90:270:0.05);
\draw (6,5.2) arc (90:270:0.05);
\draw (6,5) arc (90:270:0.05);
%lines
\draw(6,5.5) -- (7.5,5.5);
\draw(6,5.4) -- (7.5,5.4);
\draw(6,5.3) -- (7.5,5.3);
\draw(6,5.2) -- (7.5,5.2);
\draw(6,5.1) -- (7.5,5.1);
\draw(6,5) -- (7.5,5);
\draw[rounded corners](6,4.9) -- (7.65,4.9) -- (7.65,5.5) -- (9,5.5);

%pipe
\node(r1) at (12,5.5) {};
\draw (r1.center) -- ($(r1)+(1.5,0)$) arc (90:-90:0.05) -- 
($(r1)+(0,-0.1)$) arc (90:270:0.05) --
($(r1)+(1.5,-0.2)$) arc (90:-90:0.05) --
($(r1)+(0,-0.3)$) arc (90:270:0.05) --
($(r1)+(1.5,-0.4)$) arc (90:-90:0.05) --
($(r1)+(0,-0.5)$) arc (90:270:0.05);
\draw[rounded corners] ($(r1)+(0,-0.6)$) -- 
($(r1)+(1.65,-0.6)$) -- ($(r1)+(1.65,0)$) -- ($(r1)+(3,0)$);

%pipe
\node(r1) at (17,5.5) {};
\draw (r1.center) -- ($(r1)+(1.5,0)$) arc (90:-90:0.05) -- 
($(r1)+(0,-0.1)$) arc (90:270:0.05) --
($(r1)+(1.5,-0.2)$) arc (90:-90:0.05) --
($(r1)+(0,-0.3)$) arc (90:270:0.05) --
($(r1)+(1.5,-0.4)$) arc (90:-90:0.05) --
($(r1)+(0,-0.5)$) arc (90:270:0.05);
\draw[rounded corners] ($(r1)+(0,-0.6)$) -- 
($(r1)+(1.65,-0.6)$) -- ($(r1)+(1.65,0)$) -- ($(r1)+(3,0)$);

%pipe
\node(r1) at (19.5,-2.5) {};
\draw (r1.center) -- ($(r1)+(1.5,0)$) arc (90:-90:0.05) -- 
($(r1)+(0,-0.1)$) arc (90:270:0.05) --
($(r1)+(1.5,-0.2)$) arc (90:-90:0.05) --
($(r1)+(0,-0.3)$) arc (90:270:0.05) --
($(r1)+(1.5,-0.4)$) arc (90:-90:0.05) --
($(r1)+(0,-0.5)$) arc (90:270:0.05);
\draw[rounded corners] ($(r1)+(0,-0.6)$) -- 
($(r1)+(1.65,-0.6)$) -- ($(r1)+(1.65,0)$) -- ($(r1)+(3,0)$);

%pipe
\node(r1) at (30.5,5.5) {};
\draw (r1.center) -- ($(r1)+(1.5,0)$) arc (90:-90:0.05) -- 
($(r1)+(0,-0.1)$) arc (90:270:0.05) --
($(r1)+(1.5,-0.2)$) arc (90:-90:0.05) --
($(r1)+(0,-0.3)$) arc (90:270:0.05) --
($(r1)+(1.5,-0.4)$) arc (90:-90:0.05) --
($(r1)+(0,-0.5)$) arc (90:270:0.05);
\draw[rounded corners] ($(r1)+(0,-0.6)$) -- 
($(r1)+(1.65,-0.6)$) -- ($(r1)+(1.65,0)$) -- ($(r1)+(3,0)$);

%pipe
\node(r1) at (24,5.5) {};
\draw (r1.center) -- ($(r1)+(1.5,0)$) arc (90:-90:0.05) -- 
($(r1)+(0,-0.1)$) arc (90:270:0.05) --
($(r1)+(1.5,-0.2)$) arc (90:-90:0.05) --
($(r1)+(0,-0.3)$) arc (90:270:0.05) --
($(r1)+(1.5,-0.4)$) arc (90:-90:0.05) --
($(r1)+(0,-0.5)$) arc (90:270:0.05);
\draw[rounded corners] ($(r1)+(0,-0.6)$) -- 
($(r1)+(1.65,-0.6)$) -- ($(r1)+(1.65,0)$) -- ($(r1)+(3,0)$);

%pipe
\node(r1) at (10.5,12) {};
\begin{scope} [rotate around={90:(r1)}]
\draw (r1.center) -- ($(r1)+(1.5,0)$) arc (90:-90:0.05) -- 
($(r1)+(0,-0.1)$) arc (90:270:0.05) --
($(r1)+(1.5,-0.2)$) arc (90:-90:0.05) --
($(r1)+(0,-0.3)$) arc (90:270:0.05) --
($(r1)+(1.5,-0.4)$) arc (90:-90:0.05) --
($(r1)+(0,-0.5)$) arc (90:270:0.05);
\draw[rounded corners] ($(r1)+(0,-0.6)$) -- 
($(r1)+(1.65,-0.6)$) -- ($(r1)+(1.65,0)$) -- ($(r1)+(3,0)$);
\end{scope}

%pipe
\node(r1) at (7.5,15) {};
\draw (r1.center) -- ($(r1)+(1.5,0)$) arc (90:-90:0.05) -- 
($(r1)+(0,-0.1)$) arc (90:270:0.05) --
($(r1)+(1.5,-0.2)$) arc (90:-90:0.05) --
($(r1)+(0,-0.3)$) arc (90:270:0.05) --
($(r1)+(1.5,-0.4)$) arc (90:-90:0.05) --
($(r1)+(0,-0.5)$) arc (90:270:0.05);
\draw[rounded corners] ($(r1)+(0,-0.6)$) -- 
($(r1)+(1.65,-0.6)$) -- ($(r1)+(1.65,0)$) -- ($(r1)+(3,0)$);

%pipe
\node(r1) at (12,15) {};
\draw (r1.center) -- ($(r1)+(1.5,0)$) arc (90:-90:0.05) -- 
($(r1)+(0,-0.1)$) arc (90:270:0.05) --
($(r1)+(1.5,-0.2)$) arc (90:-90:0.05) --
($(r1)+(0,-0.3)$) arc (90:270:0.05) --
($(r1)+(1.5,-0.4)$) arc (90:-90:0.05) --
($(r1)+(0,-0.5)$) arc (90:270:0.05);
\draw[rounded corners] ($(r1)+(0,-0.6)$) -- 
($(r1)+(1.65,-0.6)$) -- ($(r1)+(1.65,0)$) -- ($(r1)+(3,0)$);

%pressure sensor
\node[connect] (PD1) at (1,9) {};
\node(P1) at ($(PD1)+(0,1)$) [pressure] {P};
\draw(PD1) -- (P1);

%differential pressure sensor
\node[connect] (CDP1) at (0,4) {};
\node[connect] (CDP2) at (0,-2) {};
\node[difpres] (DP1) at (-1.5,0.5) {DP};
\draw(CDP1) -| (DP1)  |- (CDP2);

%Connections
\draw(0,1.5) -- (0,2.5);
\draw(0,0.5) -- (0,-0.5);
\draw(0,-1.5) -- (0,-2.5);
%\draw[rounded corners](0,3.5)--(0,5.5)--(1.5,5.5);


\end{tikzpicture}    
\caption{Output pressure measurements}
\label{systemdiagram1}
\end{figure} 

\textbf{Estimation Results}

Together with the inputs from the lab and the nonlinear differential model the nonlinear parameter estimation is carried out. 



