\chapter{Unit Conversion}
\label{unitCON}
\textbf{Unit transformation for the pipe model}

Due to the large difference between the SI-units of flow, [$\text{m}^3/\text{s}$], and pressure, [Pa], a conversion from seconds to hours and pascal to Bar is made.

The final pipe model from \eqref{FinalPipeModel}, is shown below.   

\begin{equation}
\begin{split}
   \frac{L \rho}{A} \frac{dq}{dt} &=\Delta p - \frac{8fL}{\pi^{2}gD^5} \rho g  |q| q - k_f \frac{8}{\pi^2gD^4} \rho g |q| q - \Delta z \rho g \\
   &=\Delta p - (\frac{8fL}{\pi^{2}gD^5} + k_f \frac{8}{\pi^2gD^4}) \rho g |q| q - \Delta z \rho g
\end{split}
\label{UnitCon_eq}
\end{equation}

$1 [\text{bar}] = 10^5[\text{Pa}]$. Therefore we can rewrite \eqref{UnitCon_eq} to: 

\begin{equation}
\begin{split}
   \frac{L \rho}{A\cdot10^5} \frac{dq}{dt} &= \Delta \frac{p}{10^5} - (\frac{8fL}{\pi^{2}gD^5\cdot10^5} + k_f \frac{8}{\pi^2gD^4\cdot10^5}) \rho g |q| q - \frac{\Delta z \rho g}{10^5}\\
   \frac{L \rho}{A\cdot10^5} \frac{dq}{dt} &= \Delta p_{bar} - (\frac{8fL}{\pi^{2}gD^5\cdot10^5} + k_f \frac{8}{\pi^2gD^4\cdot10^5}) \rho g |q| q - \frac{\Delta z \rho g}{10^5}
\end{split}
\end{equation}

The conversion from $[\frac{m^3}{s}]$ to $[\frac{m^3}{h}]$ is $\frac{m^3}{s} 3600 = \frac{m^3}{s}$. \eqref{UnitCon_eq} can be written as:

\begin{equation}
   \frac{L \rho}{A\cdot10^5} \frac{d}{dt}\frac{q}{3600} = \Delta \frac{p}{10^5} - (\frac{8fL}{\pi^{2}gD^5\cdot10^5} + k_f \frac{8}{\pi^2gD^4\cdot10^5}) \rho g \frac{|q|}{3600} \frac{q}{3600} - \frac{\Delta z \rho g}{10^5}
\end{equation}

There is no need to apply the unit conversion to the final valve model from \eqref{CompleteValveModel}, due to the parameter $k_v$ being designed for the water flow in $m^3$
through the  valve in one hour and at a pressure drop across the valve of 1 Bar. 

In the pump final model \eqref{eq:PumpModel} the constants are scaled so the pump equation has the units in Bar and the flow has the units in $m^3/h$.

\textbf{Water Tower Hydraulic Capacitance Unit Transformation}

In \secref{WaterTankModel} the units of the WT hydraulic capacitance, $C_H$ are shown. This units are given in S.I units which makes it necessary to convert them into the unit system followed throughout the project. Below the unit conversion procedure for $C_H$ is carried out. 

\begin{equation}
	C_H = \frac{A}{\rho g} = \frac{m^2}{\frac{kg}{m^3} \frac{m}{s^2}} = \frac{m^2}{\frac{F}{m^2} \frac{1}{m}} =  \frac{m^2}{\frac{P}{m}} = \frac{m^3}{P}
	\label{WT_unit}
\end{equation}

\begin{minipage}[t]{0.20\textwidth}
Where\\
\hspace*{8mm} $F$ \\
\hspace*{8mm} $P$ \\
\end{minipage}
\begin{minipage}[t]{0.68\textwidth}
\vspace*{2mm}
is Force\\
is Pressure\\
\end{minipage}
\begin{minipage}[t]{0.10\textwidth}
\vspace*{2mm}
\textcolor{White}{te}$\unit{N}$\\
\textcolor{White}{te}$\unit{Pa}$
\end{minipage}

As well as with the pipes equation the pressure need to be converted from [Bar] to [Pa], applying unit conversion to \eqref{WT_unit}:

\begin{equation}
	C_H = \frac{m^{3} \cdot 10^5}{P}
	\label{WT_unit_final}
\end{equation}