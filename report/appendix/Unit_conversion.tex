\chapter{Unit Conversion}
\label{unitCON}
\textbf{Unit transformation for the pipe model}

Due to the large difference between the SI-units of flow, [$\text{m}^3/\text{s}$], and pressure, [Pa], a conversion from seconds to hours and pascal to bar is made.

The final pipe model from \eqref{FinalPipeModel}, is shown below.   

\begin{equation}
\begin{split}
   \frac{L \rho}{A} \frac{dq}{dt} &=\Delta p - \frac{8fL}{\pi^{2}gD^5} \rho g  |q| q - k_f \frac{8}{\pi^2gD^4} \rho g |q| q - \Delta z_h \rho g \\
   &=\Delta p - (\frac{8fL}{\pi^{2}gD^5} + k_f \frac{8}{\pi^2gD^4}) \rho g |q| q - \Delta z_h \rho g
\end{split}
\label{UnitCon_eq}
\end{equation}

$1 [\text{bar}] = 10^5[\text{Pa}]$. Therefore \eqref{UnitCon_eq} can be rewritren as: 

\begin{equation}
\begin{split}
   \frac{L \rho}{A\cdot10^5} \frac{dq}{dt} &=  \frac{\Delta p}{10^5} - (\frac{8fL}{\pi^{2}gD^5\cdot10^5} + k_f \frac{8}{\pi^2gD^4\cdot10^5}) \rho g |q| q - \frac{\Delta z_h \rho g}{10^5}\\
   \frac{L \rho}{A\cdot10^5} \frac{dq}{dt} &= \Delta p_{bar} - (\frac{8fL}{\pi^{2}gD^5\cdot10^5} + k_f \frac{8}{\pi^2gD^4\cdot10^5}) \rho g |q| q - \frac{\Delta z_h \rho g}{10^5}
\end{split}
\end{equation}

The coordinate transformation from $[\frac{m^3}{s}]$ to $[\frac{m^3}{h}]$ can be done by multiplying by 3600 on the left-hand side. \eqref{UnitCon_eq} can be written as:

\begin{equation}
   \frac{L \rho}{A\cdot10^5} \frac{d}{dt}\frac{q}{3600} = \frac{\Delta p}{10^5} - (\frac{8fL}{\pi^{2}gD^5\cdot10^5} + k_f \frac{8}{\pi^2gD^4\cdot10^5}) \rho g \frac{|q|}{3600} \frac{q}{3600} - \frac{\Delta z_h \rho g}{10^5}
\end{equation}

There is not any need to apply unit conversion to the final valve model described in \eqref{CompleteValveModel}, since the parameter, $k_v$, being designed for the water flow in $[m^3]$ taking into account one hour at a pressure drop across the valve of 1 bar. 
\\
In the pump final model \eqref{eq:PumpModel}, the constants are scaled so the pump equation has the units in $[bar]$ and the flow has the units in $[m^3/h]$.

\textbf{Water Tower Hydraulic Capacitance Unit Transformation}

In \secref{WaterTankModel}, the unit of the WT hydraulic capacitance, $C_H$ is shown. This unit is given in SI which makes it necessary to convert it into the unit system followed throughout the project. Below the procedure of the unit conversion for $C_H$ is shown: 

\begin{equation}
	C_H \Bigg[\frac{A}{\rho g}\Bigg] = C_H \Bigg[\frac{m^2}{\frac{kg}{m^3} \frac{m}{s^2}}\Bigg]= C_H \Bigg[\frac{m^2}{\frac{N}{m^2} \frac{1}{m}}\Bigg] =  C_H \Bigg[\frac{m^2}{\frac{Pa}{m}}\Bigg] = C_H \Bigg[\frac{m^3}{Pa}\Bigg]
	\label{WT_unit}
\end{equation}


As well as with the pipes equation, the pressure needs to be converted from [Pa] to [Bar], applying unit conversion to \eqref{WT_unit}:

\begin{equation}
	C_H \Bigg[\frac{m^3}{Pa}\Bigg] = K \cdot C_H \Bigg[\frac{m^{3}}{Bar}\Bigg]
	\label{WT_unit_final}
\end{equation}


Where $K = 10^5$, which converts the pressure to [Bar]. 