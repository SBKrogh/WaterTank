\section{Test of model predictive control}
\label{sec:MPT_test}

In this section, the model predictive controller obtained in \chapref{control_design} is implemented and tested on the water system available at Aalborg university. 

\subsection*{Purpose:}

The purpose of this test is to verify if the designed controller meets the requirements stated in \chapref{Requirements_and_constraints} and thus keeping the pressure at each CP above the desired minimum while minimizing the running cost of the system.

%\subsection{One clamp push} % \label{app:...}
\subsection*{Test equipment:}
\begin{itemize}
\item The water system at Aalborg university [AAU No: 100911]
\end{itemize}

\subsection*{Procedure:}
The designed MPC controller is tested on the system for a period corresponding to 24 hours. During the test each hour is scaled to 5 minutes in real time as decribed in \chapref{accepttest}. Prediction data of the electrical price is provided to the MPC and described in \secref{sec:cost_fkt}. The consumer behavior, resulting in valve opening, is applied to the system, this data can be found under the path: \path{CD:/Data/MPC test/prediction data}. 
In order to analyze if the MPC is minimizing the energy consumption of the system, a PI controller is implemented and tested under the same conditions as the MPC.    

\subsection*{Measuring data:}
The measurements data for the MPC can be found under the path: \path{CD:/Data/MPC test/results}, and the data for PI controller can be found under the path \path{CD:/Data/PI test}. 

\subsection*{Results:}
The results of the MPC controller:
\begin{itemize}
	\item Pma pressures
	\item WT level
	\item Running costs
\end{itemize}	

The results of the PI controller:
\begin{itemize}
	\item Pma pressures
	\item WT level
	\item Running costs
\end{itemize}	

Comparison:

\subsection*{Uncertainties of measurement:}
\begin{itemize}
\item ...
\end{itemize}

\subsection*{Conclusion:}
