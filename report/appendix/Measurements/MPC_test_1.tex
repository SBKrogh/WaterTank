\chapter{Thoughts about testing of the model predictive control}
\label{sec:MPT_test}

In this section some considerations about how the controller, designed in \chapref{control_design}, should have been tested on the available test setup are given. These are based on the desire to verify if the requirements are met, these are described in \chapref{Requirements_and_constraints} as keeping the pressure at each CP above the desired minimum while minimizing the running cost of the system.

%\subsection*{Purpose:}
%%%%%%%%%%%%%%%%%%%%%%%%%%%%%%%%%%%%%%%%%%%%%%%%%%%%%%%%%%%%%%%%%%%%%%%%%%%%%%%%%%%

The MPC is designed with a receding horizon control with a 24 hour control horizon due to the behavior of the electrical price and the water consumption which both have a periodicity of 24 hours. 
Based on this, a test conducted over a 24 hour period would allow to investigate the main characteristics of the MPC when price and consumption deviations during a day occur due to the periodicity of the data. The controller will be tested according to the requirements found in \chapref{Requirements_and_constraints}, these are: 
%\todo{wrong notation and missing requirements}
\begin{itemize}
	\item Pressure at CP, $0.08 < \bm{y} < 0.14 \:[\text{bar}]$
%
	\item Minimizing the total cost of running the system
\end{itemize}    

A results of such a test could then be compared to a system without MPC in order to verify if the MPC is capable of saving money, and in what degree.  

%%%%%%%%%%%%%%%%%%%%%%%%%%%%%%%%%%%%%%%%%%%%%%%%%%%%%%%%%%%%%%%%%%%%%%%%%%%%%%%%%%%

%\subsection{One clamp push} % \label{app:...}
% \subsection*{Test equipment:}
% \begin{itemize}
% \item The water system at Aalborg university [AAU No: 100911]
% \end{itemize}

% \subsection*{Procedure:}
% The designed MPC controller is tested on the system for a period corresponding to 24 hours. During the test each hour is scaled to 5 minutes in real time as decribed in \chapref{accepttest}. Prediction data of the electrical price is provided to the MPC and described in \secref{sec:cost_fkt}. The consumer behavior, resulting in valve opening, is applied to the system, this data can be found under the path: \path{CD:/Data/MPC test/prediction data}. 
% In order to analyze if the MPC is minimizing the energy consumption of the system, a PI controller is implemented and tested under the same conditions as the MPC.    

% \subsection*{Measuring data:}
% The measurements data for the MPC can be found under the path: \path{CD:/Data/MPC test/results}, and the data for PI controller can be found under the path \path{CD:/Data/PI test}. 

% \subsection*{Results:}
% The results of the MPC controller:
% \begin{itemize}
% 	\item Pma pressures
% 	\item WT level
% 	\item Running costs
% \end{itemize}	

% The results of the PI controller:
% \begin{itemize}
% 	\item Pma pressures
% 	\item WT level
% 	\item Running costs
% \end{itemize}	

% Comparison:

% \subsection*{Uncertainties of measurement:}
% \begin{itemize}
% \item ...
% \end{itemize}

% \subsection*{Conclusion:}
