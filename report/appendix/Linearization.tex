\chapter{Linearization}
\label{chap:Lin}

\textbf{Valve equation}

The Kv value as a function of the opening degree is for a valve in the system given by:

\begin{equation}
	Kv(OD)=kv_{100} e^{(n_{gl} \gamma)}
\end{equation}


The pressure across the valve as a function of the flow is given by:

\begin{equation}
	\mu(q)=\frac{1}{(Kv)^2} q |q|
\end{equation}

Gathering the two previous equations allows to describe the pressure pressure across the valve as a combined function of both the flow and the opening degree.

\begin{equation}
	M(q,OD)=\frac{1}{(kv_{100} e^{(n_{gl} \gamma)})^2} q |q| 
\end{equation}	


The linerization of the function $M(q,OD)$ by multi variable Taylor expansion in the operating points $\bar{q}$ and $\bar{OD}$ is given by the form

\begin{equation}
\begin{split}
	M(q,OD) &\approx M(a,b) + \frac{\partial}{\partial x} (M(a,b)) (x-a) + \frac{\partial}{\partial y} (M(a,b)) (y-b) \\
			&\approx e^{\frac{2 (\theta_{off} - \bar{OD}) n_{gl}}{\theta_{max}-\theta_{off}}+2} \bar{q} |\bar{q}| -2 \frac{e^{\frac{2 (\theta_{off} - \bar{OD}) n_{gl}}{\theta_{max}-\theta_{off}}+2} n_{gl} \hat{OD} \bar{q} |\bar{q}|}{\theta_{max}-\theta_{off}} + 2 e^{\frac{2 (\theta_{off} - \bar{OD}) n_{gl}}{\theta_{max}-\theta_{off}}+2} \hat{q} |\bar{q}|
\end{split}
\end{equation}
 \begin{minipage}[t]{0.40\textwidth}
Where\\
\hspace*{8mm} $a=\bar{q}$ \\
\hspace*{8mm} $x=\bar{q}+\hat{q}$ \\
\hspace*{8mm} $b=\bar{OD}$ \\
and \hspace*{0.7mm} $y=\bar{OD}+\hat{OD}$ 
\end{minipage}
\begin{minipage}[t]{0.48\textwidth}
\vspace*{2mm}
\hspace*{8mm}
\hspace*{8mm}
\hspace*{8mm}
\hspace*{8mm}
\end{minipage}
\begin{minipage}[t]{0.10\textwidth}
\vspace*{2mm}
\textcolor{White}{te}$\unit{\frac{m^3}{s}}$\\
\textcolor{White}{te}$\unit{\frac{m^3}{s}}$\\
\textcolor{White}{te}$\unit{\degree}$\\
\textcolor{White}{te}$\unit{\degree}$
\end{minipage}	





\bigskip
\bigskip 
\textbf{Pipe equation}

The pressure across a pipe as a function of the flow is given by:

\begin{equation}
	\mu(q)=C_p q |q|
\end{equation}



The first order linear Taylor expansion in the operating point $\bar{q}$ is given as:

\begin{equation}
	\begin{split}
		\mu(x) &\approx \mu(a) + \frac{\partial}{\partial x} \mu(a) (x-a) \\
			   &\approx C_p \bar{q} |\bar{q}| + 2 C_p \bar{q} \hat{q}
	\end{split}
\end{equation}


\bigskip
\bigskip 
\textbf{Pump equation}

Concerning the pump that connects the WT with the remaining system the rotational speed is zero. Therefore will the pumps influence be described by a resistive term which is gives a differential pressure drop as a function of the flow.

\begin{equation}
	\Delta p = (\frac{2}{{kv_{100}^2}} - a_{h2}) q |q|
\end{equation}

The first order linear Taylor expansion in the operating point $\bar{q}$ is given as:

\begin{equation}
	\begin{split}
		\Delta p(x) &\approx \Delta p(a) + \frac{\partial}{\partial x} \Delta p(a) (x-a) \\
			   &\approx (\frac{2}{{kv_{100}^2}} - a_{h2}) \bar{q} |\bar{q}| + 2 (\frac{2}{{kv_{100}^2}} - a_{h2}) \bar{q} \hat{q}
	\end{split}
\end{equation}