\chapter{Pump linearization and PI controller - V2}
\label{cha:linear_pump2}

The control structure chosen in \chapref{control_problem} and \figref{fig:control_structure} includes a PI controller, which is to designed. The PI controllers purpose is to use the optimized control output from the MPC as a control reference. 

The control focus of this project has been on the model predictive control, see \secref{sec:MPC}. Therefore a simple PI controller has been design, where the control structure can be seen on \figref{fig:simple_PI}. The steady state error is desired to be zero, which is possible by an integrator pole in the controller. The gain of the controller is then to be found. 

\begin{figure}[H]
\centering
\begin{tikzpicture} [scale=0.8,transform shape]

\draw  (2.5,1) rectangle (4.5,0);
\node at (3.5,0.5) {G(s)};

\draw  (-1,1) rectangle (1,0);
\node at (0,0.5) {D(s)};

\draw[-triangle 60] (-3,0.5) -- (-1,0.5);
\draw[-triangle 60] (4.5,0.5) -- (6,0.5);
\draw[-triangle 60] (1,0.5) -- (2.5,0.5);
\draw[-triangle 60] (-5,0.5) -- (-3.5,0.5);
%\draw[-triangle 60] (-5.5,0.5) -- (-4.5,0.5);

\draw[-triangle 60] (5.25,0.5) -- (5.25,-1) -- (-3.25,-1) -- (-3.25,0.25);

%\node at (-5.5,1) {$CP_{ref}$};
\node at (-5,0.75) {$\Delta P_{ref}$};
\node at (-2.1,0.75) {$\Delta P_{error}$};
\node at (-3.75,0.8) {\large{$+$}};
\node at (-3.6,-0.08) {\large{$-$}};
\node at (5.25,0.75) {$\Delta p$};
\node at (1.75,0.75) {\large{$\omega$}};

\draw  (-3.25,0.5) ellipse (0.25 and 0.25);
\end{tikzpicture}%
  
\caption{The structure of the PI controller.}
\label{fig:simple_PI}
\end{figure}

The output from the MPC is a differential pressure, which is to be controlled through the rotational speed of the pumps. However the pump/valve model in \secref{SystemModel} and seen beneath, is to be used, since the feedback of $\Delta P$ is measured over two valves and a pump, see \figref{fig:subsys_1} in \secref{SystemModel}.

\begin{equation*}
 \Delta p = \Big(\frac{2}{k_{v100}^2} - a_{h2}\Big)|q| q  + a_{h1} \omega_r q + a_{h0}{\omega_r}^2
\end{equation*}

The two terms $\omega$ and $q$ are time dependent variables, which is to be linearized around the operating point. Through a first order Taylor approximation the pump expression end up as see in \eqref{eq:taylor_w_q}

\begin{equation}
	\begin{split}
	\hat{\Delta P} = (4\bar{q}-2\bar{q}a_2+a_1\bar{\omega})\hat{q} + (a_1\bar{q} + 2a_0\bar{\omega})\hat{\omega}
	\end{split}
	\label{eq:taylor_w_q}
\end{equation}


\begin{equation}
	\frac{\hat{\Delta P}}{\hat{\omega}} = ??
\end{equation}