\chapter{Assumption List}
\label{assumptionlist}

\begin{center}
\begin{tabular}{| >{\centering\arraybackslash}m{1in} | >{\centering\arraybackslash}m{3in} | >{\centering\arraybackslash}m{1in} | >{\centering\arraybackslash}m{1in} |}
\hline
\textbf{Number} & \textbf{Assumptions} & \textbf{Section reference} \\
\hline
\multirow{1}{4em}{1}
& The fluid in the network is water. & \secref{PipeModel} \\ 
\hline
\multirow{2}{4em}{2} 
& All pipes in the system are filled up fully with water at all time. & \secref{PipeModel} \\ 
\hline
\multirow{1}{4em}{3} 
& The pipes have a cylindrical structure and the cross section, $A(x)$, is constant for every $x \in [0,L]$.  & \secref{PipeModel} \\ 
\hline
\multirow{1}{4em}{4} 
& The flow of water is uniformly distributed along the cross sectional area of the pipe and the flow is turbulent. & \secref{PipeModel} \\ 
\hline
\multirow{1}{4em}{5} 
& $\Delta z$, the change in elevation only occurs in pipes. & \secref{ValveModel} \\ 
\hline
\multirow{1}{4em}{6} 
& The pumps in the network are centrifugal pumps. & \secref{PumpModel} \\ 
\hline
\multirow{1}{4em}{7} 
& The storage of the WT has a constant diameter. In other words, the walls of the WT are vertical. & \secref{WaterTankModel} \\ 
\hline
\multirow{1}{4em}{8} 
& Valves in the water distribution system are modelled according to the assumption that the length,$L$, is zero. & \secref{ValveModel} \\ 
\hline
\multirow{1}{4em}{9} 
& $\mathcal{G}$ is a connected graph. & \secref{GraphTheory} \\ 
\hline
\multirow{1}{4em}{10} 
& The pipe volume is assumed to be known to an accuracy where there is not any benefit from estimating it. Thereby the estimation problem is simplified. & \secref{SubSecEstimation} \\ 
\hline
\end{tabular}
\captionof{table}{List of assumptions}
\end{center}