\chapter{Pump linearization and PI controller}
\label{cha:linear_pump}

The control structure chosen in \chapref{control_problem} and shown in \figref{fig:control_structure} includes a PI controller, which is to designed. The PI controllers purpose is to use the optimized control output from the MPC as a control reference. 

The control focus of this project has been on the model predictive control, see \secref{sec:MPC}. Therefore a simple PI controller has been design, where the control structure can be seen on \figref{fig:simple_PI}.

\begin{figure}[H]
\centering
\input{report/tikz/pump_lin.tex}  
\caption{The structure of the PI controller.}
\label{fig:simple_PI}
\end{figure}

The output from the MPC is a differential pressure, which is to be controlled through the rotational speed of the pumps. The general model for the pumps is given from \eqref{eq:PumpModel} as 

\begin{equation*}
\Delta p = -a_{h2}{q_i}^2 + a_{h1} \omega_r q_i + a_{h0}{\omega_r}^2
\end{equation*}

which is a nonlinear model. By the assumption that the flow is constant, the expression can through a Taylor expansion be linearized, with respect to $\omega$, to a small signal model.


For simplification the pump model is separated into smaller chunks, see \eqref{eq:tay_1}, then Taylor approximated, see \eqref{eq:tay_2}, where $\omega = \hat{\omega}+\bar{\omega}$.

\begin{equation}
\begin{split}
f_1(\bar{\omega}) &= a_{h1}\bar{q}\bar{\omega} \\
f_2(\bar{\omega}) &= a_{h0}\bar{\omega}^2
\end{split}
\label{eq:tay_1}
\end{equation}


\begin{equation}
\begin{split}
f_{t1}(\omega) &= f_1(\bar{\omega}) + f_1'(\bar{\omega})\cdot(\omega - \hat{\omega}) \\
f_{t2}(\omega) &= f_2(\bar{\omega}) + f_2'(\bar{\omega})\cdot(\omega - \hat{\omega})
\end{split}
\label{eq:tay_2}
\end{equation}

From \eqref{eq:tay_1} an expression for the workspace of, $\bar{\Delta P}$, can be made.

\begin{equation}
\begin{split}
\bar{\Delta P} = f_{1}(\bar{\omega}) + f_{2}(\bar{\omega}) + c
\end{split}
\label{eq:tay_3}
\end{equation}

The linear pump model can then be expressed as seen in \eqref{eq:tay_4}

\begin{equation}
\begin{split}
0 &= -(\bar{\Delta P} + \hat{\Delta P}) + f_{t1}(\bar{\omega}) + f_{t2}(\bar{\omega}) + c\\
  &= -(f_{1}(\bar{\omega}) + f_{2}(\bar{\omega}) + c + \hat{\Delta P}) + f_{t1}(\bar{\omega}) + f_{t2}(\bar{\omega}) + c \\ 
  &= -\hat{\Delta P} + f_1'(\bar{\omega})\cdot\hat{\omega} + f_2'(\bar{\omega})\cdot\hat{\omega}
\end{split}
\label{eq:tay_4}
\end{equation}

\eqref{eq:tay_4} is then Laplace transformed and solved for the input output relationship as seen on \figref{fig:simple_PI}

\begin{equation}
G(s) = \frac{\Delta P}{\omega} = f'_1(\bar{\omega}) + f'_2(\bar{\omega}) = a_{h1}\bar{q} + 2a_{h0}\bar{\omega}
\end{equation}



% \begin{equation}
% \hat{\Delta P} = a_{n1}\cdot\bar{q}\cdot\hat{\omega} + 2a_{n0}\cdot \bar{\omega}
% \label{eq:lin_pump}
% \end{equation}


\subsection*{Controller}

