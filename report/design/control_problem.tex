\section{Control Problem}
\label{control_problem}
%\vspace{-1mm}
The water distribution system explained in \secref{system_overview} is to be controlled according to the requirements descibed in \secref{Requirements_and_constraints}. These can be summarized as: 
%\vspace{-1mm}
\begin{itemize}
	\item Pressure at CP, $0.08 < \pmb{p_{cp}} < 0.14 \:[\text{bar}]$
%
	\item Minimum water exchange through the WT, $\bar{q}_{wt} > 0.06 \: \big[\frac{m^3}{day}\big]$
%
	\item Minimizing the total cost of running the system
\end{itemize}
%\vspace{-1mm}
%The WT is used in the system to reduce the operation cost of the water distribution network. Therefor the system has to be controlled in a way that obtains the lowest cost of operation. 

The system consists of four pumps, two local PMA pumps and two main pumps. Because the focus of this project is to balance between the use of the mains pumps and the WT a simple controller is designed to keep a specific differential pressure over the PMA pumps. Thereby can the system be fully controlled by the main pumps as the PMA pumps only influence the operating point with a fixed pressure lift.
% to minimize the cost of running the system

To ensure cost reducing control a more advanced controller, controlling the main pumps, is designed. This is done by using Model Predictive Control, MPC. The advantage of MPC, compared to eg. LQR is that the MPC can optimize a state by taking the expectation of the future electrical pricing into account. 

The MPC could be used to control the main pumps directly, but to mimic a real world scenario it is chosen to let the MPC set differential pressure references to the main pumps. As a consequence of this structure a controller is required to ensures that the pumps deliver the, by the MPC, specified differential pressure. An advantage of this structure is that the pump controllers can be placed locally at each pump and in an event of failing communication between the main MPC and the pump controllers, the pumps will still be operational. 
This cascade structure is shown in \figref{fig:control_structure}.  
\todo{Carsten comment: Typically the flow out of the pumping station is controlled. Here you use the pressure with has some obvious advantages: - In the case of a system failure it is the pressure that is important. The flow will typically either lead to too high or too low pressure, or tank run over. This is an important change, which you should emphasize.}
%\vspace{-1mm}
\begin{figure}[H]
\centering
\begin{tikzpicture} [scale=0.7,transform shape]

\draw  (6,1) rectangle (8,0);
\node at (7,0.5) {Sys};

\draw  (-4.5,1) rectangle (-2.5,0);
\node at (-3.5,0.5) {MPC};

\draw  (2.5,1) rectangle (4.5,0);
\node at (3.5,0.5) {Pump};

\draw  (-1,1) rectangle (1,0);
\node at (0,0.5) {PI};

\draw[-triangle 60] (8,0.25) -- (9.5,0.25);
\draw[-triangle 60] (8,0.75) -- (9.5,0.75);
\draw[-triangle 60] (4.5,0.5) -- (6,0.5);
\draw[-triangle 60] (1,0.25) -- (2.5,0.25);
\draw[-triangle 60] (-2.5,0.5) -- (-1,0.5);
%\draw[-triangle 60] (-5.5,0.5) -- (-4.5,0.5);

\draw[-triangle 60] (5.25,0.5) -- (5.25,-1) -- (0,-1) -- (0,0);
\draw[-triangle 60] (8.5,0.75) -- (8.5,2) -- (1.75,2) -- (1.75,0.75) -- (2.5,0.75);
\draw[-triangle 60] (8.5,0.25) -- (8.5,-1.5) -- (-3.5,-1.5) -- (-3.5,0);

%\node at (-5.5,1) {$CP_{ref}$};
\node at (-1.75,1) {$u(k)$};
\node at (5.25,1) {$\Delta p$};
\node at (1.75,0) {\large{$\omega$}};
\node at (9,0) {${y}$};
\node at (9,1) {$q$};

\end{tikzpicture}%
 
\caption{Cascade control structure with MPC and pump controller.}
\label{fig:control_structure}
\end{figure}
%\vspace{-1mm}
%% Comment Carsten 
%Typically the flow out of the pumping station is controlled. Here you use the pressure with has some obvious advantages:
%- In the case of a system failure it is the pressure that is important. The flow will typically either lead to too high or too low pressure, or tank run over. 

%This is an important change, which you should emphasize. 
%%

In the cascade control structure a local PI controller is used to control the differential pressure output of each main pump, a copy of the constructed PI controller will be used at each main pump as these are identical.

\todo{Vinkel, correction needed here}

%For the MPC the systems operating area, $C_T(\pmb{\Delta{p}},\pmb{q},\Gamma(k))$ has to be described. 
Then considering the total cost of running the system multiple things have to be taken into consideration. Both the model of the water distribution network, the electrical consumption of the pumps, the electrical pricing and the constrains of the system have to be taken into account. Furthermore a constrain securing that the WT do not run dry or overflow is included. By considering the total cost of running, $C_T$ this can be seen as a minimization problem:
\todo{The output \textbf{y} is the 8 pressure measurements, here also including $p_{cp}$ and $\Delta{p}_i$ so the notation here need to correcting.}
\todoque{Which things depend on time, k ?}
\begin{align}
\underset{\pmb{u}}{min} \: C_T(\pmb{\Delta{p}},\pmb{q},\Gamma(k)) &= \underset{\pmb{u}}{min} \: \sum_{k=1}^{P_H} \sum_{i=1}^{N} P_{e_{i}}(\Delta{p}_i,q_i) \cdot \Gamma(k)  \\
%
s.t \:\:\:\:\: & \Delta \dot{p}_{wt} = \pmb{M} \Delta \hat{p}_{wt}  + \pmb{N}\pmb{\hat{u}} \\
%
&\pmb{y} = \pmb{C} \Delta \hat{p}_{wt}  + \pmb{D}\pmb{\hat{u}} \\
%
& \underline{p_{cp}} < \pmb{p_{cp}} < \overline{p_{cp}} \\
%
& \underline{p_{wt}} < p_{wt} < \overline{p_{wt}} \\
%
& \bar{q}_{wt} > \underline{\bar{q}_{wt}}
\end{align}

 \begin{minipage}[t]{0.24\textwidth}
 Where\\
 \hspace*{8mm} $C_T(\pmb{\Delta{p}},\pmb{q},\Gamma(k))$ \\
 \hspace*{8mm} $P_{e_{i}}(\Delta{p}_i,q_i)$ \\
 \hspace*{8mm} N \\
 \hspace*{8mm} $P_H$ \\
 \hspace*{8mm} $\Gamma(k)$ \\
 \hspace*{8mm} $\pmb{p_{cp}}$ \\
% \hspace*{8mm}  \textcolor{White}{te}\\
 \hspace*{8mm} $p_{wt}$ \\
% \hspace*{8mm}  \textcolor{White}{te}\\
  and \hspace*{0.7mm} $\bar{q}_{wt}$	
 \end{minipage}
 \begin{minipage}[t]{0.63\textwidth}
 \vspace*{2mm}
 is the cost of running the system, \\
 is the power consumption of the $i^{th}$ pump,\\
 is the number of pumps,\\
 is the prediction horizon,\\
 is the price of electricity at time k,\\ 
 is the pressure at a given CP,\\ %, $\overline{p_{cp}}$ and $\underline{p_{cp}}$ are the upper and lower bounds
 is the pressure at the WT outlet,\\ %, $\overline{p_{wt}}$ and $\underline{p_{wt}}$ are the upper and lower bounds,
 is the average flow rate through the WT. %, $\underline{\bar{q}_{wt}}$ is the lower bound
 \end{minipage}
 \begin{minipage}[t]{0.10\textwidth}
 \vspace*{2mm}
 \textcolor{White}{te}$\unit{DKK}$\\
 \textcolor{White}{te}$\unit{W}$\\
 \textcolor{White}{te}$\unit{\cdot}$\\
 \textcolor{White}{te}$\unit{\cdot}$\\
 \textcolor{White}{te}$\unit{{DKK}/{MW}}$\\
 \textcolor{White}{te}$\unit{Bar}$\\
% \hspace*{8mm}  \textcolor{White}{te}\\
 \textcolor{White}{te}$\unit{Bar}$\\
% \hspace*{8mm}  \textcolor{White}{te}\\
  \textcolor{White}{te}$\unit{{m^3}/{day}}$
 \end{minipage}

The dynamic model of the system is explained in \chapref{modelling} and linearized in \secref{LinParamEst}. The electrical price, $\Gamma(k)$, is described in \secref{sec:cost_fkt}. From this the input vector $\pmb{u}$ should be obtained, so the cost of running the system, within the bounds of the constrains, is minimized. 



In the following sections the design of the descried control system is explained. 


%The purpose is to let the WT do the work when the price of running the pumps are high.

%\begin{align*}
%\pmb{B}
%\end{align*}


% The pressure at a given CP can be controlled by the water level in the WT which is controlled by the rotational speed of the pumps. To fulfill the requirement of a minimum pressure at a given CP the water level in the WT have to be within a operations area. A controller have to be develop that takes the pressure actuators into account. 

% The flow through the WT is controlled by pressure in the outer ring. Whenever the pressure is higher in the outer ring than in the WT, water is being pump into the WT. Furthermore the average flow rate, $\bar{q}_{wt}$, through the WT must meet the minimum requirement to insure the water quality. This can be seen as a constrain of the operation area of the system. 

% At the same time the total cost of running the system should be minimized. Therefor a cost function is needed. This cost function purpose is to find the optimal control signal which minimize the cost of running the pumps. Thereby spending the least money on running the total system. 