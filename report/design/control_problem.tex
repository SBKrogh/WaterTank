\section{Control Problem}
\label{control_problem}
%\vspace{-1mm}
The water distribution system described in \secref{system_overview} is to be controlled with respect to the requirements and constraints stated in \secref{Requirements_and_constraints} and the dynamics of the system described in \secref{SystemLin_control}. For a better clearance, the requirements and constraints are listed again below: 
%\vspace{-1mm}
\begin{itemize}
	\item Consumer pressure requirements, $0.08 < \pmb{y} < 0.14 \:[\text{bar}]$
%
	\item Minimizing the total running costs
%
	\item Performance constraints on the pumps
\end{itemize}
%\vspace{-1mm}
%The WT is used in the system to reduce the operation cost of the water distribution network. Therefor the system has to be controlled in a way that obtains the lowest cost of operation. 
The system consists of four pumps and is fully controlled by the two main pumps. The two pumps in the PMAs influence only the operating point with a fixed pressure lift. The project deals with a control that optimizes the WT such that the pressure at the end-users is maintained and the cost of pumping effort in the main pumps is minimised. In other words, the use of the main pumps and the WT is controlled according to the constraints and requirements stated above. 

In order to achieve such an optimal behavior, Model Predictive Control, MPC, is applied. With MPC, the dynamics of the model is used to predict the system behavior, therefore taking the future electrical prices into account. 

Apart from the MPC control, all pumps are controlled by PI controllers in order to deliver the required minimum pressure to the end-users. The MPC could be used to control the main pumps directly, but to mimic a real world scenario, without PMA pumps, it is chosen to let the MPC set differential pressure references to the main pumps. As a consequence of the two control method, the following cascade structure can be formulated for the two main pumps as it is shown in \figref{fig:control_structure}: 

%However, as it is explained above, the two main pumps are also controlled by MPC. 

\begin{figure}[H]
\centering
\begin{tikzpicture} [scale=0.7,transform shape]

\draw  (6,1) rectangle (8,0);
\node at (7,0.5) {Sys};

\draw  (-4.5,1) rectangle (-2.5,0);
\node at (-3.5,0.5) {MPC};

\draw  (2.5,1) rectangle (4.5,0);
\node at (3.5,0.5) {Pump};

\draw  (-1,1) rectangle (1,0);
\node at (0,0.5) {PI};

\draw[-triangle 60] (8,0.25) -- (9.5,0.25);
\draw[-triangle 60] (8,0.75) -- (9.5,0.75);
\draw[-triangle 60] (4.5,0.5) -- (6,0.5);
\draw[-triangle 60] (1,0.25) -- (2.5,0.25);
\draw[-triangle 60] (-2.5,0.5) -- (-1,0.5);
%\draw[-triangle 60] (-5.5,0.5) -- (-4.5,0.5);

\draw[-triangle 60] (5.25,0.5) -- (5.25,-1) -- (0,-1) -- (0,0);
\draw[-triangle 60] (8.5,0.75) -- (8.5,2) -- (1.75,2) -- (1.75,0.75) -- (2.5,0.75);
\draw[-triangle 60] (8.5,0.25) -- (8.5,-1.5) -- (-3.5,-1.5) -- (-3.5,0);

%\node at (-5.5,1) {$CP_{ref}$};
\node at (-1.75,1) {$u(k)$};
\node at (5.25,1) {$\Delta p$};
\node at (1.75,0) {\large{$\omega$}};
\node at (9,0) {${y}$};
\node at (9,1) {$q$};

\end{tikzpicture}%
 
\caption{Cascade control structure with MPC and PI controllers.}
\label{fig:control_structure}
\end{figure}

As it is shown, the PI controllers are responsible for setting the control inputs to the main pumps, using the feedback from the differential pressure output. It should be noted however, that typically the flow out of the pumping station is controlled instead of the pressure. Here the use of pressure feedback has the advantage that in case of system failure the pressure is important. The flow would typically either lead to too high or too low pressure or would make the WT overrun. On the other hand, the reference to the PI controllers is set by the MPC algorithm which takes the measurements from the output into account. These references are set as differential pressures. One of the advantage of such a cascade control is that the pump controllers are present for each pump, also including the PMA pumps. At the event of implementation or communication failure between the MPC and the pump controllers, the pumping remains operational. 
 
%\vspace{-1mm}

%\vspace{-1mm}
%% Comment Carsten 
%Typically the flow out of the pumping station is controlled. Here you use the pressure with has some obvious advantages:
%- In the case of a system failure it is the pressure that is important. The flow will typically either lead to too high or too low pressure, or tank run over. 

%This is an important change, which you should emphasize. 
%%

%For the MPC the systems operating area, $C_T(\pmb{\Delta{p}},\pmb{q},\Gamma(k))$ has to be described. 
In order to define an objective function for the MPC problem, the cost of running the system has to be taken into account along with the hydraulic power consumption of the pumps. Such an objective function can be written in the form: 

\begin{align}
 \Upsilon(\pmb{u}[k],\pmb{q_p}[k],c[k]) &=  \sum_{i=0}^{H_p-1} \sum_{j=1}^{N} \Big(\frac{1}{\eta_j} \cdot u_j[k+i|k] \cdot  q_{p;j}[k+i|k]\Big)\cdot c[k+i|k] \label{eqcost} 
\end{align}

%Then considering the total cost of running the system multiple things have to be taken into consideration. Both the model of the water distribution network, the electrical consumption of the pumps, the electrical pricing and the constraints of the system have to be taken into account. Furthermore a constraints securing that the WT do not run dry or overflow is included. By considering the total cost of running, $C_T$, this can be seen as a minimization problem:
%\todo{The output \textbf{y} is the 8 pressure measurements, here also including $p_{cp}$ and $\Delta{p}_i$ so the notation here need to correcting.}
%\todo{Furthermore the model have to be discreet and we need to correct $\Delta p$ compared to u.}

\begin{minipage}[t]{0.20\textwidth}
Where\\
\hspace*{8mm} $\pmb{u}$ \\
\hspace*{8mm} $\pmb{q_p}$ \\
 \hspace*{8mm} $\eta_i$ \\
\hspace*{8mm} $c[k]$ 
\end{minipage}
\begin{minipage}[t]{0.68\textwidth}
\vspace*{2mm}
is the differential pressure input to the main pumps,\\
is the flow through the main pumps,\\
is the efficiency of the $i^{th}$ main pump,\\
is the electricity price cost sequence.
\end{minipage}
\begin{minipage}[t]{0.10\textwidth}
\vspace*{2mm}
\textcolor{White}{te}$\unit{bar}$
\textcolor{White}{te}$\unit{m^3/h}$
 \textcolor{White}{te}$\unit{\cdot}$\\
\textcolor{White}{te}$\unit{DKK/MW}$
\end{minipage}

And the control problem can be formulated as a minimization problem on the form:
%\pmb{u}[k],\pmb{q_p}[k],c[k]
\begin{align}
\underset{\pmb{u}}{min} \:  \Upsilon(\cdot) &= \underset{\pmb{u}}{min} \: \sum_{i=0}^{H_p-1} \sum_{j=1}^{N} \Big(\frac{1}{\eta_j} \cdot u_j[k+i|k] \cdot q_{p;j}[k+i|k]\Big)\cdot c[k+i|k] \label{eqcost1} \\
%
\newline
s.t \:\:\:\:\: & \Delta \hat{p}_{WT} [k+i+1|k] = A_d \Delta \hat{p}_{WT}[k+i|k]  + \pmb{B_d} \pmb{\hat{u}}[k+i|k] \\ &+ \pmb{E_d} \pmb{\hat{d}}[k+i|k]  \\
%
&\pmb{\hat{y}}[k+i|k] = \pmb{C_d} \Delta \hat{p}_{WT}[k+i|k] + \pmb{D_d} \pmb{\hat{u}}[k+i|k] \\ &+ \pmb{K_d} \pmb{\hat{d}}[k+i|k] \\
%
& \underline{y} < \pmb{y} < \overline{y} \\
%
& \underline{\Delta p_{WT}} < \Delta p_{WT} < \overline{\Delta p_{WT}} \\
%
& \pmb{u} < \overline{u} \label{eqconst2} 
\end{align}

 \begin{minipage}[t]{0.24\textwidth}
 Where\\
 \hspace*{8mm} $Hp$ \\
 \hspace*{8mm} $N$ \\
 \hspace*{8mm} $\pmb{y}$ \\
 \hspace*{8mm} $\Delta p_{WT}$ 
 \end{minipage}
 \begin{minipage}[t]{0.63\textwidth}
 \vspace*{2mm}
 is the control horizon, \\
 is the number of main pumps,\\
 is the output vector with the PMA pressures,\\
 is the state which is the pressure in the WT.
 \end{minipage}
 \begin{minipage}[t]{0.10\textwidth}
 \vspace*{2mm}
 \textcolor{White}{te}$\unit{\cdot}$\\
 \textcolor{White}{te}$\unit{\cdot}$\\
 \textcolor{White}{te}$\unit{bar}$\\
 \textcolor{White}{te}$\unit{bar}$
 \end{minipage}
 
As it is shown in \eqref{eqcost1} the optimization is about to obtain an input signal such as the cost of running the pumps is minimized. Therefore the objective function gives a price of all the consumed power over the control horizon, which is a specific future time interval. Furthermore, it should be noted that both the objective function and the constraints are linear, therefore the problem is a linear programming formulation. The minimization is subject to the dynamics of the water distribution network, and the constraints. As can be seen, there is a constraint on the output of the system, which is considered as the pressure demand for the end-users. There is a constraint on the pressure in the WT which represents the states in the dynamics and furthermore, there is a constraint on the input signal to the pumps. 
\\
The dynamic model of the system is explained in \chapref{modelling} and linearized in \secref{LinParamEst}. The electrical price, $c[k]$, is described in \appref{sec:cost_fkt}. %From this the input vector $\pmb{u}$ should be obtained, so the cost of running the system, within the bounds of the constrains, is minimized. 

In the following sections the design of the described control system is explained. 


%The purpose is to let the WT do the work when the price of running the pumps are high.

%\begin{align*}
%\pmb{B}
%\end{align*}


% The pressure at a given CP can be controlled by the water level in the WT which is controlled by the rotational speed of the pumps. To fulfill the requirement of a minimum pressure at a given CP the water level in the WT have to be within a operations area. A controller have to be develop that takes the pressure actuators into account. 

% The flow through the WT is controlled by pressure in the outer ring. Whenever the pressure is higher in the outer ring than in the WT, water is being pump into the WT. Furthermore the average flow rate, $\bar{q}_{wt}$, through the WT must meet the minimum requirement to insure the water quality. This can be seen as a constrain of the operation area of the system. 

% At the same time the total cost of running the system should be minimized. Therefor a cost function is needed. This cost function purpose is to find the optimal control signal which minimize the cost of running the pumps. Thereby spending the least money on running the total system. 