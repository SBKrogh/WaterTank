\chapter{Control System Implementation}
\label{implementation_of_controller}

This chapter will explain how the controller designed in \chapref{Controller} is implemented in MATLAB Simulink.  

\section{}
It is worth mentioning that for the implementation of the control the second linear model attempt is used, as it is the most suitable model to work with. The MPC controller is implemented in MATLAB Simulink, which is connected to the model as shown in \figref{fig:control_sketch}:

\begin{figure}[H]
\centering
\begin{tikzpicture} [scale=0.67,transform shape]

\draw  (10,2) rectangle (6,0);
\node at (8,1) {Model};
\node at (9.7,1.7) {y};
\node at (9.7, 0.3) {x};
\node at (6.3, 1.7) {d};
\node at (6.3, 0.3) {u};
\draw[-triangle 60] (10, 1.7) -- (11, 1.7);

\draw[-triangle 60] (10, 0.3) -- (11, 0.3) -- (11,-3) -- (9,-3);



\draw  (9,-2) rectangle (7,-4);
\node at (8,-3) {MPC};
\node at (6.5,-2.55) {$\bm{U_{Hp}}$};
\draw  (7,-3) -- (6,-3);
\draw (6,-3.3) rectangle (4,-2.7);
\draw (8.3,-2) rectangle (7.7,-2.5);
\node at (8, -2.2) {$T_s$};

\draw  (5.8,-3.3) -- (5.8,-2.7);
\draw  (5.6,-3.3) -- (5.6,-2.7);
\draw  (5.4,-3.3) -- (5.4,-2.7);
\draw  (5.2,-3.3) -- (5.2,-2.7);
\draw  (5.0,-3.3) -- (5.0,-2.7);
\draw  (4.8,-3.3) -- (4.8,-2.7);
\draw  (4.6,-3.3) -- (4.6,-2.7);
\draw  (4.4,-3.3) -- (4.4,-2.7);
\draw  (4.2,-3.3) -- (4.2,-2.7);
\fill[ForestGreen] (4.18,-3.29) rectangle(4.01,-2.71);
\draw[-triangle 60] (4, -3) -- (3, -3) -- (3,0.3) -- (6,0.3);
\node at (3.9,-1.5) {$\bm{U_{Hp}}(1,1)$};
\draw[-triangle 60] (5,1.7)--(6,1.7);

 
 	\draw (0,1.7) -- (4.3,1.7);
 	\node [font=\tiny] at (4.3,1.5) {t};
    \draw (0,2.4)node[left,font=\tiny] {$OD$} -- (0,1.0);
    % \draw (0.2,-1)node[left,font=\tiny] {$y=-1$} -- (11.8,-1); 
    % \foreach \x in {0,0.5,...,12}{
    % \draw (\x,-0.2)node [below,font=\tiny,] {\x} -- (\x,0.2) ;
    % }
    \draw[ thick, red] (0,1.7) sin (0.7,2.4);    %% the real business in this line
    \draw[ thick, red] (0.7,2.4) cos (1.4,1.7);    %% the real business in this line
    \draw[ thick, red] (1.4,1.7) sin (2.1,0.9);    %% the real business in this line
    \draw[ thick, red] (2.1,0.9) cos (2.8,1.7);    %% the real business in this line
    \draw[ thick, red] (2.8,1.7)  sin (3.5,2.4);    %% the real business in this line
    \draw[ thick, red] (3.5,2.4) cos (4.2,1.7);    %% the real business in this line
    % \draw[ultra thick, red] (9,0) sin (10,-1);    %% the real business in this line
    % \draw[ultra thick, blue] (10,-1) cos (11,0);    %% the real business in this line

\end{tikzpicture}%


 
\caption{Sketch of the control implementation.}
\label{fig:control_sketch}
\end{figure}

In order to solve the MPC and find a global minimum for the problem specified in \eqref{eq:obj_final1},  MATLAB quadratic solver \textit{quadprog} is used. This solver, finds out the preceding problem subject to the specified constraints and restrictions set to the convex function. The interior-point-convex algorithm is chosen to carry out the calculation of the control inputs. This algorithm starts solving the problem by simplifying the constraints, and then it iterates in the interior of the feasible set and it then progressively adds the constraints and reconstructs the original problem \cite{Convex_optimization}. 


In previous sections the number of predictions, $H_p$, is introduced. It is decided to predict 24 hour forward due to the periodicity of the daily rhythm of the population. Therefore, the control input to the state-space and the disturbance added by the end-users consumption are updated every hour. This sampling time for the control input will differ from the sample time determined for the state space model, $T_s = 6.4s$. 

In \appref{sec:cost_fkt}, the electric price model is shown. Nevertheless, for the implementation of the control not only the electric price model is needed but also a model for the end-users water consumption. The end-users are assumed to have the same behavior in the water consumption as the price has throughout the day.

