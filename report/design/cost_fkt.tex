\section{Cost function}

To minimize the running cost of the system a cost function of the electrical price is needed. To predict future prices a simple moving average (SMA) is used.
This approach have been chosen, due to the fact that the learning goals of this project is not to derive a high precision predictive model that describe future electrical prices.

The SMA uses present and previous data samples to calculate future predictions, this can be expressed as seen in \eqref{equ:MA}.

\begin{equation}
x(k+1) = \frac{1}{N}\sum\limits_{k=0}^{N-1} x(-k)
\label{equ:MA}
\end{equation} 

The SMA can not take non-stationary processes into account, so if sudden changes in the price appears, the future estimates will be less precise. From \cite{Electrical_price} data price over the present has been gathered and can be seen on \figref{fig:electrical_price} together with the SMA model which utilize the previous data sample and the present. 

\begin{figure}[H]
\centering
\input{report/pictures/SMA_prediction}
\caption{Two graphs that shows the price of electricity and the SMA prediction of the price.}
\label{fig:electrical_price} 
\end{figure}

It can be seen that the SMA is not a precise predictor and have a delay of two hours. However it still keeps the dynamics of the price, which is deemed enough for this project.
