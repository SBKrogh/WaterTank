\section{Model predictive control}

Model predictive control (MPC) is an advance control method that depends on a dynamic model of the plant. MPC allows to calculate a optimal control signal taking the future into account. The control problem is explained in \secref{control_problem}.

\subsection{Constraints}

When designing a MPC the constraints have to be setup so they represent actuator slew rates, actuator ranges and constraints for the control variables. These constraints are setup as matrices called $\pmb{E}, \pmb{F} \text{and} \pmb{G}$. The constraints matrices are given by: 

\begin{align}
E \cdot (\Delta\hat u(k|k),...,\Delta\hat u(k+H_u-1|k),1) &\leq 0 \\
%
F \cdot (\hat u(k|k),...,\hat u(k+H_u-1|k),1) &\leq 0 \\
%
G \cdot (\hat z(k+H_w|k),...,\hat z(k+H_p|k),1) &\leq 0
\end{align}


 \begin{minipage}[t]{0.20\textwidth}
 Where\\
 \hspace*{8mm} $\hat u(k|k)$ \\
 \hspace*{8mm} $\Delta\hat u(k|k)$ \\
 \hspace*{8mm} $H_p$ \\
 \hspace*{8mm} $H_u$ \\
 \hspace*{8mm} $H_w$ \\
 and \hspace*{0.7mm} $\hat z(k+H_w|k)$	
 \end{minipage}
 \begin{minipage}[t]{0.68\textwidth}
 \vspace*{2mm}
 is the input vector, \\
 is the change of the input vector, \\
 is the prediction horizon,\\
 is the control horizon, \\
 is the window horizon, \\
 is the estimate of the controlled output to the time $k+H_w$ compared to current output.
 \end{minipage}

The actuator slew rate constrain, matrix $\pmb{E}$, determents how fast the actuator can change per time unit. From this the physical limit of the pumps can be described \todo{specify when we got a better idea of the limits of the pumps}. The actuator ranges describe how the control signal to the pumps should look. In the case of this project the pumps can be controlled with an input from 0 to 5. 

The constraints on the control variables is the constraints setup in \secref{control_problem}. 

To calculate the constraint matrices $H_p, H_u \text{and} H_w$ need to be determent. 