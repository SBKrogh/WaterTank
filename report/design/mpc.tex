\section{Model predictive control}

Model predictive control (MPC) is an advance control method that depends on a dynamic model of the plant. MPC allows to calculate an optimal control signal taking the future electrical pricing into account. In this section the objective function, describing the electrical price, will be setup. Furthermore the MPC will be designed. 

\section{Cost function}

To minimize the running cost of the system a cost function of the electrical price is needed. Predicting future prices is an extensive task that depends on many factors e.g user consumption and weather conditions. Due to the fact that the learning goals of this project is not to derive a high precision predictive model that describe future electrical prices, data gathered from \cite{Electrical_price} is used instead. These data is used to represent the output of a high quality cost function, where 60 hours can be seen on \figref{fig:electrical_price}. 


% Due to the fact that the learning goals of this project is not to derive a high precision predictive model that describe future electrical prices, a simple model that mimics real electrical pricing is created. This model is based on electrical pricing in Denmark from the period 27-03-2017 to 02-04-2017.  


% To predict future prices a simple moving average (SMA) is used.
% This approach have been chosen, .

% The SMA uses present and previous data samples to calculate future predictions, this can be expressed as seen in \eqref{equ:MA}.
	
% \begin{equation}
% x(k+1) = \frac{1}{N}\sum\limits_{k=0}^{N-1} x(-k)
% \label{equ:MA}
% \end{equation} 

% The SMA can not take non-stationary processes into account, so if sudden changes in the price appears, the future estimates will be less precise. From \cite{Electrical_price} data price over the present has been gathered and can be seen on \figref{fig:electrical_price} together with the SMA model which utilize the previous data sample and the present. 

\begin{figure}[H]
\centering
\input{report/pictures/SMA_prediction}
\caption{Graph of electricity prices from Denmark from the 27-03-2017 to 29-03-2017.}
\label{fig:electrical_price} 
\end{figure}

%The graph seen on \figref{fig:electrical_price}  shows an interval of a 60 hours from the 27-03-2017 to 29-03-2017 in Denmark.
% Even though the pricing seen in \figref{fig:electrical_price} is not a prediction of the future electrical pricing it is assumed to be a good representation of the pricing and therefor it will be used for this project. 

% It can be seen that the SMA is not a precise predictor and have a delay of two hours. However it still keeps the dynamics of the price, which is deemed enough for this project.


\subsection{Constraints}

When designing a MPC the constraints have to be setup so they represent actuator slew rates, actuator ranges and constraints for the control variables. These constraints are setup as matrices called $\pmb{E}, \pmb{F} \text{and} \:\pmb{G}$. The constraints matrices are given by: 

\begin{equation}
\pmb{E} \cdot [\Delta\tilde u(k|k),...,\Delta\tilde u(k+H_u-1|k),1]^T \leq 0 
\label{eq:slewrate}
\end{equation}
\begin{equation}
\pmb{F} \cdot [\tilde u(k|k),...,\tilde u(k+H_u-1|k),1]^T \leq 0 
\label{eq:actranges}
\end{equation}
\begin{equation}
\pmb{G} \cdot [\tilde y(k+H_w|k),...,\tilde y(k+H_p|k),1]^T \leq 0
\label{eq:controlvar}
\end{equation}


 \begin{minipage}[t]{0.20\textwidth}
 Where\\
 \hspace*{8mm} $\tilde u(k|k)$ \\
 \hspace*{8mm} $\Delta\tilde u(k|k)$ \\
 \hspace*{8mm} $H_u$ \\
 \hspace*{8mm} $H_w$ \\
 and \hspace*{0.7mm} $\tilde y(k+H_w|k)$	
 \end{minipage}
 \begin{minipage}[t]{0.68\textwidth}
 \vspace*{2mm}
 is the input vector, \\
 is the change of the input vector, \\
 is the control horizon, \\
 is the window horizon, \\
 is the estimate of the controlled output to the time $k+H_w$ compared to current output.
 \end{minipage}

The actuator slew rate constrain, matrix $\pmb{E}$, determents how fast the actuator can change per time unit. From this the physical limit of the pumps can be described. The actuator ranges describe how the control signal to the pumps should look. This is simply a way to describe 0 to 100 \% performance. In the case of this project the pumps can be controlled with an input from 0 to 5. The constraints on the control variables is the constraints setup in \secref{control_problem}. \todo{This part could properly be specified better.}

%To easy notation $U(k) = $

%As it can be seen from \eqref{eq:slewrate,eq:actranges,eq:controlvar} depend on three different variables $\hat u(k|k), \Delta\hat u(k|k) \text{and} \hat z(k+H_w|k)$

To calculate the constraint matrices $H_p, H_u \text{and} \: H_w$ need to be determent. In \secref{sec:cost_fkt} the electrical price is descried. From this it can be seen that the price fluctuate a lot but some periodicity can be seen every 24 hours. Therefor both $H_p$ and $ H_u $ is set to 24. 

\todoque{How sould we chose $H_w$}

% The plan from here on is: 

% 1: Describe the transformation of the constraints so they all depend on delta U(k) 

% 2: Explain how the minimization problem can be set up in general and then s.t. the new constrains 

% 3: Cover why we need an observer and what it means for how the minmization problem is set up - U(k) -> \hat U(k|k)
