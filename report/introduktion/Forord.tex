% \chapter*{Preface}

% \textcolor{red}{This project comprises of implementing a functional controller system for the provided miniature model crane. This is conducted with the purpose of relocating containers. A basic understanding of electronics is expected of the reader. The report is constructed with different chapters, each covering a main part of the entire project. First, a thorough analysis of the crane setup and the modernizations carried out to the hardware platform is presented. Furthermore, the construction of models describing the crane and verification of these is presented. The second part consists of designing controllers based on requirements derived from an earlier chapter. The controllers are implemented and tested. Finally an acceptance test is performed and this is documented in the appendix. All of the raw test information, test programs and relevant datasheets are stored on the enclosed CD. Regarding figures without a source reference, these are created specifically for this report. A big thanks to the group's supervisors Tom S. Pedersen and Kirsten M. Nielsen for help with discussing possible solutions and giving insight into the current solutions in use, concerning classical control theory. Thanks to Rasmus Viking Lømand Ravgård Christiansen, bachelor in mechanical engineering, for helping creating a \textbf{Solid Work} \cite{Solid_Work} model for parts of the crane. }

% \begin{flushright}
% Aalborg University, 20th of December 2016
% \end{flushright}







\begin{table}[H]
	\centering
	\vspace{50em}
		\begin{tabular}{c c }
			\underline{\phantom{mmmmmmmmmmmmmmmmmmm}}       & \underline{\phantom{mmmmmmmmmmmmmmmmmmm}} \\
			Dániel Bolgár			 & Simon Bjerre Krogh  \\
			\textit{dbolga16@student.aau.dk} & \textit{skrogh13@student.aau.dk}\\
			&\\
			&\\
			\underline{\phantom{mmmmmmmmmmmmmmmmmmm}}       & \underline{\phantom{mmmmmmmmmmmmmmmmmmm}} \\
		    Filip Marić			 & Nicolas Silvani\\
			\textit{fmaric16@student.aau.dk} & \textit{nsilva16@student.aau.dk} \\
			&\\
			&\\	
		\end{tabular}
\end{table}